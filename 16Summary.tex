\subsection{Summary and outline}\label{sec:160}

This introduction has served to illuminate the concerns central to the biolinguistic research programme in which context this thesis is situated. It has provided some novel synthesis of the most recent ideas in MP and provided broader comment on the ways in which the Galilean challenge can be tackled.

Labelling itself is to receive more attention in the following sections. \autoref{sec:200} constitues a historical review of approaches to labelling, beginning with the earliest work in generative grammar. The conclusions from this section frame the discussion in \autoref{sec:300}, which unravels the central issues within contemporary approaches to labelling. The formalisation itself comes in \autoref{sec:400}, which will extend \citeposs{CollinsC.StablerE_2016} formalisation of Minimalist syntax, including a novel definition of MS and, in turn, the labelling algorithm. The overarching goal is to create a formal model of syntax that is both internally consistent and has the potential to provide genuine explanation of linguistic phenomena. As will become clear in the following sections, a precise understanding of labelling is essential to this.
