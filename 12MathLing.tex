\subsection{Formalisation and mathematical linguistics}\label{sec:120}

To adopt the concise phrasing of Chris Collins (p.c.), formal work is needed in the domain of labelling in particular, ``since otherwise we don't really understand what we are doing''. It is not, though, a mischaracterisation to say that formalisation has been a concern of the generative enterprise since its inception. One need only look at one of the foundational documents of the enterprise to find the view expressed that there is an intra-theoretical benefit of formalisation: ``a formalised theory may automatically provide solutions for many problems other than those for which it was explicitly designed'', avoiding reliance upon ``[o]bscure and intuition-bound notions'' \parencite[5]{ChomskyN_1957}. This being said, there remains only a very small component of the literature that focuses on formalisation. Using the terminology established in \autoref{sec:110}, much theoretical work instead centres on computational concerns \pxref{ex:marr:1} as opposed to algorithmic details \pxref{ex:marr:2}, and Chomsky's own output often drifts into the metatheoretical, which is wholly on the computational level, dealing as it does with more abstract ontological concerns. Attempts to provide a complete summary of the theory are typically pedagogical \parencite[for example][]{AdgerD_2003, RadfordA_2004a,HornsteinN.etal_2005,SporticheD.etal_2014}, and tend to struggle to extract a completely coherent theory from the Minimalist literature \parencite{AsudehA.ToivonenI_2006}. \textcite{CollinsC.StablerE_2016} provide the partial formalisation of Minimalist syntax which forms the bedrock of the present work. Nevertheless, and by their own admission, their formalisation is incomplete, covering only a small subset of the full range of operations typically assumed in the analytical literature. Furthermore, some of its proposals will necessarily require revision in order to be in line with recent developments, notably those of \textcite{ChomskyN_2019a,ChomskyN_2021}, as will be discussed in \autoref{sec:400}.

When it comes to formalising the theory of syntax there are effectively two approaches that can be taken. The first is typified by the field of \textit{mathematical linguistics}, perhaps more accurately characterised a subfield of mathematics rather than linguistics, which seeks to uncover properties of the formal languages originally developed out of linguistic theory. Models within mathematical linguistics may depend to varying extents on empirical considerations, and the focus is rather on the formal and computational characteristics of the grammars being studied. The approach dates back to the earliest work in generative grammar \parencite{ChomskyN_1956, ChomskyN.MillerGA_1963, MillerGA.ChomskyN_1963}, and was extended into the Minimalist era by \textcite{StablerE_1997}, who formalised part of the new, feature-driven, derivational theory presented by \textcite{ChomskyN_1995}. \textcite{StablerE_1997} established the formal Minimalist Grammar {MG}, which was subject to considerable further investigation \parencite[e.g.][]{GrafT_2013}. MGs typically adopt numerous conventions which depart from standard Minimalist assumptions, including, but not limited to, being `label-free', encoding linear order, and being necessarily endocentric. Such mathematical work will not receive much more consideration here.

The second approach is the one taken by \textcite{CollinsC.StablerE_2016}, and is the one adopted in the present work. Unlike MGs, which ``were simplified to facilitate computational assessment'', this approach sets out to ``give a precise, formal account of certain fundamental notions in [M]inimalist syntax'' \parencite[43]{CollinsC.StablerE_2016}. The goal, as with the present work, is thus ``to be useful as a toolkit for [M]inimalist syntacticians'' \parencite[43]{CollinsC.StablerE_2016}, thus constrasting with the purely mathematical approach. As such, the goal is to abide as closely as possible to elements of Minimalist theory as they are actually used, to the extent that this is possible. This forms part of the justification for the review of the labelling literature presented in \autoref{sec:300}, similarly the brief review of the current state of Minimalist theory generally in \autoref{sec:140}.

This thesis does not qualify as a work of mathematical linguistics per se, which does not \textit{a priori} have to relate to the study of I-language in and of itself, perhaps instead wavering into the domain of formal language theory, a purely mathematical endeavour, albeit one that may have language-related applications, such as within natural language processing, alongside the study of parsing (on this latter point, see \nptextcite{MobbsI_2008, MobbsI_2015}, as well as contributions to \nptextcite{BerwickRC.StablerEP_2019}). The goal of formalisation within the context of biolinguistics should not be to ``play[] mathematical games'' but to ``describe[] reality'' \parencite[81]{ChomskyN_1975a}. Nevertheless, the close association maintained in this introduction between mathematical formalism and the biolinguistic programme may lend the present work to a classification as \textit{mathematical biolinguistics} in the sense of \textcite{WatumullJ_2012, WatumullJ_2013}, blending a biolinguistic ontology with the mathematical realism of \textcite{CohenM_2008} and \textcite{TegmarkM_2014}. Since the metaphysical baggage that this categorisation would beget would take us too far afield, I leave the matter aside, although it receives some discussion in \textcite[Section 2]{VanSteeneL_2021}. The crucial point here is the justification of formal investigation of the properties of natural language on biolinguistic grounds, in accordance with the resolution of the granularity problem and in the search of the simplest model that accords with the empirical facts, in line with the Galilean challenge.

