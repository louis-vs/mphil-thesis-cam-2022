\subsection{Copies and repetitions}\label{sec:420}

The copy/repetition distinction is a fundamental problem in Minimalist syntax. The problem lies in distinguishing between minimally distinct syntactic objects like those in \pxref{ex:coprep}, where the `John' sister of V in \pxref{ex:coprep:cop} represents a moved element (copy; lower copies indicated with angle brackets), and in \pxref{ex:coprep:rep} a repetition (referring to two different people, on the standard reading).%
\footnote{Morphological complications like affix-hopping \parencite{ChomskyN_1975a} are ignored. The two v's represent either the light verb or the verbal categorisers, as contextually clear (see \autoref{fn:littleV}). For a theory of English passives with respect to labelling, see \textcite{BurrowsER_2022}.}

\begin{subexamples}\label{ex:coprep}
    \item\label{ex:coprep:cop} \{John, \{was, \{$_{vP}$ \copySO{John}, \{$_{vP}$ v+seen, \{$_{VP}$\{v, \copySO{see}\}, \copySO{John}\}\}\}\}\} (passive object raising to subject via phase-edge)
    \item\label{ex:coprep:rep} \{John, \{T, \{$_{vP}$ \copySO{John}, \{$_{vP}$ v+saw, \{$_{VP}$\{v, \copySO{see}\}, John\}\}\}\}\} (standard declarative)
\end{subexamples}

This problem is unavoidable in a formalisation of Minimalist syntax---here, I intend to take a somewhat novel approach. \autoref{sec:421} will briefly review some of the options and set out a path forward. \autoref{sec:422} will go into some more depth on the formal nature of `copies'.

\subsubsection{Distinguishing copies and repetitions}\label{sec:421}

Following TLTB, GB-era symbols like \textit{trace} and indices cannot be introduced to mark movement; this would also violate the NTC. Secondly, recall that there is no distinction between Merge and Move; rather, all structure building is by Merge (and by further hypothesis all that reaches the interfaces has been constructed by Merge). A syntactic object constructed by `internal' Merge is identical to one constructed by `external' Merge (see \autoref{def:merge} below). Further, to account for `trace-invisibility' effects with respect to movement (discussed above), both NS and the interfaces need to be able to distinguish copies and repetitions. Therefore, NS, or at least all relevant operations within NS, need to be able to distinguish copies and repetitions.

Despite the centrality of this issue, \textcite{CollinsC.GroatEM_2018} come to the worrying conclusion in their review that ``no adequate proposal exists in [M]inimalist syntax for distinguishing copies and repetitions'' \parencite[2]{CollinsC.GroatEM_2018}. \textcite{CollinsC.GroatEM_2018} review a number of approaches; I will briefly touch on three, two of which are discussed by \textcite{CollinsC.GroatEM_2018}. One option is to use \textit{chains}. As shown by \CS, however, chains introduce a vast amount of machinery that complicates the definition of Merge, and should be abandoned for the sake of TLTB. Further, \CS\ prove (in their Theorem 4) that chain-based structures and the multidominance structures they use in the rest of their paper are isomorphic, which one can infer makes the chain-based theory (or at least the formulation they adopt) inferior. A second proposal, adopted by \CS\ but not discussed by \textcite{CollinsC.GroatEM_2018}, is to augment each LI into a \textit{lexical item token} (\LIk) when introduced into the lexical array. An \LIk\ is an LI with an associated unique index $k$. Copies of the same \LIk\ will thus have the same index. This option evidently violates Inclusiveness, but one could argue that this is a principled violation, since otherwise the EM/IM distinction would be unformulable.

Nevertheless, I do not adopt \LIk s in this formalisation, in the interest of staying as true as possible to the recent literature, especially \textcite{ChomskyN.etal_2019} and \textcite{ChomskyN_2021}. The third option adopted in these works is the idea of a \textit{phase-level memory}. As noted by \textcite[12]{CollinsC.GroatEM_2018}, Chomsky separately notes two possibilities. A third is proposed by \textcite{ChomskyN_2021}. These options are summarised in \pxref{ex:phasemem}.

\begin{subexamples}[preamble={\textit{How could phase-level memory distinguish copies and repetitions?}}]\label{ex:phasemem}
    \item\label{ex:phasemem:1} It must be the case that ``within each phase each selection of an LI from the lexicon is a distinct item, so that all relevant identical items are copies'' \parencite[145]{ChomskyN_2008}.
    \item\label{ex:phasemem:2} ``At TRANSFER, phase-level memory suffices to determine whether a given pair of identical terms Y, Y$'$ was formed by IM.'' Y and Y$'$ are copies if so, else they are repetitions \parencite[246-247]{ChomskyN.etal_2019}.
    \item\label{ex:phasemem:3} There is a ``convention'', ``\textsc{Stability}'', which states that certain occurrences of the same symbol are related; there is a rule ``\textsc{FormCopy} (FC)'' which assigns the \textit{Copy} relation to certain idential symbols and which must adhere to \textsc{Stability}. ``FC applies at the phase level and is interpreted (mapped to CI), not entering into further computation'' \parencite[16-17]{ChomskyN_2021}.
\end{subexamples}

\textcite{CollinsC.GroatEM_2018} interpret both \pxref{ex:phasemem:1} and \pxref{ex:phasemem:2} as being problematic for the same reason. Briefly, with reference to \pxref{ex:phasemem:2}, establishing whether IM or EM was applied in a particular derivational stage would require access to the previous state of the workspace, but this violates the strict Markovian property of the derivation (cf. \nptextcite[20]{ChomskyN_2021}), with dire consequences for interpretation. \textit{Pace} \textcite{CollinsC.GroatEM_2018}, I interpret \pxref{ex:phasemem:1} to be equivalent to introducing \LIk s as done by \CS, which affords phase (lexical array) level uniqueness to tokens. It is thus unsatisfactory by Inclusiveness (\textit{pace} the claim of \nptextcite{ChomskyN_2008}). The fate of the original conception of the \textit{Numeration} \parencite{ChomskyN_1995} fares similarly.

\pxref{ex:phasemem:3} is more cryptic yet at the same time suggestive. The strong claim, as I interpret it, is as follows. The copy/repetition distinction is required \textit{only} at the interfaces---hence, syntactic operations cannot make reference to copies or repetitions. Further, the distinction is \textit{determined} at the interface, not in the syntax. The corresponding illegitimacy or deviance of a derivation with respect to misinterpretation of copies/repetitions emerges from interpretation, but this interpretation, like Merge, may operate freely. To take SM as illustrative: the structure \pxref{ex:coprep:cop} could be pronounced ``John was seen John'', but in this case the two `John's would be parsed as repetitions by a rule of SM and so would be interpreted as gibberish at C-I by the $\theta$-criterion (each DP must be assigned one and only one \thetarole; cf. \nptextcite[36]{ChomskyN_1981}).%
\footnote{In \pxref{ex:phasemem:3}, \textcite{ChomskyN_2021} appears to imply that the copy/repetition distinction is required only at C-I---this cannot be correct, as SM needs to be able to deduce lower copies to obviate their pronunciation. It must be a part of \Transfer. This being said, it seems sensible that copy \emph{formation} not be forced by SM, since lower copies \emph{can} be pronounced, as evidenced above, and indeed \emph{are} pronounced in certain contexts in certain languages and in child language.}
Indeed, this close interaction between the interfaces is an interesting result of the hypothesised proximity of the interfaces to NS (and thus to each other) established in \autoref{sec:220}. This has further implications for minimality, which will be touched upon in \autoref{sec:460}.

Fully exploring the consequences of the FC model proposed by \textcite{ChomskyN_2021} is well beyond the scope of the present work. In particular, it will be needed to establish what impact this has on the analysis of island effects. Nevertheless, in the interest of being forward-looking, it will be adopted, accepting its preliminary nature as a caveat. With this established, it is possible to continue the formalisation. As a result of abandoning \LIk s, there will be some small adjustments in the definitions to follow as compared to \CS.

\subsubsection{Copies and multidominance}\label{sec:422}

It is important to note that, thus far, the notion `copy' has been assumed in a non-technical sense. As is clear from the discussion in \autoref{sec:140}, the generally assumed, intuitive idea is that an internally-Merged object is identical in its source and target positions. In a formalisation that uses some set-theoretic machinery, some necessary properties become apparent. For instance, a standard assumption is that sets contain unordered, unique objects---i.e. $\{a, a, b\}=\{b, a\}$. This being the case, take a structure that could plausibly be the output of IM, $X=\{a, \{a, b\}\}$. \textcite{GartnerHM_2022} points out that, on standard set-theoretic assumptions, the two instances of $a$ are not `copies'; they are identical objects. [I shall recapitulate the proof below, after Merge has been defined. which assumes the existence of a `bracket-erasure' function $sp$, and that $a$ and $b$ are \textit{urelements}, viz. indivisible.] What this entails, then, is that a \textit{multidominance} approach \parencite{CitkoB_2011, CitkoB_2011a} to SOs appears to be in line with Minimalist assumptions. In other words, for the two trees $t$ and $s$ in \pxref{ex:multidom}, for the corresponding SOs $\Zeta(t)=\Zeta(s)$.

\begin{subexamples}\label{ex:multidom}
    \item $t=$
        \begin{forest}
            [{$\alpha$} [,phantom [Y,name=a]] [o,name=b] [X [o,name=c] [Z]]]
            \draw (a) -- (b);
            \draw (a) -- (c);
        \end{forest}
    \item $s=$
        \begin{forest}
            [{$\alpha$} [Y] [X [Y] [Z]]]
        \end{forest}
\end{subexamples}
\noindent
Note that the graph-theoretic complications entailed by the representation $t$ have no theoretical status within the formalisation under discussion, they are merely the result of diagrammatic games \parencite[cf.][]{ChomskyN_2019a}. For instance, from the diagram it appears that there are two nodes that could be considered `roots', $\alpha$ and Y, say if `root' were defined graph-theoretically as `a node not dominated by another node'. However, from the set-theoretic representation \pxref{ex:multidom:set}, which is the only one that has any theoretical status having been constructed by \Merge, it is clear that there is no such confusion.

\begin{example}\label{ex:multidom:set}
    $\alpha=\Zeta(t)=\Zeta(s)=\{_\alpha\ Y, \{_X\ Y, Z\}\}$
\end{example}
\noindent
Similarly, one could protest that this would eliminating the `binary-branching' property of syntactic trees, since if, say, Y were merged with $\alpha$, this would result in a structure in which, diagrammatically, Y would have three branches connecting it with each of its occurrences. Again, however, this property of the tree diagram has no theoretical status. Indeed, (internally) Merging Y with $\alpha$ in this way would be entirely legitimate and would continue to satisfy the Extension Condition (see \autoref{sec:490}). A further illegitimate operation, namely externally merging an LI with Y, would also not be possible, since Y is not a root (as per \autoref{def:root} below). 

Returning to \citeposs{GartnerHM_2022} original concern: \textcite{ChomskyN.etal_2019} claim that multidominance approaches are misguided, precisely because they suggest the existence of ``complex graph-theoretic objects [that] are not defined by simplest MERGE''. Following the argument as set out here, this concern is unwarranted. What is part of the system is \Merge, which forms sets, and without added complication, the \textit{urelements} (indivisible elements) of these sets from the perspective of \Merge\ are LIs, which entails that multiple occurrences of the same LIs are not copies but one and the same object. \textcite{GartnerHM_2022} concludes that when analysing the formal properties of a system, one must first note whether the formal tools in use are being applied at the meta level, talking `about' I-language, or whether they are at the object level, namely part of the system itself. Secondly, one must be aware of the difference between notation and content, avoiding falling into the trap of ``excess notation over subject matter'' \parencite[5]{QuineWV_1941}. In answer to both of these points, the subject matter at hand is I-language, in particular the representations constructed by I-language, namely SOs and the labels of these SOs (features). SOs are sets formed by \Merge\ using objects from the lexicon, also hypothesised to be sets. Nevertheless, properties of sets outside of the fact that they are formed by \Merge\ should not be considered \textit{a priori} allowed. In this formalisation, I assume only the most basic---in particular, set membership, and the natural operations of union and intersection, and compositions of these operations. \textcite{GartnerHM_2022} suggests calling these sets, with limited properties, `M-sets', although I do not adopt this terminology here. Another mathematical object in use alongside sets is functions, in particular the notion borrowed from computer science, following standard computational/cognitive assumptions (cf. \nptextcite{GallistelCR.KingAP_2010}).

As a final note on multidominant structures: I am not introducing the full complexity of multidominance as set out by \textcite{CitkoB_2011a}. This theory requires complications to be introduced to the Merge operation itself, namely `Parallel Merge' and `Sidewards Merge', which are independently ruled out by a principle of computational optimality noted by \textcite{ChomskyN_2019a,ChomskyN_2021} related to accessibility, which I derive in \autoref{thm:MY}.

\subsubsection{Defining occurrences}

There are, however, occasions where different occurrences of SOs within a set-theoretic structure need to be distinguished. One can define the notion occurrence to handle this. \CS\ define occurrence in terms of immediate containment, presenting this alongside a number of useful definitions and theorems. I will not repeat these here, although I will adopt a compatible definition of occurrence which suits our present purposes, and which is (implicitly) corroborated by \textcite{EpsteinSD.etal_2020}.

\begin{definition}
    An \textit{occurrence} of an SO $A$ is a \textit{path}, a sequence of SOs $P = \langle X_1, ... , X_n \rangle$ where for all $0 < i < n$, $X_{i+1} \in X_i$, such that $X_n = A$. An occurrence of $A$ at \textit{position} $P$ is denoted $A_P$.
\end{definition}

This definition of occurrence would also enable a formalisation of \FormCopy. Since this would take us too far afield, I leave this for future work.
