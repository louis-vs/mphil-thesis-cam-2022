\subsection{Timing of labelling}\label{sec:340}

On \pxref{ex:questions:when}, \textcite{ChomskyN_2013} marked a significant shift from the earlier labelling theory assumed in \textcite{ChomskyN_2008} and earlier, by eliminating the stipulation that labelled structures were required within syntactic computation, in order for syntactic objects to be `visible' to further operations. As stated by \textcite[321]{RizziL_2015}: ``labeling can be deferred until when the structure is passed onto the interpretive systems, at the end of a phase''. \textcite[77]{BlumelA_2017a} expresses the issue perspicuously: ``labels do not enter syntactic structures blindly and as a byproduct of Merge/Narrow Syntax [i.e. as they would in classic BPS adopting \pxref{ex:twomerges:label}---LVS] but at a point where they are actually needed, namely the mapping to the semantic component/transfer''. There are, however, numerous complications related to the timing of LA with respect to Transfer.

Firstly, there is the question of when Transfer itself occurs, and which structures are affected by Transfer. In the first description of phases (following its precursor, `multiple spellout', spelled out in particular by \nptextcite{UriagerekaJ_1999,EpsteinSD_1999,EpsteinSD.etal_1998}), \textcite[131-132]{ChomskyN_2000} states that the entire phase is sent to the interfaces at the point of transfer (maintained by \nptextcite{FranksS.BoskovicZ_2001}). \textcite{ChomskyN_2001}, on the other hand, maintains two possibilities: either the entire phase is transferred \parencite[12]{ChomskyN_2001} or the \textit{complement} of the phase (i.e. the sister of the phase head) is transferred \parencite[13]{ChomskyN_2001}. However, as \textcite{BoskovicZ_2016} explores, phasal complement transfer is unsatisfying by TLTB, as phasal complements have no status. Further, the preoccupation with the phasal complement/edge distinction appears to dissolve into contradiction, as explained:

\begin{quote}\setstretch{1.0}
     If both multiple spell-out and S[uccessive]C[yclic]M[ovement] were to be defined strictly on phases, phases would be spell-out units and SCM would target phases. A problem, however, would then arise. It is standardly assumed that what is sent to spell-out is no longer accessible to the syntax. Given this assumption, it is simply not possible to state the domain for both spell-out and SCM in terms of phases. \parencite[39]{BoskovicZ_2016}
\end{quote}
\noindent
\citeposs{BoskovicZ_2016} solution is to adopt the original model whereby the entire phase is transferred at spellout, but to accompany this with the \textcite{ChomskyN_2001} model in which a phase is spelled out \emph{at the next highest phase}. This is arguably conceptually necessary in any case---as stated by \textcite[567]{RichardsMD_2007}, building upon a suggestion of \textcite{ChomskyN_2004}, transfer of an entire phase at the level of that phase would preclude any further computation.

However, \textcite{RichardsMD_2007} makes use of phasal complement transfer to derive an important result, namely an independent justification of the process of feature inheritance hypothesised by \textcite{ChomskyN_2008}. To summarise the argument briefly, \textcite{RichardsMD_2007} argues that phases and non-phases must alternate so that non-phasal heads can inherit the uninterpretable features (uFs) from the phasal heads, such that the uFs on phasal heads can be transferred at the same time they are valued, which they otherwise would not be able to as their phasal head host is part of the phase edge and thus not transferred. It would be preferable to maintain the deduction of feature inheritance, an important mechanism in Minimalist theory. It seems to me that this can be done by adopting instead \citeposs{BoskovicZ_2016} solution. In this model, with phases being spelled out at the next-highest phase, SCM is only allowed in cases where the phrase directly above a phase is not also a phase---otherwise the contents of the lower phase would no longer be accessible for movement. As a result, the phase/non-phase alternation is required for those scenarios in which SCM is allowed. This alternation then allows uninterpretable features to be inherited in the relevant cases.


