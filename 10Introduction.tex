\section{Introduction}\label{sec:100}
\epigraph{\setstretch{1.0}\itshape [Sagredo:] It always seems to me extreme rashness on the part of some when they want to make human abilities the measure of what nature can do. On the contrary, there is not a single effect in nature, even the least that exists, such that the most ingenious theorists can arrive at a complete understanding of it. This vain presumption of understanding everything can have no other basis than never understanding anything. For anyone who had experienced just once the perfect understanding of one single thing, and had truly tasted how knowledge is accomplished, would recognize that of the infinity of other truths he understands nothing.}{\setstretch{1.0}--- Galileo, \textit{Dialogue concerning the two chief world systems} (\citeyear[101]{Galileo_1967})}

The quotidian use of language often concerns itself with the correct naming of things, be these actual objects in the world or, perhaps more accurately, objects constructed by our minds through (and in spite of) our interactions with the world. On a meta level, this same nominal curiosity extends to those abstract objects which we make use of in formulating these descriptions, the invisible structure which allows sound and sign to be associated with meaning in infinitely complex ways. \textit{A priori}, there is no reason to associate particular configurations within a structure with particular \textit{labels}, adopting the typical terminology. Rather, the conceptual necessity of labels, to the extent they are required, must be derived from observations of the natural world to which language belongs, which is no trivial task.

This thesis concerns the nature of labels within the formalised realm of natural language syntax. It will present one possible set of answers to the crucial questions: what is the nature of a syntactic label, how is it assigned, and how is it then made use of in derivations? This project is manifestly ambitious, and hence these introductory sections serve to restrict the scope of the questions being pondered in ways that are extensive, yet I believe principled. Firstly, the notion of `language' itself needs to be established and scrutinised, along with a clarification of the precise object of study: language as a part of the natural world. With this in place, guiding principles for the construction of a formal model of this natural phenomenon can be assembled, manifesting in the form of constraints and heuristics that suggest themselves from general observations and truisms. These can then be made specific in the presentation of an outline of the theory that will be adopted and adapted in the subsequent sections. This introduction therefore sets the stage for a precise formalisation of syntactic labels and the processes they are involved in, subsumed within the wider metatheoretical landscape. The focus is thus placed narrowly on theoretical and metatheoretical concerns, in a specific sense to be defined---empirical discussion of the consequences of these proposals is left for future work.

\subsection{Biolinguistics and the Galilean challenge}\label{sec:110}

A major motivation of linguistics as construed in the present context is to rise to what \textcite[i.a.]{ChomskyN_2017a} formulates as the \textit{Galilean challenge}, citing a passage from Galileo's infamous \textit{Dialogo}:

\begin{quote}\setstretch{1.0}
    ``[Sagredo:] But surpassing all stupendous inventions, what sublimity of mind was his who dreamed of finding means to communicate his deepest thoughts to any other person, though distant by mighty intervals of place and time! Of talking with those who are in India; of speaking to those who are not yet born and will not be born for a thousand or ten thousand years; and with what facility, by the different arrangements of twenty characters upon a page!'' \parencite[105]{Galileo_1967}
\end{quote}

Construed narrowly, this passage refers to the alphabet, truly a human ``invention'', as opposed to endowment. Nevertheless, it would not be unfair to extrapolate from this passage, as \textcite{ChomskyN_2017a} does, a general wonder at the generative property of language, how finite knowledge can have infinite range. In the terms of the later thinker, Willhelm von Humboldt, language is characterised by the ``infinite use of finite means''. This oft-quoted aphorism is insightful within the philosophical context of modern linguistics, as explored in great detail by \textcite{ChomskyN_1966}, but it is also notable for its conflation of language knowledge and use. This Aristotelian distinction was revived in the modern generative tradition by \textcite{ChomskyN_1965}, after its occlusion within the tenets of behaviourism, in which the concept of `knowledge of language', indeed symbolic knowledge of any kind, is effectively unformulable \parencite[cf.][]{ChomskyN_1959,GallistelCR.KingAP_2010}. As outlined by \textcite[3]{ChomskyN_1986a}, this ``shift of focus [] from behaviour or the products of behaviour to states of the mind/brain that enter into behaviour'' provides the grounds for a rich research programme---that of generative grammar. Consequently, language \textit{use} will not be considered much further in this thesis, beyond the necessary empirical selection of instances of language use required to give insight into the knowledge that underlies such use. This follows the general assumptions of the generative programme that date back to \textcite{ChomskyN_1981} and earlier: ``the grammar---a certain system of knowledge---is only indirectly related to presented experience, the relation being mediated by UG [Universal Grammar, to be defined \textit{sub}---LVS]'' \parencite[4]{ChomskyN_1981}. The concern is thus with linguistic \textit{competence}, not \textit{performance} (adapting a distinction made by Saussure, \nptextcite[10]{ChomskyN_1964}).

Some further clarification of the notion of `knowledge of language', the subject matter of this thesis, is thus in order. The term `language', can be and thus far has been used in a non-technical sense, an ``informal rubric'' that allows one to ``select certain aspects of the world as a focus of inquiry'' \parencite[1]{ChomskyN_1995c}. Beyond these introductory remarks, it will be avoided in favour of more specific terms. The object of study of this thesis is language as an ``element[] of the natural world, to be studied by ordinary methods of empirical inquiry'' \parencite[1]{ChomskyN_1995c}. Furthermore, the approach to language taken is an internalist one \parencite{ChomskyN_2003}: language as a property \textit{internal} to the mind/brain of an \textit{individual}, and which is \textit{intensional}, in that it specifies a ``procedure that generates infinitely many expressions'' \parencite[263]{ChomskyN_2003}, a characterisation made plausible with the advent of the theory of computation in the 20th century, enabling the infinite to be compressed into the finite \parencite{TuringAM_1936}. The concern is thus with \textit{generation}, not production, enabling a certain precision in description abandoned by 20th century structuralist-behaviourist-empiricists, as noted by \textcite{ChomskyN_2021c} in recent remarks.

Call this naturalistic, materialist, internalist approach to language \textit{I-linguistics} \parencite[263]{ChomskyN_2003}, which concerns itself with the faculty of language (FL), an organ of the mind/brain dedicated to language, and the states it assumes---call these \textit{I-languages} \parencite[21]{ChomskyN_1986a}. I-language is thus a biological entity, necessitating study following the same principles as any other matter of biology. This forms the central tenet of \textit{biolinguistics}, a term coined by Massimo Piattelli-Palmarini (\nptextcite{MorinE.Piattelli-PalmariniM_1974}, see \nptextcite[1]{DiSciulloAM.BoeckxC_2011}) identifying a domain of study influentially formulated by \textcite[vii]{LennebergEH_1967}, namely the study of ``language as a natural phenomenon---an aspect of [man's] biological nature, to be studied in the same manner as, for instance, his anatomy''.

I-language is characterised by what has been called the \textit{Basic Property}, introduced by \textcite[\pnfmt{1}]{BerwickRC.ChomskyN_2016} and \textcite[4]{ChomskyN_2016}. The property is concisely stated as such: ``each language provides an unbounded array of hierarchically structured expressions that receive interpretations at two interfaces, sensorimotor for externalization and conceptual-intentional for mental processes'' \parencite[4]{ChomskyN_2016}. The Basic Property is arguably irreducible, a revival of Aristotle's dictum that language is ``sound with meaning'', placing considerably more emphasis on the exact nature of the ``with''. The renewed focus on the interfaces and the conditions they impose has been a hallmark of the contemporary iteration of the generative enterprise, the Minimalist Programme (MP) as set forth by \textcite{ChomskyN_1993}.\footnote{A stylistic note: when referring to the principles of and the arguments for the MP, the word \textit{Minimalism} and its morphological family will be capitalised. While we're here, it is also worth noting that single quotes `' always indicate scare quotes, and double quotes ``'' are uniformly used for quotations, with the natural caveat that quote styles are preserved within the bodies of quotations. Generally, APA 7th Edition \parencite{APA_2020} is adopted with some idiosyncracies, linguistic and personal.} This focus on the interfaces later became enshrined in the \textit{Strong Minimalist Thesis} (SMT), as in \pxref{ex:SMT}, now central to Minimalist research.

\begin{example}\label{ex:SMT}
    \textbf{Strong Minimalist Thesis (SMT)}: I-language is an optimal solution to interface conditions. \parencite[1]{ChomskyN_2001}
\end{example}

As \textcite{ChomskyN_2001} notes, the SMT only becomes an empirical thesis once the notions of `optimality' and of `interface conditions' are defined. These ideas will be explored in \autoref{sec:130} and \autoref{sec:150}, respectively. For now, it suffices to state that this thesis is a biolinguistic one, insofar as it concerns I-language and its place within the mind/brain.

As set out by \textcite{MobbsI_2015}, the \textit{argument of linguistic Minimalism} is actually ``a collection of related, but logically independent, proposals [that have] coalesced in the literature of the past twenty-five odd years'' \parencite[1]{MobbsI_2015}. The five proposals identified by Mobbs are summarised in \pxref{ex:minprops}.

\begin{subexamples}[preamble={\textbf{The Minimalist Proposals} \parencite{MobbsI_2015}}]\label{ex:minprops}
    %\setstretch{1.0}
    %\setlength{\itemsep}{1pt}
    %\setlength{\parskip}{0pt}
    %\setlength{\parsep}{0pt}
    \item\label{ex:minprops:1} \textbf{Methodological minimalism}: When faced with two empirically equivalent proposals, ``we should adopt the more parsimonious explanation, that is, the one containing fewer ancillary claims'' \parencite[38]{MobbsI_2015}.
    \item\label{ex:minprops:2} \textbf{Ontological minimalism}: assume the SMT, ``a refusal on the part of the scientist to make pre-theoretic assumptions about the design or `purpose' of language'' \parencite[41]{MobbsI_2015}.
    \item\label{ex:minprops:3} \textbf{SMT and evo-devo}: ``certain facts about language cannot clearly be explained in terms of optimality, and we are forced to propose further innate competence'' \parencite[46]{MobbsI_2015}; this innate competence should be constrained by evolutionary-developmental concerns.
    \item\label{ex:minprops:4} \textbf{The primacy of the CI interface}: ``language is optimized for the system of thought, with mode of externalization secondary'' \parencite[32]{BerwickRC.ChomskyN_2011}.
    \item\label{ex:minprops:5} \textbf{Variation}: ``[p]arameterization and diversity [are] mostly---possibly entirely---restricted to externalization'' \parencite[37]{BerwickRC.ChomskyN_2011}.
\end{subexamples}

The first three proposals collectively comprise the MP, the fourth and fifth proposals are ``specific claims about the design and origin of FL'' \parencite[2]{MobbsI_2015}. They stand logically separate, and need not all be assumed within a Minimalist work. For instance, much work in the comparative syntax tradition does not make assumption \pxref{ex:minprops:4}; rather, the parameters along which different I-languages may vary, encoded as formal features, remain a focus of investigation \parencite[see]{RobertsI_2019,SheehanM.etal_2017,BiberauerT.etal_2010}.

The first proposal, methodological minimalism, is the least controversial, but perhaps also most important. It is effectively a reformulation of \textit{Occam's Razor}, the principle that ``[w]e may assume the superiority \textit{ceteris paribus} of the demonstration which derives from fewer postulates or hypotheses'' (Aristotle, \textit{Posterior Analytics}, cited by \nptextcite{BakerA_2016}; emphasis original). Appropriately, Galileo also adopted the principle: ``it is said that Nature does not multiply things unnecessarily; that she makes use of the easiest and simplest means for producing her effects; that she does nothing in vain, and the like'' \parencite[397]{Galileo_1967}. Simplification must come with justification, however. \textcite[13]{ChomskyN_1981} notes this explicitly: ``it is evident that a reduction in the variety of systems in one part of the grammar is no contribution to these ends if it is matched or exceeded by proliferation elsewhere'' ... ``[i]t is only when a reduction in one component is not matched or exceeded elsewhere that we have reason to believe that a better approximation to the actual structure of mentally-represented grammar is achieved''. In light of methodological minimalism, Chomsky's objection here goes both ways: one must be perpetually wary of the power of the explanatory devices introduced within a theoretical framework, in other words, to avoid ``the temptation to offer a purported explanation for some phenomenon on the basis of assumptions that are of roughly the order of complexity of what is to be explained'' \parencite[233]{ChomskyN_1995}. As a methodological guideline, this serves of vital importance, especially in the discussion to follow in \autoref{sec:200} and \autoref{sec:300}, and in the formalisation itself in \autoref{sec:400}. Different construals of labelling have vastly different implications in terms of complexity (theoretical and computational) and need to be considered within the broader framework of the Minimalist theories they are a part of. Methodological minimalism also serves as a reminder of the consequences of introducing richer theoretical devices, by encouraging a holistic approach, in which the overally complexity of the theory is under constant scrutiny.

The Minimalist principles, now fully established, are thus very useful in the study of biolinguistics. To demonstrate but one example, take again the first Minimalist proposal \pxref{ex:minprops:1}. A perpetual problem of biolinguistics which is perhaps most clearly expressed by \textcite{PoeppelD.EmbickD_2005,EmbickD.PoeppelD_2015} concerns the interrelation of abstract theories of I-language with the neurobiological models that are supposed to implement the algorithms proposed in theories such as that of this thesis. Known as the `granularity problem', the question arises as to whether it is possible to reduce the often complex and theoretically rich proposals within the linguistics literature---and arguably cognitive science more generally---to the simple constructs of neurobiology. There are a number of ideas adopted within the biolinguistic literature to make this chasm of separation less ominous. Firstly, the adoption of methodological minimalism encourages the reduction of theoretical complexity, ultimately making it more likely that such a model could bridge the explanatory gap. Another set of proposals which will provide some useful framing to the present work are the three computational levels for information processing systems proposed by \textcite[25]{MarrD_1982}, adapted from \textcite{MarrD.PoggioT_1976}, as shown in \pxref{ex:marr}.

\begin{subexamples}\label{ex:marr}
    \item\label{ex:marr:1} \textbf{Computational level}: What does the computation do, and what are its goals?
    \item\label{ex:marr:2} \textbf{Algorithmic/representational level}: How are the inputs and outputs of the computation represented, and what is the algorithm that computes said outputs? 
    \item\label{ex:marr:3} \textbf{Physical/implementational level}: How are the representations and algorithms encoded physically?
\end{subexamples}

Though he was working on human vision, Marr's motivation for proposing these levels of analysis is analogous to the reasoning employed by \textcite{ChomskyN_1964,ChomskyN_1965} in devising the metric of \textit{explanatory adequacy} to evaluate a linguistic theory, later adapted into MP as the principle of \textit{genuine explanation} \parencite{ChomskyN_1993,ChomskyN_1995}. Specifically, \textcite[15]{MarrD_1982} notes that ``neurophysiology and psychophysics have as their business to \textit{describe} the behaviour of cells or of subjects but not to \textit{explain} such behaviour''. Indeed, finding anything at all interesting from observational or descriptive work (again, generalising the definitions of these terms given by \nptextcite{ChomskyN_1964}) is if anything surprising: ``[i]f one probes around in a conventional electronic computer and records the behaviour of single elements within it, one is unlikely to be able to discern what a given element is doing'' \parencite[14]{MarrD_1982}. It should thus be surprising that work along these lines in neurobiology has indeed had a lot of success. In vision, as in linguistics (in the form of the descriptive success of the structuralists, and of early investigation in neurolinguistics), some sense of progress had thus been made along such lines, but these results fall short of being explanatory when this is defined in a principled sense. In sum, ``understanding computers [equivalently, human brains---LVS] is different from understanding computations'' \parencite[5]{MarrD_1982}. This is where proper consideration of Marr's levels of analysis becomes particularly revealing, offering a positive way out, in which study of neurobiology is not the be-all-and-end-all, but rather an analysis of one aspect of a computational system. Investigation of the computational and algorithmic levels is arguably of equal importance, both to constrain analysis of the more fine-grained phenomena and also, more generally, to gain a holistic understanding of the system, and on a meta level to constrain our conception of what we can even begin to learn about the system in the first place.

Beyond the introduction, the implementational level will receive no further attention. It is hoped, however, that by clarifying the computational level and providing at least tentative steps towards an algorithmic representation, the explanatory gap between linguistics and neuroscience can be reduced, as should be a major objective of biolinguistics, which emphasises the building of interdisciplinary bridges \parencite{DiSciulloAM.BoeckxC_2011, BoeckxC_2013, WatumullJ_2013}. It could be argued that the biolinguistics programme has been circumspect since its inception in its historical focus on the computational level. As \textcite[2]{MarrD.PoggioT_1976} state, ``although the top [computational---LVS] level is the most neglected, it is also the most important[, because] the structure of the computations that underly perception depend more upon the computational \textit{problems} that have to be solved than on the particular hardware in which their solutions are implemented'' (emphasis original). This view of computational neuroscience is revisited by \textcite{GallistelCR.KingAP_2010}, who repeatedly emphasise that understanding general computational theory, as well as the specific instantiations of computations that must be taking place in order for an organism to do certain things at all, should indeed preclude analysis of the neurobiological mechanisms that could plausibly implement such computations. This `top-down' approach thus equates to a stronger thesis than Marr's. Another important aspect of the research programme advanced by \textcite{GallistelCR.KingAP_2010} is that, as a logical consequence of adopting a computational model, the \textit{symbols} making up the mental representations that are manipulated by computational procedures become of vital significance. Meanwhile, the implementational details of such representations is not inherently necessary to understanding the computation, as the three level model would indeed suggest. As \textcite{GallistelCR_2001} outlines concisely: ``[w]hat matters in representations is form, not substance''. Further, it is these mental representations which should be formalised as part of the algorithmic level, following the definition provided in \pxref{ex:marr:2}.

This focus on the computational level in search of reducing the explanatory gap further relates to a point raised by \textcite[187]{ChomskyN_1994a}: it is often the case in the history of science that the ``more `fundamental' science has had to be revised'' in order that the different levels of analysis be unified. One example given is of the dawn of quantum physics by the 1930s, which allowed theories of chemistry that failed to fit in with the classical model to be explained. In the case of language, the current state of knowledge in neuroscience is very far from being able to give any real explanation as to how I-language is implemented \parencite[cf.][]{GallistelCR.KingAP_2010}. An understanding of the computations, and in following the algorithms, can be sought without overreliance on the implementation. Thus, I adopt a more tempered stance on the biolinguistic programme than \textcite{MartinsPT.BoeckxC_2016}, who effectively insist that implementational details are required for work to be characterised as biolinguistic.

Another useful heuristic that applies to the investigation of any biological system is the `triple helix' model as proposed by \textcite{LewontinRC_2000}. An organism and its components cannot solely be explained through genetics, and the same is evidently true for I-language. Rather, one must, alongside the gene, consider the development of the organism itself, and the pressures and constraints of the environment. Analogously, \textcite{ChomskyN_2005} adapts this notion to I-language, proposing the \textit{three factor model}. The first factor is the \textit{genetic endowment}, assumed to be effectively uniform for all humans, and which allows us to interpret part of our environment as linguistic and construct grammars. The theory of this factor can be termed \textit{Universal Grammar} (UG). The `domain-specificity' of UG has come under considerable scrutiny since its inception \parencite[e.g.][]{TomaselloM_2003}, but in reality the point is somewhat moot. The first factor must \textit{a priori} play a role, since there must be something that enables a human child to acquire language, uniquely within the animal kingdom. This is a weak hypothesis, but it need not be much stronger, when approaching I-language from the perspective of the Basic Property, or in earlier terms ``from bottom up'' \parencite[4]{ChomskyN_2007}. Secondly, factor two is the environmental factor, the \textit{Primary Linguistic Data} \parencite[PLD;][10]{ChomskyN_1981}. The interaction and tension between the first and second factors forms the fundamental basis of the argument from the poverty of the stimulus \parencite{ChomskyN_1980b, ChomskyN_1986a} and also the well-known tension between descriptive and explanatory theoretical adequacy \parencite{ChomskyN_1964,ChomskyN_1965}. More specifically, the primitives provided by the genetic endowment should be sufficient to map the child's intake to the PLD, in other words to interpet the data available to the child into linguistic terms---the notion of \textit{epistemological priority} \textcite[10]{ChomskyN_1981}. As such, the PLD constitutes \textit{intake} as opposed to \textit{input}, following the terminology used by \textcite{EversA.vanKampenJ_2008}. Finally, there is the third factor, effectively a catch-all term encompassing the external, non-language-specific constraints imposed by nature. As stated by \textcite[6]{ChomskyN_2005}, the most important third factors to consider in the case of I-language are ``principles of structural architecture and developmental constraints'', in particular ``principles of efficient computation, which would be expected to be of particular significance for computational systems such as language''.

As an aside, it is worth noting that this model was made explicit at least as early as \textcite[105]{ChomskyN_2004} and, as noted by \textcite[xi]{ChomskyN_2006}, the presence of third factors has been recognised since the genesis of biolinguistics, whilst the development of MP merely made the scale of such concerns more apparent. The exact position is stated already by \textcite[59]{ChomskyN_1965}: ``there is surely no reason today for taking seriously a position that attributes a complex human achievement entirely to months (or at most years) of experience [=2nd factor---LVS], rather than to millions of years of evolution [=1st factor---LVS], or to principles of neural organization that may be more deeply grounded in physical law [=3rd factor---LVS]''. MP and the three factor model are thus a natural step to take in search of the answers to the Galilean challenge.

What is however less often appreciated, as noted by \textcite{BoeckxC_2014a}, is that the critical insight of the triple helix model as originally conceived by \textcite{LewontinRC_2000} is in the \emph{interaction} between the factors. It is thus arguably incoherent to talk about `factors' in isolation, for instance, by conceiving of some or another proposed linguistic module as `belonging to the third factor', as has become somewhat commonplace \parencite[see][]{GallegoAJ_2011}. Indeed, due to the maximally general definition of `third factors', this is arguably not a characterisation that carries much theoretical content. As such, an understanding of the interaction between the factors will be explicitly pursued here. Another example of such an approach is offered by \textcite{RobertsI_2019} in the context of understanding parameters in a Minimalist context, the hypothesis being that the emergence of parameters in the acquisition process can be explained by utilising a combination of all three factors.

These issues are of particular relevance within the discussion of labelling. As will become clear as a central theme of \autoref{sec:200} and \autoref{sec:300}, the somewhat precarious position of syntactic labels since the advent of Bare Phrase Structure \parencite{ChomskyN_1994} has led to a sizeable push towards their relegation into being a third factor constraint: ``The simplest assumption is that LA [the labelling algorithm---LVS] is just minimal search, presumably appropriating a third factor principle'' \parencite[43]{ChomskyN_2013}. The following two subsections will seek to clarify further exactly what ``simplest'' should mean in this context. Furthermore, Boeckx's aforementioned objection will be adopted in what follows, emphasising the points of \emph{interaction} between factors. In the context of labelling, these will predominantly be the first and third factors: how does ``minimal search'' (MS) interact with (constrain, enable) the computational provisions of the genetic endowment in deriving linguistic structures?

A final useful heuristic to keep in mind is the three-way distinction between \textit{metatheory}, \textit{theory} and \textit{analytic} work, as first deconstructed by \textcite{ChametzkyRA_1996}. Metatheoretical work could be considered as much a philosophical endeavour as a linguistic one: it defines the evaluation of theories, as well as how they ought to be constructed, and hence is the focus of this introduction. Theoretical work, on the other hand, is the ``deployment of metatheoretical concepts'' \parencite[xviii]{ChametzkyRA_1996}---the construction of primitives and theorems derived from them. This is finally complemented by analytic work, which involves direct investigation of the phenomena in question. By applying theories to data, they can be tested and refined. Analytic enquiry is, without doubt, the most important, and rightly most time-consuming area of linguistics---as \textcite[xviii]{ChametzkyRA_1996} states, ``[f]or linguistics to be the science of language, this must be where linguists do their work''. This being said, for analytic work to be productive it needs a sufficiently well-defined formal framework.  Furthermore, it must fall in line with the metatheoretical aims of linguistic enquiry, in order to ensure that the data being studied actually have an evidential relation with the theory being tested. Following the introduction, this thesis is firmly theoretical, operating at the computational level of analysis in \autoref{sec:200} and \autoref{sec:300}, progressing closer to the algorithmic level in \autoref{sec:400}.

To summarise the crucial metatheoretical concerns, and following \textcite[7]{ChomskyN_2021}, it is clear that UG must meet three contradictory conditions:

\begin{subexamples}
    \item\label{def:learnability} \textbf{Learnability}: UG (in conjunction with third factor principles) must be rich enough to enable language acquisition, overcoming the poverty of the stimulus.
    \item\label{def:evolvability} \textbf{Evolvability}: UG must be simple enough to have plausibly emerged under the conditions of human evolution.
    \item\label{def:universality} \textbf{Universality}: UG, as the name implies, is universal to all humans.
\end{subexamples}

\pxref{def:learnability} and \pxref{def:evolvability} evidently contradict each other: the theory must be both rich enough that a child has the resources to acquire the language(s) of their environment, but refined enough such that it could have evolved in a relatively short space of time (cf. \nptextcite{BerwickRC.ChomskyN_2011,BerwickRC.ChomskyN_2016}, for more precise discussion of the evolutionary context, and \nptextcite{BalariS.LorenzoG_2012} for a different approach). Satisfying learnability and evolvability is the ``austere requirement'' that constitutes the bare minimum for ``genuine explanation'' \parencite[14]{ChomskyN_2020a}. \pxref{def:universality} is another way of stating the \textit{Uniformity Principle} of \textcite[2]{ChomskyN_2001}: ``assume languages to be uniform, with variety restricted to easily detectable properties of utterances''. It can also be restated as the typological problem: why do languages appear to vary so much on the surface, and how is this variation constrained? This is effectively the problem solved with Chomsky's Principles and Parameters model (cf. \nptextcite{ChomskyN.LasnikH_2015}, for an overview), but this model needs to be adapted from its early, Government-Binding form \parencite{ChomskyN_1981} into something more obviously compatible with Minimalism. Some notes on this, with particular reference to how labelling may play a role, are provided in \autoref{sec:150}.


\subsection{Formalisation and mathematical linguistics}\label{sec:120}

To adopt the concise phrasing of Chris Collins (p.c.), formal work is needed in the domain of labelling in particular, ``since otherwise we don't really understand what we are doing''. It is not, though, a mischaracterisation to say that formalisation has been a concern of the generative enterprise since its inception. One need only look at one of the foundational documents of the enterprise to find the view expressed that there is an intra-theoretical benefit of formalisation: ``a formalised theory may automatically provide solutions for many problems other than those for which it was explicitly designed'', avoiding reliance upon ``[o]bscure and intuition-bound notions'' \parencite[5]{ChomskyN_1957}. This being said, there remains only a very small component of the literature that focuses on formalisation. Using the terminology established in \autoref{sec:110}, much theoretical work instead centres on computational concerns \pxref{ex:marr:1} as opposed to algorithmic details \pxref{ex:marr:2}, and Chomsky's own output often drifts into the metatheoretical, which is wholly on the computational level, dealing as it does with more abstract ontological concerns. Attempts to provide a complete summary of the theory are typically pedagogical \parencite[for example][]{AdgerD_2003, RadfordA_2004a,HornsteinN.etal_2005,SporticheD.etal_2014}, and tend to struggle to extract a completely coherent theory from the Minimalist literature \parencite{AsudehA.ToivonenI_2006}. \textcite{CollinsC.StablerE_2016} provide the partial formalisation of Minimalist syntax which forms the bedrock of the present work. Nevertheless, and by their own admission, their formalisation is incomplete, covering only a small subset of the full range of operations typically assumed in the analytical literature. Furthermore, some of its proposals will necessarily require revision in order to be in line with recent developments, notably those of \textcite{ChomskyN_2019a,ChomskyN_2021}, as will be discussed in \autoref{sec:400}.

When it comes to formalising the theory of syntax there are effectively two approaches that can be taken. The first is typified by the field of \textit{mathematical linguistics}, perhaps more accurately characterised a subfield of mathematics rather than linguistics, which seeks to uncover properties of the formal languages originally developed out of linguistic theory. Models within mathematical linguistics may depend to varying extents on empirical considerations, and the focus is rather on the formal and computational characteristics of the grammars being studied. The approach dates back to the earliest work in generative grammar \parencite{ChomskyN_1956, ChomskyN.MillerGA_1963, MillerGA.ChomskyN_1963}, and was extended into the Minimalist era by \textcite{StablerE_1997}, who formalised part of the new, feature-driven, derivational theory presented by \textcite{ChomskyN_1995}. \textcite{StablerE_1997} established the formal Minimalist Grammar {MG}, which was subject to considerable further investigation \parencite[e.g.][]{GrafT_2013}. MGs typically adopt numerous conventions which depart from standard Minimalist assumptions, including, but not limited to, being `label-free', encoding linear order, and being necessarily endocentric. Such mathematical work will not receive much more consideration here.

The second approach is the one taken by \textcite{CollinsC.StablerE_2016}, and is the one adopted in the present work. Unlike MGs, which ``were simplified to facilitate computational assessment'', this approach sets out to ``give a precise, formal account of certain fundamental notions in [M]inimalist syntax'' \parencite[43]{CollinsC.StablerE_2016}. The goal, as with the present work, is thus ``to be useful as a toolkit for [M]inimalist syntacticians'' \parencite[43]{CollinsC.StablerE_2016}, thus constrasting with the purely mathematical approach. As such, the goal is to abide as closely as possible to elements of Minimalist theory as they are actually used, to the extent that this is possible. This forms part of the justification for the review of the labelling literature presented in \autoref{sec:300}, similarly the brief review of the current state of Minimalist theory generally in \autoref{sec:140}.

This thesis does not qualify as a work of mathematical linguistics per se, which does not \textit{a priori} have to relate to the study of I-language in and of itself, perhaps instead wavering into the domain of formal language theory, a purely mathematical endeavour, albeit one that may have language-related applications, such as within natural language processing, alongside the study of parsing (on this latter point, see \nptextcite{MobbsI_2008, MobbsI_2015}, as well as contributions to \nptextcite{BerwickRC.StablerEP_2019}). The goal of formalisation within the context of biolinguistics should not be to ``play[] mathematical games'' but to ``describe[] reality'' \parencite[81]{ChomskyN_1975a}. Nevertheless, the close association maintained in this introduction between mathematical formalism and the biolinguistic programme may lend the present work to a classification as \textit{mathematical biolinguistics} in the sense of \textcite{WatumullJ_2012, WatumullJ_2013}, blending a biolinguistic ontology with the mathematical realism of \textcite{CohenM_2008} and \textcite{TegmarkM_2014}. Since the metaphysical baggage that this categorisation would beget would take us too far afield, I leave the matter aside, although it receives some discussion in \textcite[Section 2]{VanSteeneL_2021}. The crucial point here is the justification of formal investigation of the properties of natural language on biolinguistic grounds, in accordance with the resolution of the granularity problem and in the search of the simplest model that accords with the empirical facts, in line with the Galilean challenge.



\subsection{Computational and substantive optimality}\label{sec:130}

I-language is a computational procedure, in the sense pioneered by Turing and his contemporaries \parencite{TuringAM_1936}. It is therefore natural to analyse it in terms of complexity theory, terms increasingly apparent with the dawn of MP.

MP as defined by the first three proposals in \pxref{ex:minprops} allows specific hypotheses and heuristics regarding computational complexity to be made. \textcite{MobbsI_2015} provides a discursive synthesis of these, culminating in the taxonomy summarised in \pxref{ex:mobbscompopt} and adopted within this thesis.

\begin{example}\label{ex:mobbscompopt}
    \textbf{A Minimalist framework for computational optimality} \parencite{MobbsI_2015}
    \begin{itemize}
        \item \textit{Maximise Throughput (MaxTP)}
        \begin{itemize}
            \item \textit{Minimise Time Complexity (MinTC)}
            \begin{itemize}
                \item \textit{Minimise Redundant Operations (MinRO)}
                \begin{itemize}
                    \item \textit{Minimise Vacuous Operations (MinVO)}
                    \item \textit{No Tampering Condition (NTC)}
                    \item \textit{Minimise Search (MinSearch)}
                \end{itemize}
                \item \textit{Minimise Reduplication (MinRedup)}
            \end{itemize}
            \item \textit{Minimise Space Complexity (MinSC)}
            \begin{itemize}
                \item \textit{Minimise Caching of Unintroduced Items (MinCUI)}
                \item \textit{Minimise Caching of Incomplete Derivation (MinCID)}
                \item \textit{Minimise Caching of Completed Derivation (MinCCD)}
            \end{itemize}
        \end{itemize}
    \end{itemize}
\end{example}

The principle of MaxTP, coupled with the computational orientation of the theory, enables the use of ideas from computational complexity theory. The most relevant of these from the perspective of linguistic theory are effectively distilled in \pxref{ex:mobbscompopt}. Notably, NTC and MinSearch will receive particular attention and more precise definitions, especially within the formal section of the thesis, \autoref{sec:400}.

As discussed by \textcite{MobbsI_2015}, alongside computational optimality there is a notion of \textit{substantive} optimality, which ``can be thought of as the number of different types of symbol and computational operation (CO) the FL employs'' \parencite[f.n.~57]{MobbsI_2015}. In effect, substantive optimality dictates that we, as theorists, reduce the number of `tools' we introduce into our theories, especially those that cannot be justified by independent means. \textcite{MobbsI_2015} compresses the concept of substantive optimality into the principle of \textit{The Less The Better} (TLTB): ``[t]his principle merely observes that a proliferation of types of symbols, constraints on symbols, types of CO, and constraints on the output of COs (taken to be) used in the FL runs contrary to M[ethodological]M[inimalism], O[ntological]M[inimalism] as methodology, and the evo-devo hypothesis for language'' \parencite[61]{MobbsI_2015}. TLTB is thus shorthand for the three Minimalist proposals which comprise MP.

One may protest that `TLTB' is a hopelessly vague principle, too general to have any beneficial effect. As demonstrated in the following discussion, however, TLTB can be invoked to efficiently justify theoretical choices, without referencing individual Minimalist proposals from \pxref{ex:minprops}. Notwithstanding this, \textcite[f.n.~99]{MobbsI_2015} notes the complications that would arise from even attempting to formulate a narrower definition of substantive optimality. The emergence of more symbols and operations conceivably entails an evolutionary burden, and there will be a cognitive cost associated with a larger computational inventory. TLTB allows ``some ``principle of structural architecture'' [to] constrain the instantiation of new computational machinery -- although we have little idea of its character, in accordance with the obscurity of neuronal implementation'' \parencite[f.n.~99]{MobbsI_2015}. As a more streamlined instantiation of Occam's Razor, TLTB serves a useful Minimalist purpose. Indeed, specific principles posed within Minimalism are reduced to TLTB, such as the \textit{Inclusiveness Condition} \pxref{def:inclusiveness} and \textit{Full Interpretation} \pxref{def:FI}.

\begin{examples}
    \item\label{def:inclusiveness}
        \textbf{\textit{Inclusiveness Condition}}

        ``[N]o new objects are added in the course of computation apart from rearrangements of lexical properties'' \parencite[228]{ChomskyN_1995}

    \item\label{def:FI}
        \textbf{\textit{Principle of Full Interpretation (FI)}}

        ``Features can only appear in a derivation if they are already interpretable to the interfaces, or can be properly licensed for deletion [footnote omitted---LVS] before the interfaces.'' (\nptextcite[62]{MobbsI_2015}, cf. \nptextcite[95-101]{ChomskyN_1986} and \nptextcite[151,219-220]{ChomskyN_1995})%
        \footnote{Interface interpretability is covered in \autoref{sec:140}.}
\end{examples}
\noindent
Note that the combination of \pxref{def:inclusiveness} and \pxref{def:FI} entail that all properties within syntax are ultimately lexical. As will be discussed in \autoref{sec:140} and \autoref{sec:200}, this is a crucial result that emerges from the Minimalist architecture.

In sum, answering the Galilean challenge in a principled, scientific manner, entails adopting substantive optimality in the form of TLTB, alongside the principles of computational optimality as in \pxref{ex:mobbscompopt}.



\subsection{Informal theoretical summary}\label{sec:140}

This subsection introduces the main theoretical concepts that have been accepted as standard within MP and that form the basis of the discussion to follow. The theory is ultimately founded upon empirical results that will not receive any discussion here. The goal is instead to summarise and synthesise Chomsky's seminal Minimalist papers \parencite{ChomskyN_1993,ChomskyN_1994,ChomskyN_1995,ChomskyN_2000,ChomskyN_2001,ChomskyN_2004,ChomskyN_2007,ChomskyN_2008,ChomskyN_2013,ChomskyN_2014,ChomskyN_2015,ChomskyN.etal_2019,ChomskyN_2019b,ChomskyN_2020,ChomskyN_2021}. Whilst these papers could be said to represent a coherent development of a Minimalist theory over time, they do not, of course, identify a uniform contribution of Chomsky's but rather a useful summary of how general concerns have evolved in MP. Closer scrutiny of labelling in particular is postponed until \autoref{sec:200}, where the development of labelling theory is reviewed, and \autoref{sec:300}, which discusses the innovations since \textcite{ChomskyN_2013,ChomskyN_2015}.

Such a summary is not always provided in work in the domain of theoretical syntax, especially in more analytic work, but it is warranted here on account of the precision that can lack in theoretical syntax, and which this work sets out to begin to rectify. This is not intended as criticism---rather as testament to the nascent nature of the field and the sheer scale of complexity of the phenomena under investigation. As a metacritical aside, it may well be amusing though not picayune to suggest that Chomsky's own body of work resembles Aristotle's in a way criticised by Galileo explicitly in the aforecited work.%
\footnote{For similar comments from a different perspective, see \textcite{AsudehA.ToivonenI_2006}. For a considerably more disparaging assessment of Chomsky's recent work, see \textcite{BehmeC_2014,BehmeC_2015}. A full response to these criticisms would be far too great a tangent.}
When presented with such a string of references to a single author as above, an oeuvre that supposedly represents a coherent thread of theory alongside the more programmatic suggestions, one might be reminded of a certain passage from the second day of the \textit{Dialogo}. Protesting Salviati's dismissal of Aristotelian doctrines, Simplicio holds that, in order to be qualified to do so, ``one must have a grasp of the whole scheme, and be able to combine this passage with that, collecting together one text here and another very distant from it''---indeed, taking it a step further, he subsequently claims that ``[t]here is no doubt that whoever has this skill will be able to draw from his books demonstrations of all that can be known; for every single thing is in them.'' \parencite[108]{Galileo_1967}. Sagredo wittily replies: 

\begin{quote}\setstretch{1.0}
``I have a little book, much briefer than Aristotle or Ovid, in which is contained the whole of science, and with very little study one may form from it the most complete ideas. It is the alphabet, and no doubt anyone who can properly join and order this or that vowel and these or those consonants with one another can dig out of it the truest answers to any question[].'' \parencite[109]{Galileo_1967}
\end{quote}
\noindent
Thus, Galileo offers an addendum to this fascination with the alphabet and by extension with language which Sagredo proclaims on the first day, and which is so often cited by Chomsky, as discussed in \autoref{sec:110}. A goal of this thesis is to eliminate the overreliance on appeal to authority criticised by Galileo, truly approaching I-language ``from [the] bottom up'' \parencite[4]{ChomskyN_2007}---again, not mathematical game-playing, but seeking genuine explanation. The review of Minimalist theory provided in this subsection and in \autoref{sec:200} and \autoref{sec:300} makes clear how entwining the various skeins of MP is by no means a trivial task, albeit certainly a worthwhile endeavour.

\subsubsection[\CHL]{$\mathbfit{\CHL}$}\label{sec:141}

An I-language L $([F],\ \Lex,\ \MERGE,\ \AGREE,\ \TRANSFER,\ \fSM,\ \fCI)$ is a state of the faculty of language FL, a component of the human mind/brain.%
\footnote{Operations will be denoted in this subsection using capital letters, following the convention introduced by \textcite{ChomskyN.etal_2019}. When discussing operations without any particular theory in mind, CamelCase normal text will be used---this is employed e.g. in the review sections, \autoref{sec:200} and \autoref{sec:300}. For the formalised operations in \autoref{sec:400}, I adopt a different convention (see \autoref{fn:FormalConventions}).}
The initial state \Szero\ of FL, call this Universal Grammar UG, determines the set of \textit{features} available for all languages \setF, from which L selects a subset $[F]$, and assembles this subset into a lexicon \Lex\ consisting of lexical items (LIs). For each derivation, L selects a lexical array \LA\ from \Lex. UG also determines the computational procedure for human language \CHL\ which generates narrow-syntactic expressions, syntactic objects (SOs), out of LIs. In the definition of L above, \CHL\ consists of the three operations \MERGE, \TRANSFER\ and \AGREE. LIs are the `atoms of computation' for \CHL. An LI is an SO, termed a \textit{head} or \textit{minimal projection} within the context of a larger SO constructed by \CHL. On the simplest assumptions, \CHL\ is uniform for all L.%
\footnote{As proposed by \textcite[107]{ChomskyN_2004}, contra \textcite[100]{ChomskyN_2000}, where it is suggested that ``parameter setting'' be ``refinement of \CHL\ in one of the possible ways''. The theory presented here assumes that variation is confined to the lexicon, as elaborated further in \autoref{sec:150}.}
The operation \MERGE, part of \CHL, recursively constructs objects out of \LA, each of which can be mapped to a semantic representation \SEM\ by the \textit{semantic component} \fCI\ and to a phonetic representation \PHON\ by the \textit{phonological component} \fSM. The operation \TRANSFER\ hands an object generated in the narrow syntax NS by \MERGE\ to \fCI\ and \fSM, resulting in the expression $\Exp=\phonsem$. L generates a set of expressions \setExp, interpreted at the interfaces. \PHON\ is interpreted by sensorimotor systems SM, whilst \SEM\ is interpreted by conceptual-interpretive systems C-I.

\subsubsection{The interfaces}\label{sec:142}

Together, SM and C-I constitute the \textit{interfaces}, crucial to the minimalist approach proposed by \textcite{ChomskyN_1993}: ``all conditions are interface conditions; and a linguistic expression is the optimal realization of such interface conditions'' \parencite[26]{ChomskyN_1993}. As such, the interfaces SM and C-I and their respective mappings \fSM\ and \fCI\ deserve further attention. \fCI\ is assumed to be uniform for all L, \fSM\ is assumed to vary greatly. Indeed, following the Minimalist proposal \pxref{ex:minprops:4}, \fSM\ is the locus of all variation (a point that I will return to in \autoref{sec:150}). As aforementioned, \CHL\ is uniform, thus linguistic variation is confined to $[F]$, \Lex, and \fSM. \fSM\ also has the special property that it may introduce features from $[F]$ into the computation of \PHON, violating Inclusiveness \pxref{def:inclusiveness}, which necessarily holds only for \CHL. Very little is understood about \fCI, which may introduce features not present in SO, but we assume not from $[F]$ \parencite[107]{ChomskyN_2004}. If this does hold, it is sensible to include \fCI\ in the definition NS, as is typical in the literature.

Following \textcite[241]{ChomskyN.etal_2019}, there is no operation $SpellOut$, which eliminates structure before transfer to SM. Further, note that `PF' and `LF', as internal levels of representation, are undefined, following \textcite[107]{ChomskyN_2004}. Rather, there is a single, unified cycle, with \TRANSFER\ handing over syntactic objects, called \textit{phases}, to \fSM\ and \fCI. The specifics of the cycle and the precise nature of phases will be discussed further below, in \autoref{sec:145}.

An expression \Exp\ is said to \textit{converge at an interface level \IL} if it is legible at \IL, in other words if the interface condition \IC\ at \IL, $\IC(\IL)$, is satisfied. \IC\ states that ``the information in the expressions generated by L must be accessible to other systems'' \parencite[106]{ChomskyN_2004}, which is, evidently, a requirement that language be usable at all---the barest possible metric of ``good design'', a key methodological assumption of MP. By contrast, \Exp\ \textit{crashes at \IL} if it does not meet $\IC(\IL)$. By extension, the computation of \Exp\ \textit{converges} if \Exp\ converges at both SM and C-I, otherwise it \textit{crashes}. A derivation will only crash if it fails to remove all features from the resulting SO that are \textit{uninterpretable} at the interfaces before \TRANSFER\ takes place. A derivation that has removed all such features will always converge, but to varying levels of \textit{deviance} as determined by the interface systems---a suggestion of \textcite[112]{ChomskyN_2004}, reinforced by \textcite[238]{ChomskyN.etal_2019}: ``concerns about ``overgeneration'' in core syntax [i.e. NS---LVS] are unfounded; the only empirical criterion is that the grammar associate each syntactic object generated to a <SEM,PHON> pair in a way that corresponds to the knowledge of the native speaker ... ``overgeneration'' must be permitted on purely empirical grounds, since ``deviant'' expressions are systematically used in all kinds of ways''. This point will prove crucial with respect to the discussion of labelling to follow, and will also receive further attention in \autoref{sec:145}.

\subsubsection{Features and the lexicon}\label{sec:143}

Features require further attention: why should uninterpretable features exist at all in a system adhering to princples of ``good design''? The \textit{Interpretability Condition}, that ``LIs have no features other than those interpreted at the interface, properties of sound and meaning'' is ``transparently false'' \parencite[113]{ChomskyN_2000}. Rather, I-language is characterised by what \textcite[54]{BiberauerT_2019} calls ``\textit{systematic departures from Saussurean arbitrariness}''---the presence of so-called `formal', grammatical features which play a role in \CHL\ but not directly at the interfaces.

An idea that persists, from its introduction in \textcite[\pnfmt{277} \textit{et seq.}]{ChomskyN_1995} is that uninterpretable features exist to capture the displacement property of language---long considered an `imperfection', but accepted by \textcite[note 29]{ChomskyN_2004} as, in fact, the most Minimal option. On the original formulation by \textcite{ChomskyN_2000}, the fact that both of these then-considered `imperfections', uninterpretable features and displacement, appear to be intimately connected suggests that ``the two imperfections might reduce to one'' \parencite[121]{ChomskyN_2000}. Furthermore, the optimal conclusion would be that dislocation itself is required by design---either as part of \IC\ or as a consequence of the nature of the operations within \CHL. The latter is demonstrated to be the case by \textcite{ChomskyN_2004}, with the introduction of \textit{internal \MERGE} (IM) and \textit{external \MERGE} (EM), building upon the unification of syntactic operations begun by \textcite{KitaharaH_1997}. $\MERGE(X,Y)$ is considered IM if $X$ is contained within $Y$, else it is considered EM. Displacement, following the `copy' theory of movement, comes for free as a consequence of the nature of \MERGE, which is its simplest formulation does not bar access to objects that have already been merged.%
\footnote{As will be discussed in \autoref{sec:420}, the term `copy' is a bit of a misnomer, hence the scare quotes. It is nevertheless standard to assume some lossely defined form of copy theory in Minimalist work.}
Further consequences of this will follow.

\subsubsection[\AGREE]{$\mathbfit{\AGREE}$}\label{sec:144}

The existence of uninterpretable features forms part of the justification for \AGREE, the final component of L as stated above. \AGREE\ forms a relation between two SOs, one of which is termed a \textit{probe}, the other a \textit{goal}. Standardly, the probe must c-command its goal, although there are other possibilities, which will be considered in the formalisation in \autoref{sec:480}. The goal is located via some mechanism of MS, again to be clarified in \autoref{sec:450}.

In many older articulations of the theory, \AGREE\ is taken as part of the more complex operation Move, which is composed of \MERGE, \AGREE, and a third operation, pied-piping, which remains poorly understood but is given much attention, for example in \textcite{ChomskyN_1995}. Following \textcite{ChomskyN_2004}, Move will not be taken to be a part of the theory. The suggestion is that all of its empirical import can be taken over with only the more minimal operations of \CHL, in combination with IC and third factors. Labelling has much to reveal here, as will be discussed in the following sections. Indeed, whether \AGREE\ is needed at all will come under scrutiny. A preliminary motivation for this is that both labelling and \AGREE\ are effectively realisations of MS. This strongly implies some kind of redundancy. If this is the case, and \AGREE\ can be abandoned, this would lead to a simplification of \CHL, a move in line with both methodological \pxref{ex:minprops:1} and ontological minimalism \pxref{ex:minprops:2}. This is the approach taken in Chomsky's most recent work, where \AGREE\ receives almost no mention \parencite{ChomskyN_2021}. Further discussion is reserved for \autoref{sec:400}.

\subsubsection{Phases and cyclicity}\label{sec:145}

In the interaction of the subcomponents of L, a need arises to identify the units that are available to take part in an operation within \CHL. Assume therefore that \CHL\ operates within a workspace \WS, which represents the state of a derivation at any particular point.%
\footnote{The idea of the workspace within the context of modern minimalism was most notably formalised by \textcite{CollinsC.StablerE_2016}, and finds further elaboration by \textcite{ChomskyN.etal_2019} and \citeauthor{ChomskyN_2019a} (\citeyear{ChomskyN_2019a}, \citeyear{ChomskyN_2019b}, \citeyear{ChomskyN_2021}).}
Operations are ``strictly Markovian'' \parencite[20]{ChomskyN_2021}, beyond even the standard Markovian property of derivations---\WS\ does not contain previously generated items, since these are eliminated by \MERGE, in accordance with a property of computational optimality termed \textit{Minimal Yield} \parencite[MY,][19]{ChomskyN_2021}, equivalently \textit{Restrict Resources} \parencite{ChomskyN_2019a}. The formal properties of derivations beyond this will be explored in more depth in \autoref{sec:430}.

The SOs generated by \CHL\ are assumed to be \textit{bare}, in the sense of \textcite{ChomskyN_1994}. Equivalently, they are formed only by the operation \MERGE. With the notion of \WS\ established, it is possible to define \MERGE\ as a function between workspaces%
\footnote{\WS\ is analogous to a ``working memory''---in a computational, not necessarily a cognitive sense, much like the tape of a Turing machine. See \textcite{WatumullJ_2012,WatumullJ_2015} for a possible formalisation of the linguistic Turing machine, in which these issues come to light.}
In previous formulations, \MERGE\ is typically considered to be a binary operation, which takes two SOs $X$ and $Y$ and combines them to form the set $\{X,Y\}$, itself an SO. For \textcite{ChomskyN_2021} who borrows much from \nptextcite{CollinsC.StablerE_2016}: \MERGE\ operates on a sequence of SOs $\sigma$, such that each SO in $\sigma$ is accessible and that $\sigma$ exhausts \WS; \MERGE\ is free to take any two objects and merge them together, mapping \WS\ to a new workspace $WS'$. The definition in \autoref{sec:400} offers a more precise account. An important consequence is that the application of \MERGE\ is free, in the sense of \textcite{ChomskyN.etal_2019}, meaning that constraints on \MERGE\ must fall out from the conjunction of IC, third factors, and other operations such as \AGREE. Labelling, it will be argued, surely also plays a role. This therefore does not entail that \MERGE\ must be `triggered', as assumed in stricter MGs and many other Minimalist theories such as that of \textcite{AdgerD_2003}.

Finally, it is worth briefly characterising the core functional categories (CFCs) and, in turn, the nature of phases. Following \textcite{ChomskyN_2000}, the CFCs are taken to be C, expressing force and mood (and possibly abbreviating a number of categories taken to form the \textit{left periphery}, following \nptextcite{RizziL_1997}), T, expressing tense and event structure, and v*, the light verb head of transitive constructions, expressing argument structure.%
\footnote{Cf. \autoref{fn:littleV} on \littleV.}
These functional categories are `core' in the sense of being the locus of agreement and dislocation generally. Following \textcite{ChomskyN_2008}, CP and v*P are phases, C and v* their respective \textit{phase heads}. This is arguably problematic, as T is very obviously involved in Case, \phiF-feature agreement and movement---with the EPP%
\footnote{Extended Projection Principle, classically formulated as the requirement that [Spec,TP] be filled (cf. \nptextcite{ChomskyN_1981}). Now, EPP-features are interpreted more generally, as the requirement that a head needs its specifier to be filled, usually by movement of the external argument in the case of T. This is the \textit{generalised} EPP-feature \parencite[see][]{HaegemanL_1996,LaenzlingerC_1998,RobertsI_2004}. EPP-features may be obviated by labelling and other considerations, as discussed further in \autoref{sec:300}. Eliminating the EPP has been a long-term goal in generative syntax---cf. \textcite{BoskovicZ_2007}.}
being a classic example. \textcite{ChomskyN_2008} resolves this by making explicit the idea of \textit{inheritance}: ``for T, \phiF-features and Tense appear to be derivative, not inherent: basic tense and also tenselike properties (e.g. irrealis) are determined by C (in which they are inherent)'' \parencite[143]{ChomskyN_2008}.%
\footnote{The earliest published mention of inheritance comes from \textcite{ChomskyN_2007}, actually inheriting the idea from Marc Richards, later published as \textcite{RichardsMD_2007}, which works off of a manuscript version of \textcite{ChomskyN_2008} distributed even earlier.}
Thus, ``Agree and Tense are inherited from C, the phase head'' \parencite[143-144]{ChomskyN_2008}.

Derivations proceed \textit{strictly cyclically}, phase-by-phase. Further, \CHL\ constructs objects in parallel in the workspace, but all operations occur effectively instantaneously at the phase level. As stated by \textcite[116]{ChomskyN_2004}: ``TRANSFER has a ``memory'' of phase length, meaning [] that operations at the phase level are in effect simultaneous''. This, presumably, makes the apparent countercyclicity of inheritance only apparent. Further, operations apply freely---order does not matter; any deviant or crashing derivations that result are discarded by the interfaces. More conclusions are possible, as reiterated by \textcite[143]{ChomskyN_2008}: ``along with Transfer, all other operations will also apply at the phase level, as determined by the label/probe. That implies that IM should be driven only by phase heads''. Labels clearly play a significant role: the label of the phase is always the probe for \AGREE\ (obscured by the fact that the agreement properties of T are inherited from C). The interactions between labelling and agreement are discussed in \autoref{sec:480}.

One of the most important consequences of strict cyclicity is the Phase-Impenetrability Condition (PIC) as in \pxref{ex:PIC}, from \textcite[108]{ChomskyN_2000}.

\begin{example}\label{ex:PIC}
\setlength{\parskip}{0pt}\setlength{\parsep}{0pt}
\textit{Phase-Impenetrability Condition}

In phase $\alpha$ with head H, the domain of H is not accessible to operations outside $\alpha$, only H and its edge are accessible to such operations.
\end{example}
\noindent
The PIC proves critical in discussions of locality, taking the place of Subjacency \parencite{ChomskyN_1973} and Barriers \parencite{ChomskyN_1986} in previous frameworks. The place of labelling within the context of locality and the phase will be a key point of analysis within \autoref{sec:300}.


\subsection{Interfaces, variation and variability}\label{sec:150}

It is worth breifly expanding upon \autoref{sec:142} in order to provide some more detail on the precise status of the `interfaces' in contemporary Minimalist theory, in particular with respect to features.

The concept of `features' is elaborated in \autoref{sec:143}. Following the Borer-Chomsky Conjecture \parencite{BakerMC_2008}, all linguistic variation---equivalently, everything that is learned; the second factor---is restricted to the lexicon, and hence to the arrangement of lexical features within the lexicon, and their treatment by I-language. One could question exactly what these features are---some properties, like interpretability, have already been discussed. One may further question how distinct the difference sets of features (phonological, semantic, and syntactic) are. Standardly, features come in interpretable-uninterpretable (or valued-unvalued) pairs, where interpretability is an interface property, as will be assumed in \autoref{sec:460}. Nevertheless, some authors argue for a `substance-free' system, adopting the term introduced by \textcite{HaleM.ReissC_2008}: a system is \textit{substance-free} if it involves ``computation over abstract mental entities'' \parencite[22]{HaleM.ReissC_2008}, in other words being \textit{symbolic} in the sense applied to cognitive science by \textcite{PylyshynZW_1984,GallistelCR_2001,GallistelCR.KingAP_2010}, inter alia. By contrast, a theory in which `symbols' are actually embodied in percepts would be considered \textit{substanceful}. For example, a phonological theory in which phonemes directly correspond to aspects of phonetic substance, as in standard generative phonology \parencite{ChomskyN.HalleM_1968}, is substanceful. In the present case, a syntactic theory in which categories `leech' off of semantic properties would be considered substanceful. \textcite{ZeijlstraH_2014} argues for this approach in syntax, although a ful discussion of this would take us too far afield.

Classic formulations allow the label of a syntactic object to be a (categorial) feature, a bundle of features, or a lexical item, and in more recent approaches more complex objects are allowed to serve as labels (see \autoref{sec:300}). As such, an understanding of the feature inventory \setF\ will be essential to formalising the answer to the question of what can be a label. The question of what (kinds of) features are allowed thus has direct bearing on the topic of this thesis. However, adopting the BCC, as typically done in Minimalist work, entails that features also have a significant impact on the class of humanly computable I-languages.%
\footnote{Using the more precise term introduced by \textcite[3]{HaleM.ReissC_2008}, as opposed to a weaker alternative like `possible languages'. Adapting ideas originating in Evolutionary Phonology \parencite{BlevinsJ_2004} to I-language generally, the set of possible I-languages may be irrecovably restricted by historical, cultural, and anthropological factors of a very different nature to the first and third factors considered here to go into a theory of I-language. This is, effectively, a truism that emerges upon consideration of the second factor: the data of the environment have no \textit{a priori} justification to be the way they are except for the fact that they are \textit{a priori} constrained by first and third factors. A different course of history could have led to there being a completely mutually exclusive set of possible languages available to the linguist to study and the child to learn throughout time, but if a child from our world were to travel to this hypothetical world, we still want to say that they could learn the language. Hence the concept of \textit{humanly computable}.}
Thus, the question of language variation---in effect, rephrasing \pxref{def:universality}---is unavoidable, but also inevitable in a feature-based theory. These issues will be explored further in the subsequent discussion.

Another aspect of the theory presented in \autoref{sec:140} is that \MERGE\ can apply freely. There seem to be a number of good reasons, theory-internal but formalisation-external, to adopt the free Merge approach. The approach receives particular justification by \textcite{ChomskyN.etal_2019}, and further reasons crop up on occasion in the following discussion. The general point is that `overgeneration', traditionally thought of as the bane of a sound theory, is actually good, if considered in a restricted manner. Namely, it allows `deviant' structures to be generated, which perhaps satisfy constraints at one interface, but are to some variable degree uninterpretable at another. This brings the theory of I-language more in line with general empirical observations, in which grammaticality is a gradient property of expressions \parencite{SprouseJ.etal_2018}.


\subsection{Summary and outline}\label{sec:160}

This introduction has served to illuminate the concerns central to the biolinguistic research programme in which context this thesis is situated. It has provided some novel synthesis of the most recent ideas in MP and provided broader comment on the ways in which the Galilean challenge can be tackled.

Labelling itself is to receive more attention in the following sections. \autoref{sec:200} constitues a historical review of approaches to labelling, beginning with the earliest work in generative grammar. The conclusions from this section frame the discussion in \autoref{sec:300}, which unravels the central issues within contemporary approaches to labelling. The formalisation itself comes in \autoref{sec:400}, which will extend \citeposs{CollinsC.StablerE_2016} formalisation of Minimalist syntax, including a novel definition of MS and, in turn, the labelling algorithm. The overarching goal is to create a formal model of syntax that is both internally consistent and has the potential to provide genuine explanation of linguistic phenomena. As will become clear in the following sections, a precise understanding of labelling is essential to this.

