\section{Conclusion and future prospects}\label{sec:500}

The formalisation programme instigated by \CS\ is a novel one, and one fraught with complication as a consequence of the nascence of Minimalist theory and its diverse fragmentation. The formalisation process has thus exposed the fragility of some of the formal concepts underlying Minimalist syntax, particularly as they pertain to labelling. Elements of the formalisation presented in \CS\ have been majorly revised, for two reasons: (a) to bring the formalisation more in line with recent developments in MP, and (b) to introduce a novel approach to MS and labelling.

As clear from the outset, empirical discussion has not been the focus, in the interest of providing a clear formal grounding, making some steps towards unifying the diversity of labelling theories that abound in the literature. Space constraints have prevented discussion of even simple examples which would demonstrate how the algorithms proposed in \autoref{sec:400} derive syntactic structures. There is clear scope for future work in this area. Particular pertinent structures which could be analysed in this framework are standard cases of \textit{wh}-movement, English passives, and also the structures given in \pxref{ex:empirical} below.

\begin{subexamples}\label{ex:empirical}
    \item\label{ex:japsubj} 
        Japanese multiple subject constructions \parencite{KunoS_1973,SaitoM_2016,EpsteinSD.etal_2020}:

        \digloss{Bunmeikoku-ga dansei-ga heikin-zyumyoo-ga mizika-i}
                {civilized.country-\NOM{} male-\NOM{} average-life.span-\NOM{} short-\PRS{}}
                {It is in civilized countries that male’s average life span is short.}

        \parencite[example from][131]{SaitoM_2016}

    \item\label{ex:romverb}
        Verb movement in Romance (\nptextcite{SchifanoN_2018} and Ian Roberts, p.c.)

        \digloss{Antoine confond probablement (*confond) le poème}
                {A. confuse.\TSG.\PRS{} probably {} the poem}
                {A. is probably confusing the poem.}

        \parencite[example from][63]{SchifanoN_2018}

\end{subexamples}

Consideration of these will certainly require adjustments to the formalisation. It is hoped that, by making the formal representations and operations that are employed in the literature more explicit, further insight can be gained into the operation of FL on the computational and algorithmic levels.
