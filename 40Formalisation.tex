\section{Formalisation}\label{sec:400}

The formalisation to follow is framed in a manner inspired by the presentation of \textcite{CollinsC.StablerE_2016}, henceforth \CS. In \autoref{sec:410} through \autoref{sec:440}, fundamental definitions are adapted from \CS. In most cases, however, this adaptation is not verbatim; rather, numerous adjustments are documented, in line with the discussions within the preceding sections. \autoref{sec:450} presents the core of the labelling algorithm: the procedure of minimal search. This subsection builds upon the algorithm presented by \textcite{KeH_2019}, adapting it into the style of \CS\ and adjusting its operation in line with the results of the preceding investigation. \autoref{sec:460} presents a barebones formalisation of features, enabling the definition of the labelling algorithm itself in \autoref{sec:470}. Agreement is given brief attention in \autoref{sec:480}. Finally, some key theoretical results are discussed in \autoref{sec:490}.

Before embarking, there are some initial caveats to bear in mind. Firstly, the model assumed by \CS\ is one which does not assume in full the proposals of Distributed Morphology (DM), in particular the mechanism of Late Insertion \parencite{HalleM.MarantzA_1993}. Adopting this aspect of DM would require a major reformulation of Transfer, which is beyond the scope of the present work (cf. \nptextcite[f.n.~2]{MilwayD_2021} for a similar point). Late Insertion entails that lexical items are not directly bundled with phonological features, but such features will play only a limited role in this formalisation in any case. Nevertheless, one of the assumptions within DM will be adopted, as it is now now commonly held within a range of theoretical approaches. Namely, lexical \emph{roots} are considered to be `bare', without any \emph{syntactic} features (see \autoref{def:lexroot} and \nptextcite{MarantzA_1997, BorerH_2005, BorerH_2005a, BorerH_2013, BaukeL.BlumelA_2017a}). Properties such as category are instead provided by closed class categorisers like \littleN, \littleV, and \littleA.%
\footnote{\label{fn:littleV}One consequence of this is that categoriser \littleV\ must be totally distinct from the `light verb' \littleV\ that appears above VP and introduces the external argument. \littleVP\ has a complicated history, developing out of Larsonian VP-shells \parencite{LarsonRK_1988}, termed `light v(erb)' by \textcite[315-316]{ChomskyN_1995}, and extended by \textcite{KratzerA_1996}, who terms the phrase VoiceP. VoiceP is also used by subsequent researchers, but tends to carry more theoretical baggage. In the interest of staying true to the literature, especially in the context of phases, where \littleVP\ standardly refers to the lower (thematic) phase, I will use \littleV\ to refer to both the light verb and the categoriser; the semantics should be clear from the context.}
I will follow \textcite{ChomskyN_2015} in \autoref{sec:310} by assuming that roots cannot provide a label (as a result of having no syntactic features to project) and that, consequently, the root must Merge directly with its categoriser as the first step of a derivation. The root is thus the zero-element of \textcite{WatumullJ_2015}; the start symbol of the derivation.%
\footnote{Note that this contradicts the assertion of \textcite[7]{AdgerD.RobertsI_} that it is phase heads which provoke the start of computation. This cannot literally be true, however, since computation must begin before the phase head is introduced, since the phase head can itself only be introduced by a computational operation (in my formalisation, \Select, see \autoref{sec:430}). This issue is discussed further in \autoref{sec:440}.}

Basic (naïve) set theory is assumed. Standard notation is as follows, borrowing partly from \CS[43]. Sets are written with curly braces $\{...\}$ and are unordered. The following symbols are used to represent relations between sets and their elements: $\in$ (is a member of), $\cup$ (union), $\cap$ (intersection), $\subseteq$ (is a subset of), $\subset$ (is a proper subset of).%
\footnote{For a set $A$, $A\subseteq A$ but $A\nsubset A$.}
The empty set is written $\emptyset$ or $\{\}$. Given sets $A$ and $B$, the set difference $A-B=\{\ x\ |\ x\in A,\ x\notin B\ \}$. A sequence $\langle ... \rangle$ is ordered; the empty sequence is written $\varepsilon$ or $\langle\rangle$. A sequence of length 2 is a pair, one of length $n$ is an $n$-tuple. Free variables are assumed to be universally quantified, such that $x$ is shorthand for $\forall x,\ x$. The Cartesian product is represented by $\times$; the following shorthand using indices will also appear: $X^n = X_1 \times X_2 \times ... \times X_n$. The arbitary union is notated as $\bigcup X = x_1 \cup ... \cup x_n$ for a set $X=\{x_1,...,x_n\}$, and can also be given limits. The Kleene closure $A^* = \bigcup_{i=0}^{\infty} A^i$.%
\footnote{For a fuller explanation of the elements of naïve set theory beyond the scope of this thesis, see \textcite{KaplanskyI_1972,EndertonHB_1977}. See \textcite[87]{HopcroftJE.etal_2013} on Kleene closure (a.k.a. the `Kleene star').}
A \textit{function} is a mapping between sets, and can take any number of \textit{parameters} (when a function is invoked, these will be called \textit{arguments}). For notational ease, where a set is passed in as an argument, it may either be taken itself as the entire argument, or as shorthand for its members being the arguments (see \autoref{def:matchLI} for an example of the latter case).

Tree diagrams will occasionally be used in place of complex bracketed sets. These trees will often encode more information than is present in the sets themselves (such as labels and linear order) for expositional reasons, as is standard. For lack of space, the precise relationship between graph-theoretic trees and sets will not be explored. It suffices to say that there is a surjective (many-to-one) function $\Zeta(t)$ which maps a tree $t$ onto its corresponding SO---multiple trees could be used to represent the same SO.%
\footnote{See \textcite[Chapter 15]{AvigadJ.etal_2017} on functions in set theory.}

\subsection{Preliminaries}\label{sec:410}

\CS\ provide a number of definitions which lay out the foundations of a formalisation of Minimalist syntax. Many of these will, however, need some revision in light of the preceding discussions in \autoref{sec:100}, \autoref{sec:200} and \autoref{sec:300}. The purpose of this subsection is to lay out a revised set of fundamental definitions.

Naturally, the first definition to come is that of I-language itself. (cf. \autoref{sec:141}).%
\footnote{\label{fn:FormalConventions}Some notes on conventions: sets are indicated with capital letters (or words in all caps), functions with $CamelCase$.}

\begin{definition}\label{def:UG}
    \setstretch{1.0}
    UG is a 9-tuple:%
    \[UG=\langle \Fphon, \Fsyn, \Fsem, \Select, \Merge, \Agree, \Label, \Transfer, \FormCopy, \ffSM, \ffCI \rangle\]%
    where $\Fphon\cap\Fsyn=\Fsyn\cap\Fsem=\Fsem\cap\Fphon=\emptyset$.
\end{definition}
\noindent
Thus, I-language consists of three non-intersecting sets of features, and six further functions. Although termed UG, constituent operations are intended to draw on domain-general mechanisms as much as possible, in line with OM and MM. The choice of feature sets is intended to be as theory-neutral as possible, without making any claims as to the precise structure of features and lexical items except where necessary. In the elaboration of \Label\ and \Agree, some development of the feature theory will be required.%
\footnote{For one possible formal theory of features, see \textcite{AdgerD_2010,AdgerD.SvenoniusP_2011}. Cf. also \textcite{CarlsonJOE_2010,SongC_2019,StockwellR_2015,RobertsI_2019} for theories of features with more or less coverage and with varying degrees of formality.}

The definition is notably more complex that that adopted by \CS. Following \textcite{MilwayD_2021}, I have included \Agree\ as a function. I have also added \Label, to be defined. As per \autoref{sec:141}, \ffSM\ and \ffCI\ are the phonological and semantic mappings respectively. I include these as they ought to be defined in a complete formal theory of I-language, although they will receive little attention here. \FormCopy, an operation adapted from \textcite{ChomskyN_2021}, receives justification in \autoref{sec:420}.

As described in \autoref{sec:100}, UG is (broadly) species invariant. Variation is accounted for via the lexicon, \LEX, as per the BCC. In \Szero, $\LEX=\emptyset$. In the final state, \LEX\ consists of lexical items, composed of features. This is accounted for in the following definitions, lifted from \CS.

\begin{definition}
    A lexical item is a triple: $\LI=\langle \PHON,\SYN,\SEM \rangle$ where $\PHON\in(\Fphon)^*$, $\SYN\subseteq\Fsyn$, and $\SEM\subseteq\Fsem$.%
\end{definition}

\begin{definition}\label{def:lexroot}
    An LI $\langle \PHON,\SYN,\SEM \rangle$ is a \textit{lexical root} (L-root) iff $\SYN=\emptyset$.%
        \footnote{I leave open the possibility that \SEM\ is empty for roots (see \nptextcite{BorerH_2013}). Since I leave the feature theory in \autoref{sec:460} very vague, this is not a problem. Note however that if `interpretable' features are members of \Fsem\ and can serve as labels, this would leave \autoref{thm:rootinvis} underivable.}
\end{definition}

\begin{definition}
    A \textit{lexicon} is a finite set of LIs.
\end{definition}

\begin{definition}
    I-language is a pair $L = \langle \LEX, UG \rangle$, where \LEX\ is a lexicon.
\end{definition}


\subsection{Copies and repetitions}\label{sec:420}

The copy/repetition distinction is a fundamental problem in Minimalist syntax. The problem lies in distinguishing between minimally distinct syntactic objects like those in \pxref{ex:coprep}, where the `John' sister of V in \pxref{ex:coprep:cop} represents a moved element (copy; lower copies indicated with angle brackets), and in \pxref{ex:coprep:rep} a repetition (referring to two different people, on the standard reading).%
\footnote{Morphological complications like affix-hopping \parencite{ChomskyN_1975a} are ignored. The two v's represent either the light verb or the verbal categorisers, as contextually clear (see \autoref{fn:littleV}). For a theory of English passives with respect to labelling, see \textcite{BurrowsER_2022}.}

\begin{subexamples}\label{ex:coprep}
    \item\label{ex:coprep:cop} \{John, \{was, \{$_{vP}$ \copySO{John}, \{$_{vP}$ v+seen, \{$_{VP}$\{v, \copySO{see}\}, \copySO{John}\}\}\}\}\} (passive object raising to subject via phase-edge)
    \item\label{ex:coprep:rep} \{John, \{T, \{$_{vP}$ \copySO{John}, \{$_{vP}$ v+saw, \{$_{VP}$\{v, \copySO{see}\}, John\}\}\}\}\} (standard declarative)
\end{subexamples}

This problem is unavoidable in a formalisation of Minimalist syntax---here, I intend to take a somewhat novel approach. \autoref{sec:421} will briefly review some of the options and set out a path forward. \autoref{sec:422} will go into some more depth on the formal nature of `copies'.

\subsubsection{Distinguishing copies and repetitions}\label{sec:421}

Following TLTB, GB-era symbols like \textit{trace} and indices cannot be introduced to mark movement; this would also violate the NTC. Secondly, recall that there is no distinction between Merge and Move; rather, all structure building is by Merge (and by further hypothesis all that reaches the interfaces has been constructed by Merge). A syntactic object constructed by `internal' Merge is identical to one constructed by `external' Merge (see \autoref{def:merge} below). Further, to account for `trace-invisibility' effects with respect to movement (discussed above), both NS and the interfaces need to be able to distinguish copies and repetitions. Therefore, NS, or at least all relevant operations within NS, need to be able to distinguish copies and repetitions.

Despite the centrality of this issue, \textcite{CollinsC.GroatEM_2018} come to the worrying conclusion in their review that ``no adequate proposal exists in [M]inimalist syntax for distinguishing copies and repetitions'' \parencite[2]{CollinsC.GroatEM_2018}. \textcite{CollinsC.GroatEM_2018} review a number of approaches; I will briefly touch on three, two of which are discussed by \textcite{CollinsC.GroatEM_2018}. One option is to use \textit{chains}. As shown by \CS, however, chains introduce a vast amount of machinery that complicates the definition of Merge, and should be abandoned for the sake of TLTB. Further, \CS\ prove (in their Theorem 4) that chain-based structures and the multidominance structures they use in the rest of their paper are isomorphic, which one can infer makes the chain-based theory (or at least the formulation they adopt) inferior. A second proposal, adopted by \CS\ but not discussed by \textcite{CollinsC.GroatEM_2018}, is to augment each LI into a \textit{lexical item token} (\LIk) when introduced into the lexical array. An \LIk\ is an LI with an associated unique index $k$. Copies of the same \LIk\ will thus have the same index. This option evidently violates Inclusiveness, but one could argue that this is a principled violation, since otherwise the EM/IM distinction would be unformulable.

Nevertheless, I do not adopt \LIk s in this formalisation, in the interest of staying as true as possible to the recent literature, especially \textcite{ChomskyN.etal_2019} and \textcite{ChomskyN_2021}. The third option adopted in these works is the idea of a \textit{phase-level memory}. As noted by \textcite[12]{CollinsC.GroatEM_2018}, Chomsky separately notes two possibilities. A third is proposed by \textcite{ChomskyN_2021}. These options are summarised in \pxref{ex:phasemem}.

\begin{subexamples}[preamble={\textit{How could phase-level memory distinguish copies and repetitions?}}]\label{ex:phasemem}
    \item\label{ex:phasemem:1} It must be the case that ``within each phase each selection of an LI from the lexicon is a distinct item, so that all relevant identical items are copies'' \parencite[145]{ChomskyN_2008}.
    \item\label{ex:phasemem:2} ``At TRANSFER, phase-level memory suffices to determine whether a given pair of identical terms Y, Y$'$ was formed by IM.'' Y and Y$'$ are copies if so, else they are repetitions \parencite[246-247]{ChomskyN.etal_2019}.
    \item\label{ex:phasemem:3} There is a ``convention'', ``\textsc{Stability}'', which states that certain occurrences of the same symbol are related; there is a rule ``\textsc{FormCopy} (FC)'' which assigns the \textit{Copy} relation to certain idential symbols and which must adhere to \textsc{Stability}. ``FC applies at the phase level and is interpreted (mapped to CI), not entering into further computation'' \parencite[16-17]{ChomskyN_2021}.
\end{subexamples}

\textcite{CollinsC.GroatEM_2018} interpret both \pxref{ex:phasemem:1} and \pxref{ex:phasemem:2} as being problematic for the same reason. Briefly, with reference to \pxref{ex:phasemem:2}, establishing whether IM or EM was applied in a particular derivational stage would require access to the previous state of the workspace, but this violates the strict Markovian property of the derivation (cf. \nptextcite[20]{ChomskyN_2021}), with dire consequences for interpretation. \textit{Pace} \textcite{CollinsC.GroatEM_2018}, I interpret \pxref{ex:phasemem:1} to be equivalent to introducing \LIk s as done by \CS, which affords phase (lexical array) level uniqueness to tokens. It is thus unsatisfactory by Inclusiveness (\textit{pace} the claim of \nptextcite{ChomskyN_2008}). The fate of the original conception of the \textit{Numeration} \parencite{ChomskyN_1995} fares similarly.

\pxref{ex:phasemem:3} is more cryptic yet at the same time suggestive. The strong claim, as I interpret it, is as follows. The copy/repetition distinction is required \textit{only} at the interfaces---hence, syntactic operations cannot make reference to copies or repetitions. Further, the distinction is \textit{determined} at the interface, not in the syntax. The corresponding illegitimacy or deviance of a derivation with respect to misinterpretation of copies/repetitions emerges from interpretation, but this interpretation, like Merge, may operate freely. To take SM as illustrative: the structure \pxref{ex:coprep:cop} could be pronounced ``John was seen John'', but in this case the two `John's would be parsed as repetitions by a rule of SM and so would be interpreted as gibberish at C-I by the $\theta$-criterion (each DP must be assigned one and only one \thetarole; cf. \nptextcite[36]{ChomskyN_1981}).%
\footnote{In \pxref{ex:phasemem:3}, \textcite{ChomskyN_2021} appears to imply that the copy/repetition distinction is required only at C-I---this cannot be correct, as SM needs to be able to deduce lower copies to obviate their pronunciation. It must be a part of \Transfer. This being said, it seems sensible that copy \emph{formation} not be forced by SM, since lower copies \emph{can} be pronounced, as evidenced above, and indeed \emph{are} pronounced in certain contexts in certain languages and in child language.}
Indeed, this close interaction between the interfaces is an interesting result of the hypothesised proximity of the interfaces to NS (and thus to each other) established in \autoref{sec:220}. This has further implications for minimality, which will be touched upon in \autoref{sec:460}.

Fully exploring the consequences of the FC model proposed by \textcite{ChomskyN_2021} is well beyond the scope of the present work. In particular, it will be needed to establish what impact this has on the analysis of island effects. Nevertheless, in the interest of being forward-looking, it will be adopted, accepting its preliminary nature as a caveat. With this established, it is possible to continue the formalisation. As a result of abandoning \LIk s, there will be some small adjustments in the definitions to follow as compared to \CS.

\subsubsection{Copies and multidominance}\label{sec:422}

It is important to note that, thus far, the notion `copy' has been assumed in a non-technical sense. As is clear from the discussion in \autoref{sec:140}, the generally assumed, intuitive idea is that an internally-Merged object is identical in its source and target positions. In a formalisation that uses some set-theoretic machinery, some necessary properties become apparent. For instance, a standard assumption is that sets contain unordered, unique objects---i.e. $\{a, a, b\}=\{b, a\}$. This being the case, take a structure that could plausibly be the output of IM, $X=\{a, \{a, b\}\}$. \textcite{GartnerHM_2022} points out that, on standard set-theoretic assumptions, the two instances of $a$ are not `copies'; they are identical objects. [I shall recapitulate the proof below, after Merge has been defined. which assumes the existence of a `bracket-erasure' function $sp$, and that $a$ and $b$ are \textit{urelements}, viz. indivisible.] What this entails, then, is that a \textit{multidominance} approach \parencite{CitkoB_2011, CitkoB_2011a} to SOs appears to be in line with Minimalist assumptions. In other words, for the two trees $t$ and $s$ in \pxref{ex:multidom}, for the corresponding SOs $\Zeta(t)=\Zeta(s)$.

\begin{subexamples}\label{ex:multidom}
    \item $t=$
        \begin{forest}
            [{$\alpha$} [,phantom [Y,name=a]] [o,name=b] [X [o,name=c] [Z]]]
            \draw (a) -- (b);
            \draw (a) -- (c);
        \end{forest}
    \item $s=$
        \begin{forest}
            [{$\alpha$} [Y] [X [Y] [Z]]]
        \end{forest}
\end{subexamples}
\noindent
Note that the graph-theoretic complications entailed by the representation $t$ have no theoretical status within the formalisation under discussion, they are merely the result of diagrammatic games \parencite[cf.][]{ChomskyN_2019a}. For instance, from the diagram it appears that there are two nodes that could be considered `roots', $\alpha$ and Y, say if `root' were defined graph-theoretically as `a node not dominated by another node'. However, from the set-theoretic representation \pxref{ex:multidom:set}, which is the only one that has any theoretical status having been constructed by \Merge, it is clear that there is no such confusion.

\begin{example}\label{ex:multidom:set}
    $\alpha=\Zeta(t)=\Zeta(s)=\{_\alpha\ Y, \{_X\ Y, Z\}\}$
\end{example}
\noindent
Similarly, one could protest that this would eliminating the `binary-branching' property of syntactic trees, since if, say, Y were merged with $\alpha$, this would result in a structure in which, diagrammatically, Y would have three branches connecting it with each of its occurrences. Again, however, this property of the tree diagram has no theoretical status. Indeed, (internally) Merging Y with $\alpha$ in this way would be entirely legitimate and would continue to satisfy the Extension Condition (see \autoref{sec:490}). A further illegitimate operation, namely externally merging an LI with Y, would also not be possible, since Y is not a root (as per \autoref{def:root} below). 

Returning to \citeposs{GartnerHM_2022} original concern: \textcite{ChomskyN.etal_2019} claim that multidominance approaches are misguided, precisely because they suggest the existence of ``complex graph-theoretic objects [that] are not defined by simplest MERGE''. Following the argument as set out here, this concern is unwarranted. What is part of the system is \Merge, which forms sets, and without added complication, the \textit{urelements} (indivisible elements) of these sets from the perspective of \Merge\ are LIs, which entails that multiple occurrences of the same LIs are not copies but one and the same object. \textcite{GartnerHM_2022} concludes that when analysing the formal properties of a system, one must first note whether the formal tools in use are being applied at the meta level, talking `about' I-language, or whether they are at the object level, namely part of the system itself. Secondly, one must be aware of the difference between notation and content, avoiding falling into the trap of ``excess notation over subject matter'' \parencite[5]{QuineWV_1941}. In answer to both of these points, the subject matter at hand is I-language, in particular the representations constructed by I-language, namely SOs and the labels of these SOs (features). SOs are sets formed by \Merge\ using objects from the lexicon, also hypothesised to be sets. Nevertheless, properties of sets outside of the fact that they are formed by \Merge\ should not be considered \textit{a priori} allowed. In this formalisation, I assume only the most basic---in particular, set membership, and the natural operations of union and intersection, and compositions of these operations. \textcite{GartnerHM_2022} suggests calling these sets, with limited properties, `M-sets', although I do not adopt this terminology here. Another mathematical object in use alongside sets is functions, in particular the notion borrowed from computer science, following standard computational/cognitive assumptions (cf. \nptextcite{GallistelCR.KingAP_2010}).

As a final note on multidominant structures: I am not introducing the full complexity of multidominance as set out by \textcite{CitkoB_2011a}. This theory requires complications to be introduced to the Merge operation itself, namely `Parallel Merge' and `Sidewards Merge', which are independently ruled out by a principle of computational optimality noted by \textcite{ChomskyN_2019a,ChomskyN_2021} related to accessibility, which I derive in \autoref{thm:MY}.

\subsubsection{Defining occurrences}

There are, however, occasions where different occurrences of SOs within a set-theoretic structure need to be distinguished. One can define the notion occurrence to handle this. \CS\ define occurrence in terms of immediate containment, presenting this alongside a number of useful definitions and theorems. I will not repeat these here, although I will adopt a compatible definition of occurrence which suits our present purposes, and which is (implicitly) corroborated by \textcite{EpsteinSD.etal_2020}.

\begin{definition}
    An \textit{occurrence} of an SO $A$ is a \textit{path}, a sequence of SOs $P = \langle X_1, ... , X_n \rangle$ where for all $0 < i < n$, $X_{i+1} \in X_i$, such that $X_n = A$. An occurrence of $A$ at \textit{position} $P$ is denoted $A_P$.
\end{definition}

This definition of occurrence would also enable a formalisation of \FormCopy. Since this would take us too far afield, I leave this for future work.


\subsection{Merge, workspaces and derivations}\label{sec:430}

With the atoms of computation in place, it is possible to implement the most important definitions---namely, what representations syntax constructs and how these are computed.

\begin{definition}\label{def:SO}
    X is a \textit{syntactic object} SO iff:%
    \footnote{`If and only if', i.e. logical equivalence ($\leftrightarrow$).}%
    \begin{enumerate}[(i)]
        \item\label{def:SO:i}
            X is a lexical item, or
        \item\label{def:SO:ii}
            X is a set of SOs formed by applicated of \Merge.
    \end{enumerate}
\end{definition}
\noindent
SOs are thus defined recursively.%
\footnote{`Recursion' is used in this thesis in the strictly mathematical sense, as extensively discussed by \textcite{WatumullJ.etal_2014a}.}
Condition \ref{def:SO:ii} is much stronger than the definition in \CS. I believe it is justified following the preceding discussion in \autoref{sec:422}: by opening up the definition of SO to be simply any set that contains other SOs, we stray into the territory of notational games---ontologically speaking, a set is only an SO if it is formed by \Merge, to be defined in \autoref{def:merge}.

A couple of useful relations can be taken straight from \CS, primarily to simplify some definitions to come.

\begin{definition}
    For SOs $A$ and $B$, $B$ \textit{immediately contains} $A$ iff $A \in B$.
\end{definition}

\begin{definition}
    For SOs $A$ and $B$, $B$ \textit{contains} $A$ iff
    \begin{enumerate}[(i)]
        \item $B$ immediately contains $A$, or
        \item for some SO $C$, $B$ immediately contains $C$ and $C$ contains $A$.
    \end{enumerate}
\end{definition}
\noindent
I add to these a further definition for ease of exposition, effectively to represent the inverse of containment (cf. \nptextcite{ChomskyN_2019a,EpsteinSD.etal_2020}).

\begin{definition}
    For SOs $A$ and $B$, $B$ is \textit{a term of} $A$ if $A$ contains $B$.
\end{definition}
\noindent
\CS's definition of a `lexical array' is not required following the elimination of \LIk s, in line with \textcite{ChomskyN.etal_2019}, cf. also \CS\ (f.n.~4). How this plays with cyclicity and phases will be discussed in \autoref{sec:450}. Consequently, a `stage' of a derivation is equivalent to a workspace: ``WS [the workspace] represents the stage of a derivation at any point'' \parencite[245]{ChomskyN.etal_2019}.

\begin{definition}
    A \textit{workspace} $W$ is either a set of SOs or $\emptyset$.
\end{definition}
\noindent
As per \CS[47], ``[a] workspace includes all the syntactic objects that have been built up at a particular stage in the derivation''. Note that, using \autoref{def:merge}, a workspace is not actually an SO, as in \textcite[37]{ChomskyN_2020a} but unlike in \CS.

\begin{definition}\label{def:root}
    For any SO $X$ and workspace $W$, if $X \in W$, $X$ is a \textit{root} in $W$.
\end{definition}
\noindent
Careful not to confuse this with the very distinct notion of L-root in \autoref{def:lexroot}. There may be multiple roots in a workspace. This definition will prove useful in the definition of Merge and derivations below.

Next, it is time to derive the operation \Merge\ itself. It is possible, as done by \CS, to define Merge much as done in \autoref{sec:140} and in \pxref{ex:twomerges:nolabel}---namely, to say that Merge takes two inputs and returns a set as output. Formally, however, either Merge must make reference to the workspace, or there must be some additional stipulation outside of Merge as to the nature of a legitimate derivation. The latter option is taken by \CS, but this entails `derivation' to have properties that extend beyond UG, but which are not given a clear third factor justification. Instead, I adopt here the ternary definition from \textcite{ChomskyN_2021}, which is provided more formal structure by \textcite{SeelyTD_2021}, and which is adopted here. This definition further requires a notion of \textit{accessibility}, which determines which SOs are available to be Merged.

\begin{definition}\label{def:access:1}
    An SO $X$ is \textit{W-accessible} within a workspace $W$ iff $W$ contains $X$.
\end{definition}
\noindent
Note that SOs in \LEX\ are always accessible in some broader sense, since they can be externally Merged. This should, however, come at a cost---EM is penalised over IM because of the increased search space (and hence computationally by MinSearch). Hence, the more restricted form of accessibility, W-accessibility, is defined. This notion is to be revised below, in \autoref{def:access:2} and \autoref{def:access:3}.

Chomsky claims that EM is more computationally complex than IM as a consequence of requiring `massive search'. This is accurate only by hypothesis, however. \textit{A priori}, with no assumptions made about the structure of the lexicon, it is not possible to make any claims as to how lexical search operates. There is no reason to believe that a search algorithm similar to one used to trawl syntactic structures would be in place in the lexicon. Indeed, one could quite easily conceive of a data structure for the lexicon that requires no search at all, and that has a constant-time access algorithm. For example, imagine there is a deterministic function $\chi(x)$ that returns a unique output for each input. Each unique output corresponds to a possible location in memory where a lexical item can be stored. Assuming that the procedure of applying Merge has some kind of key $k$ ready for which the lexicon will be `searched', the procedure can merely run $\chi(k)$, which returns the needed location in memory, such that the full LI can be accessed.%
\footnote{This data structure is known as a \textit{hash map} or \textit{hash table} in computer science. It seems unrealistic that the lexicon actually works like this, but the actual implementational details are irrelevant at this algorithmic level of analysis (see \autoref{sec:110}).}

Nevertheless, we want to capture the idea that `NS-internal' computation is in some way more optimal than computation which accesses the lexicon. Hence, I introduce the operation \Select, which must operate in order to introduce new items into the workspace from \LEX. This definition is borrowed from \CS\ (f.n.~4), adapted to drop the notion of lexical array.

\begin{definition}
    For an SO $X\in\LEX$ and workspace $W$, $\Select(X, W)=\{X\} \cup W$.
\end{definition}
\noindent
Next, the primary way of manipulating the workspace, and the most fundamental operation in syntax: \Merge.

\begin{definition}\label{def:merge}
    $\Merge(P, Q, W)=\{\{P, Q\}, X_1, ..., X_n\}=W'$, such that
    \begin{enumerate}[(i)]
        \item $P$ and $Q$ are W-accessible within workspace $W$,
        \item $P \neq Q$, and
        \item for all SOs Y, $(Y \in W \wedge Y \notin \{P, Q\}) \rightarrow Y \in \{X_1, ... X_n\}$.
    \end{enumerate}
\end{definition}
\noindent
In sum, the \textit{domain} of \MERGE\ is all the W-accessible SOs in a workspace; the \textit{codomain} of \MERGE\ is the (discretely infinite) set of all SOs (cf. \nptextcite{WatumullJ_2015}). The final condition on Merge is required according to \textcite{ChomskyN_2021} in accordance with the SMT. It sustains MY (see \autoref{sec:145}). Importantly, there is no condition on the nature of the input SOs $P$ and $Q$ other than that they are W-accessible, which is substantively necessary (by Inclusiveness), and that they are distinct, ruling out self-Merge (\textit{pace} \nptextcite{AdgerD_2013}, see \CS, p.~48). IM and EM are thus totally equivalent at time of Merge, and thus cannot be distinguished even on the phase-level, assuming stricly Markovian derivations (see \autoref{sec:145} and also \autoref{def:WspaceAccess} below). The only difference is that EM requires application of \Select\ at the previous stage of a dervation.

Next, derivations themselves may be defined, making use of this new definition of Merge.

\begin{definition}\label{def:derivation:1}
    A \textit{derivation} within $L$ is a finite sequence of workspaces $\langle W_1, ..., W_n \rangle$, for $n \geq 1$, such that:
    \begin{enumerate}[(i)]
        \item $W_1 = \emptyset$,
        \item For all $i$, such that $1 \leq i < n$, and for some (accessible, distinct) SOs $A$, $B$,:
        \begin{enumerate}[(a)]
            \item (\textit{derivation-by-select}) $W_{i+1}=Select(A, W_i)$, or
            \item (\textit{derivation-by-merge}) $A$ or $B$ is a root and $W_{i+1}=Merge(A, B, W_i)$.
        \end{enumerate}
    \end{enumerate}
\end{definition}
\noindent
The root condition entails that Merge is always `at the root', which is necessary to derive MY. From this emerges a natural definition of \textit{workspace accessibility}.

\begin{definition}\label{def:WspaceAccess}
    A workspace $W$ is \textit{accessible} at stage $i$ of a derivation $\langle W_1, ..., W_n \rangle$ iff $W = W_i$.
\end{definition}
\noindent
This captures the strict Markovian property of derivations. Indeed, workspace accessibility constitutes a generalisation of the Goldfish Property introduced in \pxref{def:gp}. Whilst in the context of its introduction, the property applied only to LA, being a principle of computational optimality it should hold for every operation within $L$. The property can thus be generalised into a principle, and ideally would emerge as a theorem.

Finally, the culmination of a derivation is a single SO.

\begin{definition}
    A syntactic object $X$ is \textit{derivable} within $L$ iff there is a derivation $\langle W_1, ..., W_n \rangle$ where $W_n = \{X\}$.
\end{definition}



\subsection{Phases and cyclic transfer}\label{sec:440}

A conclusion from \autoref{sec:200} and \autoref{sec:300} is that there should optimally be only one computational cycle involved in generating syntactic structures. Operations that are countercyclic should be avoided, as well as assumptions that there can be multiple cycles operating in series---this latter point was essential to GB but abandoned in Minimalism. As a consequence, one must beware of introducing cycles underhandly, masquerading as other operations or as being `at the phase-level'.

On my view, Transfer is a potential source of accidental multiple cyclicity. One questions that could arise with respect to this is, what kinds of objects does Transfer `send' to the interface? But even this may be a misnomer, as the scare quotes illustrate: as emphasised by \textcite{ChomskyN_2021} and noted above in \autoref{sec:220}, derivational access can in theory be at any point. If this is the case, the interface should be able to see the entire generated structure at once. Also, along similar lines, it is the entire workspace that should be `transferred', viz. viewed. Otherwise, if interfaces were arbitrarily able to select parts of the workspace to view, there would be massive overgeneration (i.e. beyond what may be covered as deviance, cf. \nptextcite{ChomskyN_2019a,ChomskyN_2021}).

A problem with Transfer that is relevant here is what \CS[67] dub the \textit{Assembly Problem}, labelling an issue that arose in the original presentation of multiple spellout \parencite{UriagerekaJ_1999}. The problem boils down to a tension between the requirement to derive subjacency effects, namely phase impenetrability, and the need to retain memory of where, in the syntactic structure that is being derived, the transferred element was. An additional, related problem is that the internal structure of objects that have been `transferred' may need to be visible to later operations---\CS[72--73] describe how this is the case with the phenomenon of remnant movement. \CS\ formalise what they call the \textit{plug-back-in} model, but this option both erases the internal structure of what is transferred and requires the definition of SO to be expanded. This would also prevent Agree from crossing phase boundaries, which is argued for by \textcite{BoskovicZ_2007a}. Further, this clearly violates the NTC, contradicting computational optimality. \CS[73--74] sketch an alternative which focuses on the crucial notion, namely \textit{accessibility}. To capture the main effect of phase impenetrability, \Merge\ should not be able to apply to transferred elements. We have already established the importance of accessibility with the notion of W-accessibility in \autoref{def:access:1}, so this will clearly need to be modified in order to deal with cyclic transfer. \CS[73--74] suggest that the computational system should ``keep a set of syntactic objects that have been transferred, and then block all access to those transferred elements''---in determining accessibility, the set of transferred elements needs to be checked.

\begin{definition}
    A set of transferred elements $T$ is either a set of SOs or $\emptyset$.
\end{definition}

\begin{definition}\label{def:stage}
    A stage of a derivation $S_i = \langle W_i, T_i \rangle$, where $W_i$ is a workspace and $T_i$ is the set of transferred elements.
\end{definition}
\noindent
The introduction of more complex derivational stages will require some adjustments to the definitions given in \autoref{sec:430}. First, let's redefine W-accessibility.

\begin{definition}\label{def:access:2}
    An SO $X$ is \textit{W-accessible} at stage $\stage{i}$ iff $W$ contains $X$ and $T$ does \textit{not} contain $X$.
\end{definition}
\noindent
Transfer can now be defined simply, as an operation ranging over sets of transferred elements.

\begin{definition}\label{def:transfer}
    $Transfer(X, T) = T \cup \{X\}$, for set of transferred elements T and $X$ an SO.
\end{definition}

Transfer should be no more complex than this. However, there appears to be a conflict here with how phases operate---all operations should be on the phase-level, including Merge, Agree, Label and Transfer. This cannot literally be true, since this entails lookahead, as noted by [EKS] and accepted by [Chomsky]: operations must occur before any phase head has been Merged. When coupled with the greater interface proximity afforded by the Minimalist architecture, I believe this entails a return to a stricter, bottom-up cyclicity. Further, there is no need for any algorithm or operation to occur outside of this cycle, as all operations are interface-driven.

Phases exist to further restrict accessibility, creating subjacency effects \parencite{ChomskyN_1973}. In the Minimalist literature, this comes in the form of the PIC (see \autoref{sec:145}). Phases also define valid Transfer domains---transfer occurring at any other point does not construct a valid derivation. Note that the principle that operations take place freely and that access to the derivation can be at any point \parencite{ChomskyN_2021} entails that Transfer, like Merge, is not `triggered' by, say, the Merging of a phase head. It may be possible to extend this formalisation to capture a broader interpretation of the idea that all operations take place `on the phase level', which is more in line with typical Minimalist assumptions \parencite[cf.][]{AdgerD.RobertsI_}. There is not space here to discuss developments along these lines.

With \Transfer\ being defined as in \autoref{def:transfer}, it is necessary to redefine derivation, originally \autoref{def:derivation:1}, incorporating the more complex stages of \autoref{def:stage} and introducing derivation-by-transfer.

\begin{definition}\label{def:derivation:2}
    A \textit{derivation} within $L$ is a finite sequence of stages $\langle \stage{1}, ..., \stage{n} \rangle$, for $n \geq 1$, such that:
    \begin{enumerate}[(i)]
        \item $W_1 = T_1 = \emptyset$,
        \item For all $i$, such that $1 \leq i < n$, and for some (accessible, distinct) SOs $A$, $B$,:
        \begin{enumerate}[(a)]

            \item (\textit{derivation-by-select}) $T_{i+1} = T_i$ and $W_{i+1} = Select(A, W_i)$, or

            \item (\textit{derivation-by-merge}) $T_{i+1} = T_i$ and $A$ or $B$ is a root and $W_{i+1} = Merge(A, B, W_i)$.

            \item (\textit{derivation-by-transfer}) $T_{i+1} = Transfer(X, T_i)$ for $X$ a W-accessible SO contained in $W_i$ and $Y$ a root, such that $Label(X)$ and $Label(Y)$ are phasal and $Y$ contains $X$.

        \end{enumerate}
    \end{enumerate}
\end{definition}

Note that derivation-by-transfer requires \Label, although I reserve definition of this until \autoref{sec:470}, after the definitions of MS and of features. I have also left the notion of \textit{phasal} undefined---indeed, I leave this entirely open-ended, as there has not been space to review approaches to phases in enough detail to establish a way forward. Where necessary, I will adopt the standard approach discussed in \autoref{sec:145}: a label $\alpha$ is \textit{phasal} if the categorial feature corresponding to $C$ or $v^*$ is a member of $\alpha$.%
\footnote{Features are defined in \autoref{sec:460}, labels in \autoref{sec:470}.}



\subsection{Minimal search}\label{sec:450}

It was established in \autoref{sec:130} and \autoref{sec:140} that MS is of central importance to the operation of \CHL, in particular in the operations of agreement and labelling. Minimal Search is a subcomponent of computational optimality, as represented in \pxref{ex:mobbscompopt}. Optimal search algorithms have thus been an object of study within computer science effectively since its inception. Deciding on the algorithm or class of algorithms which is employed by I-language is an empirical matter. As discussed by \textcite{KeH_2019,KeH_2021}, it may be the case that different subcomponents of I-language employ different search algorithms. There is no \textit{a priori} reason that this should not be the case, even adhering to TLTB. Different search algorithms may be more appropriate and thus more optimal for different kinds of data. Nevertheless, \textcite{KeH_2019} proposes that labelling and agree can indeed be unified under MS, following the conjecture of \textcite{ChomskyN_2013,ChomskyN_2015}. I maintain this proposal here, formalising a unified MS algorithm, hypothesised to form the basis of \Label\ and \Agree, to be defined in \autoref{sec:470} and \autoref{sec:480}, respectively.

\subsubsection{Accessibility and trace-invisibility}

Before discussing the algorithm itself, it must be established what elements are actually available to serve as labels. The notion of W-accessibility was previously defined in \autoref{def:access:1} and \autoref{def:access:2} to account for this, however, it will need to be revised a final time.

It was established in \autoref{sec:300} that lower copies of moved elements are invisible to the labelling algorithm. This allows labelling to derive many cases of movement, as per \textcite{ChomskyN_2013} and subsequent work on labelling, in particular within GDA \parencite{MoroA.RobertsI_2020}. Optimally, if lower copies are invisible to labelling, they should be invisible to all operations.

\begin{definition}\label{def:access:3}
    An SO $X$ is \textit{W-accessible} at stage $\stage{i}$ iff
    \begin{enumerate}[(i)]
        \item\label{def:access:3:i}
            $W$ contains $X$,
        \item\label{def:access:3:ii}
            $T$ does \textit{not} contain $X$, and
        \item\label{def:access:3:iii}
            $X$ is at position $P_1$ and there is no occurrence of $X$ at a position $P_2$ such that $X_{P_1}$ is a term of the sister of $X_{P_2}$.
    \end{enumerate}
\end{definition}
\noindent
Condition \ref{def:access:3:iii} is necessarily somewhat stipulative. Ideally, it would be possible to derive trace-invisibility from something more fundamental; I leave this for future work. In this regard, note that \ref{def:access:3:iii} does conceal the c-command relation (cf. \autoref{fn:c-command}). This undoubtedly carries some significance in relation to the operation of MS.%
\footnote{Note that the output of \FormCopy\ could be used to find the occurrences of $X$. Since I have not formalised \FormCopy, I leave this possibility to future analysis, cf. \autoref{sec:420}.}

\subsubsection{DFS or BFS?}

The `minimal' in MS comes from the ``previously implicitly assumed but unnoted'' property that ``Minimal Search terminates whenever a target is found'' \parencite[3]{KeH_2021}. In other words, there can only be one candidate target found by MS, so no comparisons between candidate targets are necessary. The question is, then, exactly how this target is reached. Both \textcite{KeH_2019} and \textcite{MilwayD_2021} note the observation from computer science that there are two broad classes of search algorithm: \textit{depth-first} search (DFS) and \textit{breadth-first} search (BFS). DFS prioritises travelling `down' in a tree, corresponding to travelling into more deeply embedded sets, as diagrammed in \pxref{ex:dfs}. BFS explores all nodes at one tier before progressing to the next tier, as diagrammed in \pxref{ex:bfs}. The numbers represent the order in which nodes are traversed.

\begin{example}\label{ex:dfs}
    \begin{forest}
        [1 [2 [3] [4]] [5 [6] [7]]]
    \end{forest}
\end{example}

\begin{example}\label{ex:bfs}
    \begin{forest}
        [1 [2 [4] [5]] [3 [6] [7]]]
    \end{forest}
\end{example}
\noindent
Applying these algorithms to structures produced by \Merge\ results in a number of issues. Most significantly, \Merge\ produces sets, with no linear order, not trees, as diagrammed above, which are encoded with linear order. As illustrated, both DFS and BFS make use of linear order to determine the search order. As a result, both would be catastrophic if used in the context of labelling. For example, take the trees above: assume that LA is tasked with labelling the node indicated with 1. Assume further that the lowest tier consists solely of heads. In \pxref{ex:dfs}, 3 will serve as the label, in \pxref{ex:bfs}, 4 will serve as the label. (Note that, in this case, both algorithms reach the same node, which may not be the case in a more complex example.) Since the algorithm reaches these nodes first, and since they are heads and thus cannot be searched into, they are immediately returned by the algorithm. In the context of SOs, as opposed to trees, this entails making an arbitrary decision as to which node to choose first, which is clearly empirically unjustified. Further, DFS specifically presents issues. As \textcite{KeH_2019} notes, it does not respect c-command relations, unlike BFS, which has ingrained a notion of superiority, since it prioritises exploring nodes on the same tier. Instead, DFS primarily makes use of the containment relation. Additionally, DFS simply derives the wrong results: DFS would entail travelling potentially many levels deep into a structure, when if it just looked at the initial sister it would find a head immediately. DFS just does not capture the kinds of relations found in language, and should be discarded. Despite this, \textcite[17]{MilwayD_2021} argues that DFS ``retains a certain theoretical and aesthetic appeal'' and thus should remain under consideration. He notes that some authors, namely \textcite{BrananK.ErlewineMY_,PremingerO_2019}, argue for a DFS-based MS algorithm. However, these proposals require that Merge be defined asymmetrically, making this implausible in a system that uses set-Merge as in \autoref{def:merge}, eliminating linear order in line with the SMT.%
\footnote{\textcite[17]{MilwayD_2021} claims that \citeposs{KeH_2019} algorithm is ``parallelized DFS'', although this directly contradicts \citeposs[48]{KeH_2019} own assertion that ``[t]he search algorithm in the definition of minimal search [] is breadth-first''.}

\textcite{MilwayD_2021} overcomes the linear order issue by appealing to what he terms `Minimal Tiered BFS' \parencite[15]{MilwayD_2021}. In such a system, all SOs on a particular tier of the BFS algorithm are considered part of the same set, and are accessed simultaneously in order to identify the target. In the course of the algorithm, structure is thus ignored. I adopt Tiered BFS here.

\subsubsection{Domain and target}\label{sec:452}

Before presenting the formalisation of MS in \autoref{sec:453}, the parameters to the algorithm need to be established. As pointed out by \textcite{KeH_2019}, the search algorithm (SA) is only one aspect of MS. In order to operate, SA needs two further elements a \textit{search domain} (SD) and a \textit{search target} (ST). In order to unify MS, SA and SD may be provided as parameters. This enables SA to be highly flexible---which is a highly desirable outcome, since SA is presumed to be a third factor. Indeed, the parameterisation of SA quite neatly demonstrates the interaction between factors discussed in \autoref{sec:110}, with the first factor specifying ST. Indeed, I would suggest that the second factor is also incorporated into ST, since it presumably relies on lexical specification to determine whether the target has been found, and the lexicon is the source of syntactic variation (see \autoref{sec:150}).

There must necessarily be constraints on the parameters for SA. \textcite[44]{KeH_2019} states that SD consists of sets and ST features. In the present formalisation, SD shall be sets of SOs in particular. ST, however, I assume to be more complex. The primary reason for this is that ST being a specific feature or set of features seems appropriate for Agree, but not for Label. In the latter case, any set of features can be considered a label---what is important is the structure more generally. If SD for Label is effectively `anything goes' (as long as its something that bears features, i.e. a non-root head), then a problem arises. Namely, feature sharing arangements, discussed in \autoref{sec:300} as being crucial to modern labelling theory, are impossible under such a theory. Indeed, this is noted by \textcite{KeH_2019}, who proposes that, in scenarios where two heads are found simultaneously, it is the pair of heads that serves as the label, rather than their intersecting features. This is effectively justified by appealing to \textcite{TakitaK_2020}, who argues that labelling is required only by SM, not C-I. This is a radical departure from the assumptions of the preceding discussion, and so ought to be avoided. A second, more general reason to suggest a more complex ST is that it allows SA to be much more flexible. As SA is presumed to be a domain-general third factor, this is a desirable outcome.

The question is then of how to formalise a generalised ST. The solution I adopt is to allow ST to be a function definition, which takes an SD as its domain. In the process of search, SA applies the function at each tier. The codomain of the ST function is then $\emptyset$ plus the range of possible matched items, whether this be individual (sets of) features (as in the case of Agree and feature-sharing label) or entire heads (as may be also the case for Label). ST also determines the output of the search procedure itself, as SA returns whatever is matched, which is the output of ST. This enables what \textcite[23]{ShimJY_2018} terms ``comparison search'', but without necessarily imposing a greater computational burden as he argues.

\subsubsection{Defining minimal search}\label{sec:453}

It is now possible to formally define our SA.

\begin{definition}\label{def:MS}
    For SD $\delta$, a set of SOs, and ST $\tau$, a unary function:
    \begin{enumerate}[(i)]
        \item\label{def:MS:i}
            If $\delta = \emptyset$, $\Sigma(\delta,\tau) = \emptyset$,
        \item\label{def:MS:ii}
            Else if $\tau(X) \neq \emptyset$, $X \in \delta$, then $\Sigma(\delta,\tau)=\tau(X)$,
        \item\label{def:MS:iii} 
            Else, $\Sigma(\delta,\tau) = \Sigma(\bigcup\{X \in \delta : X$ is W-accessible and $X \notin \LEX\},\tau)$.
    \end{enumerate}
\end{definition}
\noindent
Conditions \ref{def:MS:ii} and \ref{def:MS:iii} represent the crucial recursive workings of the operation. This part of the algorithm first checks if the current SD contains a matching element, and if it does, it returns whatever the matching function itself returns. Then comes the recursive step: if no matching element is found, perform the algorithm again using the arbitrary union of all SOs immediately contained by SD that are not lexical items. Condition \ref{def:MS:i} is a fallback that allows search to fail entirely. This foreseeably results in the derivation crashing at the interfaces in most cases, although this is not \textit{a priori} necessary. For instance, adjuncts might be unlabelled \parencite[see][]{BlumelA_2017a}.

As planned, this algorithm combines the tripartite architecture formalised by \textcite{KeH_2019} with \citeposs{MilwayD_2021} formal framework, inherited from \CS. In particular, the arbitrary union definition of tiers is taken from \textcite[16]{MilwayD_2021} as is the recursive operation of the algorithm. \autoref{def:MS} also incorporates the novel flexibility of \ST. A further benefit of this latter point is that the output of \MS\ does not have to be arbitrarily defined for each instantiation of the function. \textcite{KeH_2019} has to do this in his formalisation, because (a) he does not paramaterise \MS\ (but rather defines it as a tuple), and (b) he does not allow ST to be a function. In my case, a single call to \MS\ contains all the information needed to determine its precise operation.



\subsection{Features}\label{sec:460}

A full formalisation of features and agreement is beyond the scope of this thesis. However, it is necessary to have some conception of features, as these are what serve as labels, treated in \autoref{sec:470}. The nature of features is also inimicly tied to the nature of agreement, discussed in \autoref{sec:480}. A relatively neutral approach to the nature of features will be taken based on the system formalised by \textcite{AdgerD_2006,AdgerD_2010}, in order to match up best with the labelling theories that have been discussed and the approach that will be adopted in \autoref{sec:470}.

\textcite{AdgerD_2006,AdgerD_2010} proposes a formal hierarchy of feature systems, recapitulated in \pxref{ex:FH}.

\begin{subexamples}[preamble={\textbf{\textit{Feature system hierarchy}}}]\label{ex:FH}
    \item \textbf{Privative}: ``atomic features may be present or absent, but have no other properties'' \parencite[187]{AdgerD_2010}.
    \item \textbf{Privative with interpretability}: privative features may be prefixed with $u$ to indicate uninterpretability, enabling \textit{checking} relations to be formed between features \parencite[cf.][]{ChomskyN_1995}.
    \item \textbf{Binary attribute-value}: features are ordered pairs $\langle Att,Val \rangle$ where $Att$ is drawn from a set of attributes, and $Val$ is $+$, $-$ or empty.
    \item\label{ex:FH:iv} \textbf{Multi-valent attribute-value}: features are attribute-value pairs, values are drawn from a larger set of values.
    \item \textbf{Recursive attribute-value}: features are attribute-value pairs, values may themselves be features.
\end{subexamples}
\noindent
For reasons there is no room to discuss, \textcite{AdgerD_2006,AdgerD_2010} decisively argues that a multi-valent attribute-value feature system \pxref{ex:FH:iv} is the option most compatible with Minimalism as generally practised. A version of this system is adopted by \textcite{MilwayD_2021}, who makes the simplifying assumption that values can be encoded as integers. This eliminates the need for much of \citeposs{AdgerD_2010} formal apparatus. Whilst it is safe to say that this apparatus would be required in a fully-fledged theory of features, especially with respect to interface interpretation, I will also adopt a simplifying assumption. Instead of using integers to symbolise feature values, I will use an equivalent set of atomic symbols, more similarly to \textcite{AdgerD_2006,AdgerD_2010}. I will also use $\emptyset$ as a notational convention to indicate the lack of value (as done by \nptextcite{AdgerD_2010}), as a clearer alternative to blank space.

Another simplifying assumption is the adoption of the feature sets defined in \autoref{def:UG}. Within this formalisation, I will assume that all possible features, i.e. all possible attribute-value pairs, are defined in UG. This greatly limits the power of the system, but is a necessary simplifying assumption to avoid too much digression. Naturally, the formalisation present here can easily be extended to accomodate a more fully-fledged feature theory of an equivalent level of complexity to the attribute-value system formalised by \textcite{AdgerD_2006,AdgerD_2010}. This being said, \textcite{AdgerD_2010} makes some assumptions that I intentionally do not assume here, since he adopts a feature-driven system, akin to `Triggered-Merge' in \CS, which is in opposition to the (free) Merge-driven system I have adopted here (cf. \autoref{sec:140}). Thus, the feature system formalised here is necessarily excessively simple, which will have bearing on potential empirical consequences that there will unfortunately not be room to discuss.

With this in place, I adopt the following definitions.

\begin{definition}
    $Att$ is the set of feature \textit{attributes}.
\end{definition}

\begin{definition}
    $Val$ is the set of feature \textit{values} $\{+, -, ...\}$.
\end{definition}

\begin{definition}
    A \textit{feature} $f$ is a pair $\langle a, v \rangle$ where $a \in Att$ and $v \in Val$. $v$ may be empty; if this is the case $f$ is considered \textit{unvalued}.
\end{definition}

\begin{definition}
    A feature $f$ is \textit{interpretable} iff $f \in \Fsem$. A feature is \textit{uninterpretable} iff $f \in \Fsyn$.
\end{definition}
\noindent
Note that this definition allows dissociation between feature value and feature interpretability, which is employed by some feature theories, notably by \textcite{PesetskyD.TorregoE_2007}.

It will also be useful to define a simple function \Match, which determines if features are matching for attribute, but not value.

\begin{definition}\label{def:match}
    For two features $f_1 = \feature{1}$, $f_2 = \feature{2}$,
    \begin{enumerate}[(i)]
        \item if $f_1 = f_2$, $\Match(f_1,f_2) = \emptyset$,
        \item else if $Att_1 = Att_2$ and one of $Val_1, Val_2$ is empty, $\Match(f_1,f_2) = \{f_1, f_2\}$ or $\pair{f_2}{f_1}$, such that the valued feature is first.%
            \footnote{This condition is totally arbitrary, and is adopted solely to maintain the assumption that shared labels are pairs as opposed to sets.}
        \item else, $\Match(f_1,f_2) = \emptyset$.
    \end{enumerate}
\end{definition}
\noindent
This definition can be developed into a function \MatchLI, which returns the set of all sets/pairs of features that satisfy \Match\ between two LIs.

\begin{definition}\label{def:matchLI}
    For two LIs $X$ and $Y$, let the set $S = X \times Y$. For all $s \in S$, $\MatchLI(X, Y) = \{\Match(s) : \Match(s) \neq \emptyset\}$.
\end{definition}


\subsection[Defining \Label]{Defining $\mathbfit{\Label}$}\label{sec:470}

Finally, the machinery is in place to enable us to define \Label. Conceivably, with \autoref{def:MS} in place, all that is required is to define the SD \SD\ and the ST \ST. Establishing \SD\ appears simple: it is the SO to be labelled itself. \ST\ is more tricky. As discussed in \autoref{sec:450}, feature-sharing complicates the picture.

Following the discussion in \autoref{sec:320}, however, entails that we must also consider how exactly \Label\ is applied within derivations. Namely, there are two broad classes of possible labelling processes, as discussed in \autoref{sec:320}: labelling can be bottom-up, or top-down. In \autoref{sec:320}, I outline an informal definition of bottom-up Label using GP. Since labelling is usually considered top-down, and since bottom-up operation is ruled out by \textcite{KeH_2019,AdgerD.RobertsI_}, I elect to formalise only top-down \Label, \LabelTD, in this section. In \autoref{sec:472}, I note how one could go about formalising bottom-up \Label, \LabelBU.

The following helper function will prove useful for determining the lexical items within a set of SOs.

\begin{definition}
    For a set of SOs $\SD$, $Heads(\SD) = \{X \in \SD : X \in \LEX$ and for $X$, $\SYN \neq \emptyset\}$.
\end{definition}

The condition that a head must have syntactic features excludes L-roots from providing labels, and generally participating in syntactic relations. Ideally, this would be derived from more fundamental principles; this is reserved for future work.

\subsubsection{Top-down labelling}\label{sec:471}

\begin{definition}\label{def:TD:ST}
    For set of SOs \SD,
    \begin{enumerate}[(i)]

        \item\label{def:TD:ST:i}
            if $\SD = \emptyset$, then $\ST_{TD}(\SD) = \emptyset$,

        \item\label{def:TD:ST:ii}
            else if $Heads(\SD) = \{X\}$, for $X$ an SO, then $\ST_{TD}(\SD) = \{X\}$,

        \item\label{def:TD:ST:iii}
            else if for any set of LIs $S = \{X,Y\} \in [Heads(\SD)]^2$ such that $\MatchLI(S) \neq \emptyset$, $\ST_{TD}(\SD) = \MatchLI(S)$,

        \item\label{def:TD:ST:iv}
            else $\ST_{TD}(\SD) = \emptyset$.

    \end{enumerate}
\end{definition}

Condition \ref{def:TD:ST:ii} is the simple case where, at a particular tier, if there is only one head, this is chosen as the label. Condition \ref{def:TD:ST:iii} enables feature-sharing where there are multiple heads that match features. Note that if there are multiple heads but they do not match features, the function will return $\emptyset$. Hence, encoded in this definition is the proposal, implicit in \textcite{ChomskyN_2013}, that the feature pair in a feature-sharing arrangement is a valued-unvalued combination.

Recursive MS can thus be defined simply.

\begin{definition}\label{def:TD:label}
    For an SO $X$, $\LabelTD(X) = \MS(X, \ST_{TD})$.
\end{definition}

Top-down labelling requires a complex \ST\ in order to account for feature sharing. Interestingly, it does not impose any restraint on its operation: since \ST\ searches only for heads, it is not dependent on any previous computation. It can thus apply totally freely, at the expense of greater time complexity, since search needs to find the head on every occasion, even when this might be deep within a nested symmetric structure.

Despite this general appeal, my formalisation of top-down labelling has the same fatal flaw as is implicit in \citeposs{KeH_2019} formalisation: it fails in $\{XP, YP\}$ structures where the heads of each phrase are differentially nested. For instance, if $X$ is less deeply embedded than $Y$, $X$ will be selected as the head. Since \textcite{NakashimaT_2021} claims that this differential level of embedding does actually play such a role in labelling (in the form of his `Symmetry Condition on Labelling'), I leave the option of this formalisation open. However, I believe that bottom-up labelling is actually more in line with what is generally assumed in the Minimalist literature, and I proceed to formalise this option in the next subsection, replacing \autoref{def:TD:ST} and \autoref{def:TD:label}.

Note also that it is possible to derive a crucial conclusion of \textcite{ChomskyN_2015} as a theorem, namely the fact that L-roots never provide labels.

\begin{theorem}\label{thm:rootinvis}
    For the SO $\alpha = \{X, R\}$, for any SO $X$ and an L-root $R$, $\Label(\alpha)=\Label(X)$.
\end{theorem}
\noindent
Another important case can be derived.

\begin{theorem}\label{thm:standardcase}
    For the SO $\alpha = \{X, YP\}$, for any $X \in \LEX$ such that $X$ is not an L-root, $\Label(\alpha)=\Label(X)$.
\end{theorem}

\subsubsection{Bottom-up labelling}\label{sec:472}

Bottom-up labelling behaves quite differently. In order to accommodate the fact that accessibility as defined in \autoref{def:access:3} is sensitive to higher structure, bottom-up labelling needs to retain two kinds of memories. One, as discussed in \autoref{sec:320}, is the `goldfish' memory of the labels assigned at the previous level. The other is a set of variables that are incrementally assigned as the labelling algorithm traverses up the structure, encountering symmetric $\{XP, YP\}$ structures that cannot be labelled by feature sharing. These must be assigned an indeterminate label $\alpha$, which can be resolved upon movement, since at this point the lower occurrence is no longer accessible by trace invisibility. This then perlocates through the label ledger.

Labels, on this view, are not constructed on the fly. Rather, they are assembled from the ground up. Generated labels can then be placed in a ledger---similarly to transferred structures, and similarly to \Agree\ (see \autoref{sec:480}). This would require a redefinition of derivation, which I do not offer here.

What this entails is that ST for \LabelBU\ will be substantively simpler, but SD will be more complex, as a result of needing access to labels applied at the previous stage of derivation. 

\subsubsection{Comparison}\label{sec:473}

In terms of optimality, there is a final comparison to be made between \LabelTD\ and \LabelBU.

\LabelTD\ allows the label of an element to be identified at any time. If the algorithm fails, a label cannot be assigned. This presumably triggers GDA movement. \LabelBU\ operates purely cyclically but requires a memory. This memory serves to reduce the overall amount of search, since only four elements ever need to be checked, namely the two labels and two SOs of the previous tier. This violates MinCCD for the sake of MinSearch, trading time complexity for space complexity. It is less clear how GDA works with \LabelBU. Without further specifying the algorithms, a more precise comparison is not available.


\subsection[\Agree]{$\mathbfit{\Agree}$}\label{sec:480}

The fundamental purpose of syntax is to eliminate uninterpretable features that are present in lexical items so that they can be interpreted at the interfaces (cf. \autoref{sec:143}). Relations between features within separate SOs encode syntactic dependencies \parencite[cf.][]{AdgerD_2010}. All such dependencies need to be evaluated within the syntax in order to be legible. In many cases, application of \Merge\ suffices to satisfy feature dependencies. For instance, take subcategorisation: as mentioned in \autoref{sec:200} subcategorisation was one of the motivations for the development of labelling theory because of the periscope property, whereby the selecting head need only look at the label of the selected structure. The original formulation of a separate, non-Merge relation by \textcite{ChomskyN_2000,ChomskyN_2001} was motivated by two constructions in particular: English existential constructions and long-distance dependencies in many languages.

\textcite{MilwayD_2021} formalises Agree within the \CS\ framework. This could reasonably trivially be adapted into the present framework. However, I would suggest a further revision: namely, that Agree operates on the labels provided by \Label\ (either top-down or bottom up) rather than raw SOs. Since Agree is not the focus of this thesis, a full formalisation will not be provided. It seems reasonable to assume that there is a definable $\Agree(N, R)=R'$ function, which maps a set of labels $N$ and a ledger $R$ to a new ledger $R'$.



\subsection{Key conclusions}\label{sec:490}

This subsection summarises some of the issues that arise from the formalisation as present.

\subsubsection{Computational and substantive optimality}\label{sec:491}

There are a number of properties of this theory that are in line with MaxTP and TLTB. Most importantly, this theory maintains MY as a theorem.

\begin{theorem}[MY]\label{thm:MY}
    For consecutive stages $S_i$ and $S_{i+1}$, $S_{i+1}$ may only have up to one more W-accessible object than $S_i$.
\end{theorem}

The NTC and the Extension Condition can also both be derived as theorems as done by \CS[58-59]. Note, however, that both \CS\ and \textcite{MilwayD_2021} ultimately propose theories that violate NTC. They respectively claim that Transfer and Agree require lower structure to be substituted. In my formalisation, I adopt ledgers instead, again trading time complexity for space complexity.

\subsubsection[\Agree\ and \Label]{$\mathbfit{\Agree}$ and $\mathbfit{\Label}$}\label{sec:492}

Crucially, \Agree\ and \Label\ conspire to form syntactic dependencies throughout constructed SOs. In cases where a label is a valued-unvalued pair $\pair{\feature{i}}{\featureUV{i}}$, \Agree\ would not need to operate. This clearly identifies the precise nature of the redundancy between \Agree\ and \Label. Namely, it is not the fact that both operations utilise MS that creates the redundancy, since as shown by the formalisation in \autoref{sec:450}, they each use different instantiations of SA. Instead, the redundancy is in their outcomes: they both result in creating relations between features.

\subsubsection[\texorpdfstring{$Label \in UG$}{Label in UG}?]{\texorpdfstring{$\mathbfit{Label \in UG}$}{Label in UG}?}\label{sec:493}

Labelling theory is not yet at the stage where \Label\ can be totally dissociated from UG. As shown in \autoref{sec:450}, whilst SA is potentially domain-general, its parameters, ST and SD, are narrowly syntactic. As a result, \Label\ is required to be included in \autoref{def:UG}. In line with OM and MM, future work could help determine whether this stipulation could be dropped.

