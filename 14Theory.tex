\subsection{Informal theoretical summary}\label{sec:140}

This subsection introduces the main theoretical concepts that have been accepted as standard within MP and that form the basis of the discussion to follow. The theory is ultimately founded upon empirical results that will not receive any discussion here. The goal is instead to summarise and synthesise Chomsky's seminal Minimalist papers \parencite{ChomskyN_1993,ChomskyN_1994,ChomskyN_1995,ChomskyN_2000,ChomskyN_2001,ChomskyN_2004,ChomskyN_2007,ChomskyN_2008,ChomskyN_2013,ChomskyN_2014,ChomskyN_2015,ChomskyN.etal_2019,ChomskyN_2019b,ChomskyN_2020,ChomskyN_2021}. Whilst these papers could be said to represent a coherent development of a Minimalist theory over time, they do not, of course, identify a uniform contribution of Chomsky's but rather a useful summary of how general concerns have evolved in MP. Closer scrutiny of labelling in particular is postponed until \autoref{sec:200}, where the development of labelling theory is reviewed, and \autoref{sec:300}, which discusses the innovations since \textcite{ChomskyN_2013,ChomskyN_2015}.

Such a summary is not always provided in work in the domain of theoretical syntax, especially in more analytic work, but it is warranted here on account of the precision that can lack in theoretical syntax, and which this work sets out to begin to rectify. This is not intended as criticism---rather as testament to the nascent nature of the field and the sheer scale of complexity of the phenomena under investigation. As a metacritical aside, it may well be amusing though not picayune to suggest that Chomsky's own body of work resembles Aristotle's in a way criticised by Galileo explicitly in the aforecited work.%
\footnote{For similar comments from a different perspective, see \textcite{AsudehA.ToivonenI_2006}. For a considerably more disparaging assessment of Chomsky's recent work, see \textcite{BehmeC_2014,BehmeC_2015}. A full response to these criticisms would be far too great a tangent.}
When presented with such a string of references to a single author as above, an oeuvre that supposedly represents a coherent thread of theory alongside the more programmatic suggestions, one might be reminded of a certain passage from the second day of the \textit{Dialogo}. Protesting Salviati's dismissal of Aristotelian doctrines, Simplicio holds that, in order to be qualified to do so, ``one must have a grasp of the whole scheme, and be able to combine this passage with that, collecting together one text here and another very distant from it''---indeed, taking it a step further, he subsequently claims that ``[t]here is no doubt that whoever has this skill will be able to draw from his books demonstrations of all that can be known; for every single thing is in them.'' \parencite[108]{Galileo_1967}. Sagredo wittily replies: 

\begin{quote}\setstretch{1.0}
``I have a little book, much briefer than Aristotle or Ovid, in which is contained the whole of science, and with very little study one may form from it the most complete ideas. It is the alphabet, and no doubt anyone who can properly join and order this or that vowel and these or those consonants with one another can dig out of it the truest answers to any question[].'' \parencite[109]{Galileo_1967}
\end{quote}
\noindent
Thus, Galileo offers an addendum to this fascination with the alphabet and by extension with language which Sagredo proclaims on the first day, and which is so often cited by Chomsky, as discussed in \autoref{sec:110}. A goal of this thesis is to eliminate the overreliance on appeal to authority criticised by Galileo, truly approaching I-language ``from [the] bottom up'' \parencite[4]{ChomskyN_2007}---again, not mathematical game-playing, but seeking genuine explanation. The review of Minimalist theory provided in this subsection and in \autoref{sec:200} and \autoref{sec:300} makes clear how entwining the various skeins of MP is by no means a trivial task, albeit certainly a worthwhile endeavour.

\subsubsection[\CHL]{$\mathbfit{\CHL}$}\label{sec:141}

An I-language L $([F],\ \Lex,\ \MERGE,\ \AGREE,\ \TRANSFER,\ \fSM,\ \fCI)$ is a state of the faculty of language FL, a component of the human mind/brain.%
\footnote{Operations will be denoted in this subsection using capital letters, following the convention introduced by \textcite{ChomskyN.etal_2019}. When discussing operations without any particular theory in mind, CamelCase normal text will be used---this is employed e.g. in the review sections, \autoref{sec:200} and \autoref{sec:300}. For the formalised operations in \autoref{sec:400}, I adopt a different convention (see \autoref{fn:FormalConventions}).}
The initial state \Szero\ of FL, call this Universal Grammar UG, determines the set of \textit{features} available for all languages \setF, from which L selects a subset $[F]$, and assembles this subset into a lexicon \Lex\ consisting of lexical items (LIs). For each derivation, L selects a lexical array \LA\ from \Lex. UG also determines the computational procedure for human language \CHL\ which generates narrow-syntactic expressions, syntactic objects (SOs), out of LIs. In the definition of L above, \CHL\ consists of the three operations \MERGE, \TRANSFER\ and \AGREE. LIs are the `atoms of computation' for \CHL. An LI is an SO, termed a \textit{head} or \textit{minimal projection} within the context of a larger SO constructed by \CHL. On the simplest assumptions, \CHL\ is uniform for all L.%
\footnote{As proposed by \textcite[107]{ChomskyN_2004}, contra \textcite[100]{ChomskyN_2000}, where it is suggested that ``parameter setting'' be ``refinement of \CHL\ in one of the possible ways''. The theory presented here assumes that variation is confined to the lexicon, as elaborated further in \autoref{sec:150}.}
The operation \MERGE, part of \CHL, recursively constructs objects out of \LA, each of which can be mapped to a semantic representation \SEM\ by the \textit{semantic component} \fCI\ and to a phonetic representation \PHON\ by the \textit{phonological component} \fSM. The operation \TRANSFER\ hands an object generated in the narrow syntax NS by \MERGE\ to \fCI\ and \fSM, resulting in the expression $\Exp=\phonsem$. L generates a set of expressions \setExp, interpreted at the interfaces. \PHON\ is interpreted by sensorimotor systems SM, whilst \SEM\ is interpreted by conceptual-interpretive systems C-I.

\subsubsection{The interfaces}\label{sec:142}

Together, SM and C-I constitute the \textit{interfaces}, crucial to the minimalist approach proposed by \textcite{ChomskyN_1993}: ``all conditions are interface conditions; and a linguistic expression is the optimal realization of such interface conditions'' \parencite[26]{ChomskyN_1993}. As such, the interfaces SM and C-I and their respective mappings \fSM\ and \fCI\ deserve further attention. \fCI\ is assumed to be uniform for all L, \fSM\ is assumed to vary greatly. Indeed, following the Minimalist proposal \pxref{ex:minprops:4}, \fSM\ is the locus of all variation (a point that I will return to in \autoref{sec:150}). As aforementioned, \CHL\ is uniform, thus linguistic variation is confined to $[F]$, \Lex, and \fSM. \fSM\ also has the special property that it may introduce features from $[F]$ into the computation of \PHON, violating Inclusiveness \pxref{def:inclusiveness}, which necessarily holds only for \CHL. Very little is understood about \fCI, which may introduce features not present in SO, but we assume not from $[F]$ \parencite[107]{ChomskyN_2004}. If this does hold, it is sensible to include \fCI\ in the definition NS, as is typical in the literature.

Following \textcite[241]{ChomskyN.etal_2019}, there is no operation $SpellOut$, which eliminates structure before transfer to SM. Further, note that `PF' and `LF', as internal levels of representation, are undefined, following \textcite[107]{ChomskyN_2004}. Rather, there is a single, unified cycle, with \TRANSFER\ handing over syntactic objects, called \textit{phases}, to \fSM\ and \fCI. The specifics of the cycle and the precise nature of phases will be discussed further below, in \autoref{sec:145}.

An expression \Exp\ is said to \textit{converge at an interface level \IL} if it is legible at \IL, in other words if the interface condition \IC\ at \IL, $\IC(\IL)$, is satisfied. \IC\ states that ``the information in the expressions generated by L must be accessible to other systems'' \parencite[106]{ChomskyN_2004}, which is, evidently, a requirement that language be usable at all---the barest possible metric of ``good design'', a key methodological assumption of MP. By contrast, \Exp\ \textit{crashes at \IL} if it does not meet $\IC(\IL)$. By extension, the computation of \Exp\ \textit{converges} if \Exp\ converges at both SM and C-I, otherwise it \textit{crashes}. A derivation will only crash if it fails to remove all features from the resulting SO that are \textit{uninterpretable} at the interfaces before \TRANSFER\ takes place. A derivation that has removed all such features will always converge, but to varying levels of \textit{deviance} as determined by the interface systems---a suggestion of \textcite[112]{ChomskyN_2004}, reinforced by \textcite[238]{ChomskyN.etal_2019}: ``concerns about ``overgeneration'' in core syntax [i.e. NS---LVS] are unfounded; the only empirical criterion is that the grammar associate each syntactic object generated to a <SEM,PHON> pair in a way that corresponds to the knowledge of the native speaker ... ``overgeneration'' must be permitted on purely empirical grounds, since ``deviant'' expressions are systematically used in all kinds of ways''. This point will prove crucial with respect to the discussion of labelling to follow, and will also receive further attention in \autoref{sec:145}.

\subsubsection{Features and the lexicon}\label{sec:143}

Features require further attention: why should uninterpretable features exist at all in a system adhering to princples of ``good design''? The \textit{Interpretability Condition}, that ``LIs have no features other than those interpreted at the interface, properties of sound and meaning'' is ``transparently false'' \parencite[113]{ChomskyN_2000}. Rather, I-language is characterised by what \textcite[54]{BiberauerT_2019} calls ``\textit{systematic departures from Saussurean arbitrariness}''---the presence of so-called `formal', grammatical features which play a role in \CHL\ but not directly at the interfaces.

An idea that persists, from its introduction in \textcite[\pnfmt{277} \textit{et seq.}]{ChomskyN_1995} is that uninterpretable features exist to capture the displacement property of language---long considered an `imperfection', but accepted by \textcite[note 29]{ChomskyN_2004} as, in fact, the most Minimal option. On the original formulation by \textcite{ChomskyN_2000}, the fact that both of these then-considered `imperfections', uninterpretable features and displacement, appear to be intimately connected suggests that ``the two imperfections might reduce to one'' \parencite[121]{ChomskyN_2000}. Furthermore, the optimal conclusion would be that dislocation itself is required by design---either as part of \IC\ or as a consequence of the nature of the operations within \CHL. The latter is demonstrated to be the case by \textcite{ChomskyN_2004}, with the introduction of \textit{internal \MERGE} (IM) and \textit{external \MERGE} (EM), building upon the unification of syntactic operations begun by \textcite{KitaharaH_1997}. $\MERGE(X,Y)$ is considered IM if $X$ is contained within $Y$, else it is considered EM. Displacement, following the `copy' theory of movement, comes for free as a consequence of the nature of \MERGE, which is its simplest formulation does not bar access to objects that have already been merged.%
\footnote{As will be discussed in \autoref{sec:420}, the term `copy' is a bit of a misnomer, hence the scare quotes. It is nevertheless standard to assume some lossely defined form of copy theory in Minimalist work.}
Further consequences of this will follow.

\subsubsection[\AGREE]{$\mathbfit{\AGREE}$}\label{sec:144}

The existence of uninterpretable features forms part of the justification for \AGREE, the final component of L as stated above. \AGREE\ forms a relation between two SOs, one of which is termed a \textit{probe}, the other a \textit{goal}. Standardly, the probe must c-command its goal, although there are other possibilities, which will be considered in the formalisation in \autoref{sec:480}. The goal is located via some mechanism of MS, again to be clarified in \autoref{sec:450}.

In many older articulations of the theory, \AGREE\ is taken as part of the more complex operation Move, which is composed of \MERGE, \AGREE, and a third operation, pied-piping, which remains poorly understood but is given much attention, for example in \textcite{ChomskyN_1995}. Following \textcite{ChomskyN_2004}, Move will not be taken to be a part of the theory. The suggestion is that all of its empirical import can be taken over with only the more minimal operations of \CHL, in combination with IC and third factors. Labelling has much to reveal here, as will be discussed in the following sections. Indeed, whether \AGREE\ is needed at all will come under scrutiny. A preliminary motivation for this is that both labelling and \AGREE\ are effectively realisations of MS. This strongly implies some kind of redundancy. If this is the case, and \AGREE\ can be abandoned, this would lead to a simplification of \CHL, a move in line with both methodological \pxref{ex:minprops:1} and ontological minimalism \pxref{ex:minprops:2}. This is the approach taken in Chomsky's most recent work, where \AGREE\ receives almost no mention \parencite{ChomskyN_2021}. Further discussion is reserved for \autoref{sec:400}.

\subsubsection{Phases and cyclicity}\label{sec:145}

In the interaction of the subcomponents of L, a need arises to identify the units that are available to take part in an operation within \CHL. Assume therefore that \CHL\ operates within a workspace \WS, which represents the state of a derivation at any particular point.%
\footnote{The idea of the workspace within the context of modern minimalism was most notably formalised by \textcite{CollinsC.StablerE_2016}, and finds further elaboration by \textcite{ChomskyN.etal_2019} and \citeauthor{ChomskyN_2019a} (\citeyear{ChomskyN_2019a}, \citeyear{ChomskyN_2019b}, \citeyear{ChomskyN_2021}).}
Operations are ``strictly Markovian'' \parencite[20]{ChomskyN_2021}, beyond even the standard Markovian property of derivations---\WS\ does not contain previously generated items, since these are eliminated by \MERGE, in accordance with a property of computational optimality termed \textit{Minimal Yield} \parencite[MY,][19]{ChomskyN_2021}, equivalently \textit{Restrict Resources} \parencite{ChomskyN_2019a}. The formal properties of derivations beyond this will be explored in more depth in \autoref{sec:430}.

The SOs generated by \CHL\ are assumed to be \textit{bare}, in the sense of \textcite{ChomskyN_1994}. Equivalently, they are formed only by the operation \MERGE. With the notion of \WS\ established, it is possible to define \MERGE\ as a function between workspaces%
\footnote{\WS\ is analogous to a ``working memory''---in a computational, not necessarily a cognitive sense, much like the tape of a Turing machine. See \textcite{WatumullJ_2012,WatumullJ_2015} for a possible formalisation of the linguistic Turing machine, in which these issues come to light.}
In previous formulations, \MERGE\ is typically considered to be a binary operation, which takes two SOs $X$ and $Y$ and combines them to form the set $\{X,Y\}$, itself an SO. For \textcite{ChomskyN_2021} who borrows much from \nptextcite{CollinsC.StablerE_2016}: \MERGE\ operates on a sequence of SOs $\sigma$, such that each SO in $\sigma$ is accessible and that $\sigma$ exhausts \WS; \MERGE\ is free to take any two objects and merge them together, mapping \WS\ to a new workspace $WS'$. The definition in \autoref{sec:400} offers a more precise account. An important consequence is that the application of \MERGE\ is free, in the sense of \textcite{ChomskyN.etal_2019}, meaning that constraints on \MERGE\ must fall out from the conjunction of IC, third factors, and other operations such as \AGREE. Labelling, it will be argued, surely also plays a role. This therefore does not entail that \MERGE\ must be `triggered', as assumed in stricter MGs and many other Minimalist theories such as that of \textcite{AdgerD_2003}.

Finally, it is worth briefly characterising the core functional categories (CFCs) and, in turn, the nature of phases. Following \textcite{ChomskyN_2000}, the CFCs are taken to be C, expressing force and mood (and possibly abbreviating a number of categories taken to form the \textit{left periphery}, following \nptextcite{RizziL_1997}), T, expressing tense and event structure, and v*, the light verb head of transitive constructions, expressing argument structure.%
\footnote{Cf. \autoref{fn:littleV} on \littleV.}
These functional categories are `core' in the sense of being the locus of agreement and dislocation generally. Following \textcite{ChomskyN_2008}, CP and v*P are phases, C and v* their respective \textit{phase heads}. This is arguably problematic, as T is very obviously involved in Case, \phiF-feature agreement and movement---with the EPP%
\footnote{Extended Projection Principle, classically formulated as the requirement that [Spec,TP] be filled (cf. \nptextcite{ChomskyN_1981}). Now, EPP-features are interpreted more generally, as the requirement that a head needs its specifier to be filled, usually by movement of the external argument in the case of T. This is the \textit{generalised} EPP-feature \parencite[see][]{HaegemanL_1996,LaenzlingerC_1998,RobertsI_2004}. EPP-features may be obviated by labelling and other considerations, as discussed further in \autoref{sec:300}. Eliminating the EPP has been a long-term goal in generative syntax---cf. \textcite{BoskovicZ_2007}.}
being a classic example. \textcite{ChomskyN_2008} resolves this by making explicit the idea of \textit{inheritance}: ``for T, \phiF-features and Tense appear to be derivative, not inherent: basic tense and also tenselike properties (e.g. irrealis) are determined by C (in which they are inherent)'' \parencite[143]{ChomskyN_2008}.%
\footnote{The earliest published mention of inheritance comes from \textcite{ChomskyN_2007}, actually inheriting the idea from Marc Richards, later published as \textcite{RichardsMD_2007}, which works off of a manuscript version of \textcite{ChomskyN_2008} distributed even earlier.}
Thus, ``Agree and Tense are inherited from C, the phase head'' \parencite[143-144]{ChomskyN_2008}.

Derivations proceed \textit{strictly cyclically}, phase-by-phase. Further, \CHL\ constructs objects in parallel in the workspace, but all operations occur effectively instantaneously at the phase level. As stated by \textcite[116]{ChomskyN_2004}: ``TRANSFER has a ``memory'' of phase length, meaning [] that operations at the phase level are in effect simultaneous''. This, presumably, makes the apparent countercyclicity of inheritance only apparent. Further, operations apply freely---order does not matter; any deviant or crashing derivations that result are discarded by the interfaces. More conclusions are possible, as reiterated by \textcite[143]{ChomskyN_2008}: ``along with Transfer, all other operations will also apply at the phase level, as determined by the label/probe. That implies that IM should be driven only by phase heads''. Labels clearly play a significant role: the label of the phase is always the probe for \AGREE\ (obscured by the fact that the agreement properties of T are inherited from C). The interactions between labelling and agreement are discussed in \autoref{sec:480}.

One of the most important consequences of strict cyclicity is the Phase-Impenetrability Condition (PIC) as in \pxref{ex:PIC}, from \textcite[108]{ChomskyN_2000}.

\begin{example}\label{ex:PIC}
\setlength{\parskip}{0pt}\setlength{\parsep}{0pt}
\textit{Phase-Impenetrability Condition}

In phase $\alpha$ with head H, the domain of H is not accessible to operations outside $\alpha$, only H and its edge are accessible to such operations.
\end{example}
\noindent
The PIC proves critical in discussions of locality, taking the place of Subjacency \parencite{ChomskyN_1973} and Barriers \parencite{ChomskyN_1986} in previous frameworks. The place of labelling within the context of locality and the phase will be a key point of analysis within \autoref{sec:300}.
