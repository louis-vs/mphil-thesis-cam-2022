\subsection{X-bar theory and `projection'}\label{sec:230}

With the development of the EST and the Y-model, however, also came a major refinement of the rule system in the form of the development of X-bar theory \parencite{ChomskyN_1970,JackendoffR_1977}, which also marks a major step in the development of labelling.%
\footnote{The presentation of X-bar theory here naturally glosses over details that were subject to debate at the time, such as the number of `bar'-levels that a head is able to project. A model roughly following \textcite{ChomskyN_1981} is assumed, incorporating some suggestions towards unification proposed by \textcite{MuyskenP_1982}. Chronologically, X-bar somewhat predates the Y-model, instead being created to deal with certain properties of nominalisations \parencite{ChomskyN_1970}; this and other historiographical concerns extend beyond the scope of this discussion.}
The central notion of X-bar theory is that every phrase is \textit{endocentric}---to wit, every phrase has a \textit{head}. In modern terms, as elaborated in \autoref{sec:140}, a head is an LI. It would not be a mischaracterisation, then, to say that a critical aspect of the development of X-bar theory was in the adoption of what in modern terms would be called a deterministic labelling algorithm---what was in the terminology of the time called \textit{projection}. The head \textit{projects} its properties up through the structure. Returning to our examples, the rule \pxref{ex:PSrulearb21} is is no longer isomorphic to the arbitrary rule \pxref{ex:PSrulearb22}, because the relation between $\zeta$ and $\beta$ ($\zeta$ is the head of $\beta$) must be represented in the structure. Hence, an endocentric formulation of the rule is as in \pxref{ex:PSrulearb3}.

\begin{example}\label{ex:PSrulearb3}
$\zeta P \rightarrow \epsilon P\ \zeta$
\end{example}

Introducing endocentricity is not sufficient within the X-bar framework, however. Its second vital contribution was of intermediate `bar' levels: intermediate projections that were effectively \textit{exocentric} when considered maximally locally. Bar-level projections offer a position in the structure for specifiers and adjuncts. The structures entailed by bar-levels are exocentric in terms of immediate dominance, as only the intermediate projection that immediately dominates the head dominates a head at all, since higher intermediate projections alongside the maximal projection dominate only other non-minimal projections. The head is still determined at every point, however, since a bar-level intermediate projection is marked as such (hence the bar), meaning that its category must continue to project up the tree, maintaining this property of projection, as the head is represented at every level within a category. When understood less strictly than in terms of immediate dominance, X-bar structures do thus conform to endocentricity, in the sense that every structure has a head and the head of a given structure can be determined at every point. Meanwhile, the maximal projection in the specifier, complement or adjunct position is barred from projection, being already marked as a maximal projection. As a result, $\epsilon$ in \pxref{ex:PSrulearb2} is also altered in the endocentric \pxref{ex:PSrulearb3}, since it is neither a phrase, nor a head of any other phrase. It also needs to be part of an endocentric structure, in order to satisfy the X-bar schema. A X-bar compliant representation of the structure described by the example grammar presented thus far is therefore shown in \pxref{ex:PSrulearb4}, with the bar-levels of $\zeta P$ represented, and the internal structure of $\epsilon P$ abbreviated.

\begin{example}\label{ex:PSrulearb4}
    \begin{forest}
        [{$\zeta P$} [{$\epsilon$P}] [{$\bar{\zeta}P$} [{$\zeta$}]]]
    \end{forest}
\end{example}

The significant development here is that the schema can be generalised. The major claim of X-bar theory is thus that all phrases satisfy the general schema in \pxref{ex:xbarschema}, the informal rule equivalent given in \pxref{ex:xbarrules}.

\begin{example}\label{ex:xbarschema}
    \begin{forest}
        [XP [Spec] [{$\bar{X}$} [X] [Comp]]]
    \end{forest}
\end{example}

\begin{subexamples}\label{ex:xbarrules}
    \item $XP \rightarrow (Spec)\ \bar{X}$
    \item $\bar{X} \rightarrow \bar{X}\ Adj$ [optional, unordered, can be repeated]
    \item $\bar{X} \rightarrow X\ (Comp)$
\end{subexamples}
\noindent
The \textit{specifier} Spec, \textit{complement} Comp, and adjuncts must be maximal projections, viz.~they themselves need to be fully qualified XPs. The presence of Spec and Comp and their required properties in particular instantiations of the schema reduce to selection (equivalently, subcategorisation), which is a lexical concern.

In effect, what this system does is greatly reduce the generative capacity of the PS-rule system by constraining the available rules to those presented in \pxref{ex:xbarrules}. It also makes a strong claim of endocentricity, captured explicitly in the \textit{Projection Principle} of Government-Binding theory \parencite[GB;][x]{ChomskyN_1981}. GB constitutes the next significant development in the theory, incorporating X-bar theory, alongside a highly modular organisation of the grammar accompanied by many improvements on the specifics of the EST model. The Projection Principle is defined as in \pxref{ex:PP}.

\begin{example}\label{ex:PP}
\setlength{\parskip}{0pt}\setlength{\parsep}{0pt}
\textit{Projection Principle}

[L]exical structure must be represented categorically at every syntactic level. \parencite[84]{ChomskyN_1986}
\end{example}
\noindent
The ``categorical'' representation is, in modern terms, the \textit{label}. Following the Projection Principle, the syntax is separated from the lexicon: lexical properties are \textit{projected}, allowing their properties to move upwards in the structure, and forwards in the derivation.%
\footnote{Abstracting somewhat from the representational/derivational issue. See \autoref{sec:130}.}
In part as a result of this principle, phrase structure rules can be eliminated entirely---all structure is formed via X-bar projections from lexical items. In modern terms, the labels within the structure allow it to be correctly interpreted at the interfaces, subsuming the GB notions of filters and conditions at the various levels of representation. Importantly, endocentricity comes as a corollary of the Projection Principle and the X-bar schema: there is simply no way of generating a structure that is not ultimately headed/projected by an LI, viz.~one that is exocentric. Crucially, this interpretation is made possible by the assumption that the putatively exocentric bar-level projections are invisible in terms of endocentricity. Bar-invisibility holds for other formal relations involved in GB, here glossing over the technical details of the contemporaneous theory. Nevertheless, the principle that syntactic objects can be invisible to future operations (hence being in some sense `frozen' in place) continues to be of relevance in present-day theorising and affords further attention in order to break down the assumptions ingrained within the X-bar framework, an objective which indeed proves to be a theme of subsequent developments, as will become clear notably in \autoref{sec:300}.

Note that the fact that the Projection Principle deduces a deterministic method of identifying labels, allowing categorial information to be preserved throughout the derivation, was not part of the original motivation for the principle within the contemporaneous theoretical context. Instead, the Projection Principle was necessary to deal with issues surrounding subcategorisation. As detailed by \textcite[29-34]{ChomskyN_1981}, in ST there is a redundancy between lexically-specified subcategorisation frames and PS-rules such as \pxref{ex:PSrulearb21}, which specifies that, within an NP constructed by this categorial rule, a determiner can have a noun complement. In our terms, this redundancy is unsatisfactory both by methodological minimalism and computationally in terms of MaxTP. Reducing the categorial component to the X-bar schema and forcing lexical information to project upwards through the representation was the proposed solution. Furthermore, the Projection Principle coupled with X-bar theory naturally leads to a further hypothesis on the nature of syntactic structures. The hypothesis is dubbed by \textcite[178]{HornsteinN.etal_2005} the \textit{Periscope Property}, and it claims that ``there are [] no known cases where a syntactic relation cares about anything, but the head'' \parencite[178]{HornsteinN.etal_2005}. Generally speaking, in configurations like \pxref{ex:periscope}, $\beta$ may be selected by $\alpha$ no matter the intervening material.

\begin{example}\label{ex:periscope}
    \begin{forest}
        [{} 
            [{$\alpha$} ]
            [{$\beta P$}
                [{... $\beta$ ...},roof ]
            ]
        ]
    \end{forest}
\end{example}
\noindent
As per \textcite[98]{HornsteinN_2021}, the Periscope Property highlights three facts about subcategorisation \pxref{ex:periscope:subcat}.

\begin{subexamples}\label{ex:periscope:subcat}
    \item In the configuration \pxref{ex:periscope}, $\alpha$ \textit{can} select for $\beta$.
    \item In \pxref{ex:periscope}, $\alpha$ \textit{cannot} select for anything else within $\beta P$.
    \item The selection relation is \textit{linearly unbounded}.
\end{subexamples}

To illustrate, take a determiner D selecting for an NP. The structure of the NP is irrelevant to the syntactic operation of selection---the contents of any complement, specifiers or adjuncts are ignored, even if the heads of these projections are in some sense `closer' to the selecting head. Rather, the N head of the selected NP is always considered the target for subcategorisation. Violating this principle would be catastrophic for any theory of subcategorisation that otherwise adheres to the model; retaining the effects of the periscope property thus seems to be essential in any theory of syntactic relations that resembles GB in this manner.%
\footnote{\label{fn:telescope}There is a growing body of literature that does in fact reject the Periscope Principle in favour of what \textcite{BrodyM_2000} denotes \textit{telescoped} representations. Such a theory as developed by \textcite{AdgerD_2013} attributes the purported effects of the periscope illusion to properties of C-I, which is purported to provide a skeletal schema to aid interpretation of telescoped structures. Whilst this work is thus thoroughly Minimalist in character, reducing properties of the grammar to interface conditions, it requires a radically different approach to labelling than is more standardly assumed in the literature, and is thus beyond the (peri)scope of this thesis. One must also keep in mind Chomsky's warning as mentioned above, that ``reduction in one component [not be] matched or exceeded elsewhere'' \parencite[13]{ChomskyN_1981}. The reader is referred to \textcite{AdgerD_2013} for a complete exposition of a telescoped theory and some consequences.}
This is not to say, however, that projection must retain its traditional form. Referring back to the example given, the periscope effect may in fact be derivable from the properties of the search operation which finds the `nearest' head. This hints at the prospect of decoupling labelling from structure building, a much later development in the theory, and one given considerable attention below.
