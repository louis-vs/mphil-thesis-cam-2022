\subsection{Merge and labels}\label{sec:250}

The possible options for the (informal) definition of the fundamental combinatorial operation of syntax are thus as listed in \pxref{ex:twomerges}.

\begin{subexamples}\label{ex:twomerges}
    \item\label{ex:twomerges:label} $Merge(X,Y) = \{K,\{\alpha,\beta\}\}$, where $K\in\{\alpha,\beta\}$
    \item\label{ex:twomerges:nolabel} $Merge(X,Y) = \{X,Y\}$
\end{subexamples}

Some final notes on option \pxref{ex:twomerges:label}: it is worth considering why the value of K should be restricted as stipulated. \textcite[4]{ChomskyN_1994} justifies the restriction of the label $K$ by appealing to both IC and TLTB. First, by TLTB (both Inclusiveness and FI) $K$ must be $\alpha$, $\beta$, or their union or intersection. The union can be ruled out: ``the union will not only be irrelevant but contradictory if $\alpha$, $\beta$ differ in value for some feature'', assumed to be the ``normal case'' \parencite[4]{ChomskyN_1994}. On the other hand, the intersection also does not suffice: ``the intersection of $\alpha$, $\beta$ will generally be irrelevant to output conditions, often null'' \parencite[4]{ChomskyN_1994}, null for the same reason that the union would be contradictory, viz.~because features may differ in value. The latter two options would be nonsensical at the interface, hence $K$ must be one of the two mergees. Therefore, projection is maintained, explicitly, encoded into the combinatorial operation. Abandoning labels entirely would obviate the need to consider this issue, but if labels are maintained in any form, it is clearly important to consider how the set of possible labels is restricted.

\textcite[247]{ChomskyN.etal_2019} note that it is a nontrivial question as to why the label in \pxref{ex:twomerges:label} cannot undergo head movement. Indeed, an additional problem here is that labels are already indistinguishable from copies formed by movement, without further inspection. Whilst structure preservation dictates that movement of a head in such a manner would be prohibited in the case where $K=X=\alpha^{\circ}$, for $\alpha^{\circ}$ a head, at higher levels of projection the syntactic object that serves as the label would be indistinguishable from a phrase at the point of labelling, and thus could be condidered a case of raising.%
\footnote{With `structure preservation' here formalised as the Uniformity Condition on Chains by \textcite[253]{ChomskyN_1995}, see \textcite[f.n.~1]{RobertsI_2001}.}
However, assuming the definiton of terms introduced by \textcite{ChomskyN_1995} and highlighted by \textcite{SeelyTD_2006}, this additional problem is avoided (see also \autoref{fn:seely}). For \textcite{ChomskyN_1995}, any structure formed by Merge is a term, and the members of the members of any term are terms. This recursive ``members of members'' definition ensures that labels, which are merely members of the set formed by \pxref{ex:twomerges:label}, can never be terms. However, another consequence of defining terms as such, as \textcite{SeelyTD_2006} demonstrates, is that labels must be invisible to the syntax anyway, as, not being terms, they cannot participate in syntactic relations. Thus, the postulation of labels within the syntax seems somewhat pointless, notwithstanding their potential significance at the interface. Indeed, as \textcite{CollinsC.SeelyTD_2020} argue, the now mainstream adoption of \pxref{ex:twomerges:nolabel} suggests that label-free syntax is in some fundamental sense the right approach.

