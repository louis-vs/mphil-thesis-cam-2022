\subsection[\Agree]{$\mathbfit{\Agree}$}\label{sec:480}

The fundamental purpose of syntax is to eliminate uninterpretable features that are present in lexical items so that they can be interpreted at the interfaces (cf. \autoref{sec:143}). Relations between features within separate SOs encode syntactic dependencies \parencite[cf.][]{AdgerD_2010}. All such dependencies need to be evaluated within the syntax in order to be legible. In many cases, application of \Merge\ suffices to satisfy feature dependencies. For instance, take subcategorisation: as mentioned in \autoref{sec:200} subcategorisation was one of the motivations for the development of labelling theory because of the periscope property, whereby the selecting head need only look at the label of the selected structure. The original formulation of a separate, non-Merge relation by \textcite{ChomskyN_2000,ChomskyN_2001} was motivated by two constructions in particular: English existential constructions and long-distance dependencies in many languages.

\textcite{MilwayD_2021} formalises Agree within the \CS\ framework. This could reasonably trivially be adapted into the present framework. However, I would suggest a further revision: namely, that Agree operates on the labels provided by \Label\ (either top-down or bottom up) rather than raw SOs. Since Agree is not the focus of this thesis, a full formalisation will not be provided. It seems reasonable to assume that there is a definable $\Agree(N, R)=R'$ function, which maps a set of labels $N$ and a ledger $R$ to a new ledger $R'$.

