\section*{Abstract}
\addcontentsline{toc}{subsection}{Abstract}
In any theoretical discipline, it is important to pay careful attention to the precise nature of the formal primitives that are employed in explanations of natural phenomena. The case is no different for language, on the particular conception that a language is a state of knowledge of an individual (I-language), and that the faculty of language is a biological organ that must be understood as a part of the natural world. This biological connection crucially forms the basis of the field of biolinguistics. Under particular investigation here is the hierarchical structure-generating engine of I-language, what is termed (generative) syntax.

This thesis has three broad objectives. The first is to clarify and justify the approach to language outlined above, and to summarise and evaluate theoretical devices that are standardly accepted within the Minimalist Programme, a particular framework of I-linguistic investigation. The second and third objectives concern `labelling', the hypothesised process by which the objects constructed by syntax are given names. It is established that almost every aspect of the purpose and operation of labelling is up for debate. The second object is thus to clarify the central issues that emerge in the labelling literature, first grounding the discussion in terms of the historical development of syntactic theory, then proceeding to evaluate more recent proposals. The final task is to (partially) formalise a particular theory of I-language, modifying and extending the framework constructed by \textcite{CollinsC.StablerE_2016}. This process reveals the fundamental fragility of many of the critical concepts underlying Minimalist syntax. Suggestions towards improving this situation and extending the empirical coverage of the theory are also presented.
