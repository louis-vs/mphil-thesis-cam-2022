\subsection{Form of labels}\label{sec:330}

Question \pxref{ex:questions:what} arises as a result of the change of direction impelled by \textcite{ChomskyN_2013}: a label no longer needs to be an LI, rather ``it must be that LA seeks features, not only LIs -- or perhaps seeks only features, in which case it would be similar to probe-goal relations generally, specifically Agree'' \parencite[45]{ChomskyN_2013}. This is reminiscent a pre-BPS label system of the sort described in \autoref{sec:200}, where categorial features, potentially with associated features, projected. Labels are not limited to LIs, but are more free, as may be determined by the structural context. It also brings Label even closer to Agree, which has always been assumed to search for individual features. This has also previously been assumed for Move (see \nptextcite[261-271]{ChomskyN_1995} on the `Move F' operation). Selecting only the required features for the label, potentially via sharing, is clearly more optimal, by MinSC.

With this in place, it must be established whether there are any restrictions regarding which features can serve as labels. Conforming with TLTB, it would be most optimal \emph{not} to have any stipulations on this matter. Rather, in principle, any feature can serve as a (potentially shared) label. Nevertheless, certain labels may be blocked for at least three reasons: (a) certain labels may never arise because the configurations in which they would emerge never arise; (b) some labels may create structures that are nonsensical at the interface; (c) the selection of certain labels may cause the derivation to halt irrecovably (crash).

\textcite{ChomskyN_2015} does stipulate that certain heads are `weak' and cannot provide labels. He notes this for roots, which I derive simply from their featurelessness in \autoref{sec:450}, Theorem X. More tricky is why T in particular should be unable to serve as a label. I will note two related possibilities, which there is no room to explore: (a) Chomsky 2021 says T doesn't exist; (b) T never serves as a label for independent structural reasons. Each of these options requires empirical investigation that is beyond the scope of this thesis. 

Notwithstanding this, I will propose a hypothesis on the form of labels which should guide future analytic Minimalist work into the form of labels, on the basis of the preceding discussion.

\begin{example}
    \textit{Hypothesis on the form of labels}

    The features that can compose labels are determined by the set of derivable structural configurations in an I-language. There are no extensional bans on valid labels.
\end{example}

I will not return to this directly, although it serves as a guiding light within \autoref{sec:400}.
