\subsection{`Label-free' (narrow) syntax}\label{sec:260}

Returning to the debate encapsulated in \pxref{ex:twomerges}, we can now consider some consquences of adopting \pxref{ex:twomerges:nolabel}. As per \textcite{CollinsC_2002}, an alternative mechanism is needed to determine the next step in a derivation. He terms this device the \textit{locus of derivation}, and it effectively serves as a temporary label stored in working memory which directs the progression of the derivation.%
\footnote{Using `working memory' in a computational, not neurobiological, sense.}
This, in effect, allows the derivation to keep track of its current state, without storing this state as a label within the structure. In turn, this enables a key simplifying consequence, according to \textcite[48]{CollinsC_2002}: ``[t]he major difference between a locus and a label is that there is only one locus in a derivation, while there are many labels [since] each constituent has a different label''. Interestingly, however, as noted by \textcite{SeelyTD_2006}, this cannot be the case---rather, ``it is arguable that there are as many [l]oci in a derivation as there are lexical categories with unchecked probes/selectors'' \parencite[213]{SeelyTD_2006}. Furthermore, in the case of categories that serve as probes/selectors, the label and the locus are equivalent, as per \textcite{CollinsC_2002}. Therefore, it is not at all clear that labels have been fully eliminated under the \textcite{CollinsC_2002} model. Rather, a middle ground has appeared to have emerged, in which labels are eliminated from Merge as per \pxref{ex:twomerges:nolabel} but labels/loci are retained in order to make derivations possible at all.


An alternative middle ground is already staked by \textcite{ChomskyN_2000}, asking the question: ``Are labels predictable?'' \parencite[133]{ChomskyN_2000}, effectively approaching the issue from the other side to the \textcite{CollinsC_2002} approach. Firstly, assume $label(\alpha)=\alpha$, where $\alpha$ is an LI (already implicit in \nptextcite{ChomskyN_1994}). (Set-)Merge is assumed to be symmetrical, unlike the original formulation above, where the formation of the label is intrinsic to the operation, stipulated with recourse to IC. This being the case, in order to create the asymmetry presumed to be required for a label to be formed, and thus for the structure to be legible, one of $\alpha,\beta$ must be a \textit{selector}, with which the other has merged in order to satisfy a selectional requirement. The selector is assumed to provide the label in every case; \CHL\ is therefore able to identify the label. On the basis of this, \textcite[135]{ChomskyN_2000} claims that ``[i]n all cases, then, the label is redundant ... The label is determined and available for operations within \CHL\ or for interpretation at the interface, but is indicated only for convenience''. The exact position is staked out by \textcite[109]{ChomskyN_2004}: ``a label [] is always a head. In the worst case, the label is determined by an explicit rule []. A preferable result is that the label is predictable by general rule. A still more attractive outcome is that [I-language] requires no labels at all''. The latter option would be the `label-free' option advocated by \textcite{CollinsC_2002,SeelyTD_2006}. Further, ``operations are ``driven'' by labels'' \parencite[109]{ChomskyN_2004}---if this is the case, it leads to the even stronger conclusion that there can be no general Spec-head relation, as was essential to the analysis in \textcite{ChomskyN_2005} as a consequence of the definition of `checking domain', a hangover of GB's m-command which allowed Spec-head relations to take place. Similarly, as noted by \textcite{BlumelA_2017a}, there is also consequently ``no upper limit to the number of specifiers, a notion that has no status in the theory'' \parencite[51]{BlumelA_2017a}, a welcome result, abandoning a stipulation of X-bar theory in line with TLTB and also enabling a number of empirical results to be derived (see \nptextcite{RichardsN_2001}).

\textcite{ChomskyN_2008} offers an informal definition of the aforementioned ``general rule''---in other words, a \textit{labelling algorithm} (LA)---as restated in \pxref{ex:OPlabels}

\begin{subexamples}[preamble={\textit{The Labelling Algorithm of \textcite[145]{ChomskyN_2008}}}]\label{ex:OPlabels}
    \item\label{ex:OPlabels:a} In $\{H,\alpha\}$, $H$ an LI, $H$ is the label.
    \item\label{ex:OPlabels:b} If $\alpha$ is internally merged to $\beta$, forming $\{\alpha,\beta\}$ then the label of $\beta$ is the label of $\{\alpha,\beta\}$
\end{subexamples}
\noindent
The IM condition of \textcite{ChomskyN_2000} is thus retained, albeit within a theory adopting Simplest Merge plus LA. This is implicitly problematic: there is in principle no formal difference between the operations EM and IM themselves---they are merely instances of Merge with slightly different inputs---so there is a key stipulation in play. A further problem, noted by \textcite[f.n.~34]{ChomskyN_2008}, is that the LA \pxref{ex:OPlabels} is unable to label $\{\alpha,\beta\}$ structures that are formed by \emph{external} Merge. However, this scenario must occur at least twice in a derivation. For one, the very first step of a derivation must by definition be EM of two LIs. Secondly, EM of the external argument (to vP) necessarily involves EM of two phrases. In addition to these issues, it is apparent that this labelling algorithm, as with its predecessors, needs to take place at the time of Merge. This is required by LA \pxref{ex:OPlabels} because of \pxref{ex:OPlabels:b}, where the context of Merge determines the label. These issues receive further attention in \autoref{sec:300} below.

This primitive first attempt at an LA is subsequently developed, notably by \textcite{ChomskyN_2013,ChomskyN_2015}, into what \textcite[4]{BoskovicZ_2016a} terms a `label-or-not' system. Within such a theory, labelling is not a part of Merge, as in \pxref{ex:twomerges:label}, and thus Simplest Merge \pxref{ex:twomerges:nolabel} is adopted, as in \textcite{ChomskyN_2004}. However, unlike in the label-free system \parencite{CollinsC_2002,SeelyTD_2006}, it is still required for there to be a way of assigning labels ``that license[] SOs so that they can be interpreted at the interfaces'' \textcite[43]{ChomskyN_2013}. Alongside this move, \textcite{ChomskyN_2013} abandons the stipulation that the label may be determined by the operation of IM. There must, then, be a way of assigning labels which enables the correct interpretation to be provided for non-deviant sentences. To achieve this, \textcite{ChomskyN_2013} introduces an LA which applies at the phase level, before transfer to the interfaces. Various, more specific conceptions of LA are the topic of \autoref{sec:300} to follow. Notably, the converse---that labels may trigger operations including IM---remains a possibility. Indeed, this possibility is reframed as a central argument for the importance of LA, which appears to recast labels in a manner very similar to that of the loci of \textcite{CollinsC_2002}, albeit without abandoning the use of labels entirely.

