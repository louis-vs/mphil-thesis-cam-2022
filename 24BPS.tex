\subsection{Bare Phrase Structure}\label{sec:240}

For now, the next step in the development of labelling theory comes with the introduction of the Bare Phrase Structure model \parencite[BPS;][]{ChomskyN_1994}. It may initially appear that endocentricity can no longer result from a generalisation of the Projection Principle as described above: either the simplest combinatorial operation doesn't project at all, leaving the nature of the object formed unknown, or projection has to be encoded into the operation directly, which appears to be a stipulation. In the original formulation of BPS, \textcite{ChomskyN_1994,ChomskyN_1995} predominantly adopts the latter option: the operation `Merge' produces a labelled object $\{K,\{\alpha,\beta\}\}$, where $K\in\{\alpha,\beta\}$---with this latter stipulation, the Projection Principle remains intact, now without the baggage of the full X-bar structure. One consequence of this is in setting out a programme by which extraneous phrase structure introduced by the X-bar can be eliminated wherever possible, a challenge notably taken up by \textcite{BoskovicZ_1997} in his postulation of the \textit{Minimal Structure Principle} \parencite[25]{BoskovicZ_1997}, an evaluation metric which selects the representation that makes use of the fewest projections.

Another important consequence of BPS is that the notions of `intermediate' and `phrasal' projections are eliminated from the representation. With the minimalist stipulation $K\in\{\alpha,\beta\}$, it is simply not possible for these diacritic properties to be stored in the structure. This is a positive development: from the perspective of methodological minimalism, diacritics are highly disfavoured, as they present issues of arbitrariness and complexity along similar lines to the objections against PS-rules discussed above. Put simply, they violate the Inclusiveness Condition \pxref{def:inclusiveness}, a subcondition of TLTB (see \autoref{sec:130}). Furthermore, the information previously represented by the bar ($\bar{\ \ }$) and phrase (P) diacritics can instead be derived relationally from the structure relatively trivially using the set of definitions in \pxref{ex:minmaxproj}, adapted from \textcite[242-243]{ChomskyN_1995}, actually following a suggestion originally from \textcite{MuyskenP_1982} in an X-bar context \parencite[cf. also][]{ChomskyN.LasnikH_2015}.

\begin{samepage}% TODO: sort out this page break
\begin{subexamples}[preamble={\textbf{Levels of Projection} \parencite{ChomskyN_1995, MuyskenP_1982}}]\label{ex:minmaxproj}
    \item \textbf{Maximal Projection} ($X^{max}$, XP): a category that does not project any further.
    \item \textbf{Intermediate Projection} ($\bar{X}$, X'): a category that both is a projection and itself projects.
    \item \textbf{Minimal Projection} ($X^{min}$, $X^{\circ}$, X): a category that is not a projection, i.e. a head.
\end{subexamples}
\end{samepage}
\noindent
Note that the diacritic symbols used in \pxref{ex:minmaxproj} are not to be interpreted literally---that is, they exist solely for expositional convenience and are not actually present in the computational system. As a result, including the label within the Merge operation appears to enable maximally efficient computation of the projection level of a particular constituent: minimising space complexity by omitting diacritics, and minimising the search space and in turn time complexity of calculating the projection level. Endocentricity is also maintained; indeed, endocentricity is even more apparent: since diacritically labelled `bar-levels' are completely eliminated from BPS, and adopting the $K\in\{\alpha,\beta\}$ assumption, the head of a phrase is identical to the label of the phrase. In all cases, the head therefore must be the first element identified by search.%
\footnote{\label{fn:seely}Unfortunately, this generalisation does not actually hold when considering the full exposition of BPS presented by \textcite{ChomskyN_1994}. As pointed out by \textcite{SeelyTD_2006}, \textcite{ChomskyN_1994} deliberately excludes labels from being \textit{terms} in his definition thereof. \textcite{SeelyTD_2006} argues that labels must therefore be syntactically inert, incorporating this into an argument for the elimination of labels following \textcite{CollinsC_2002}. This argument will receive further attention in \autoref{sec:250} below; for now, though, it suffices to state that the heads are apparent at every level of a BPS representation, without further stipulations.}
Similarly, periscoping is maintained as all features of the lexical head are necessarily accessible at every level of projection, as all features of the head travel with the label, because of the condition that $K$ must be identical to (viz.~a copy of) the head. A second consequence of BPS, one that is supported by the definitions in \pxref{ex:minmaxproj}, is that a category can serve as both a minimal and maximal projection simultaneously, represented $X^{min/max}$. Hence, the full projection that is required by the default X-bar schema, is no longer stipulated in BPS, instead being optional, determined by other properties of a particular derivation. The BPS model is thus much more flexible than that of X-bar, a flexibility that hinges upon the nature of the implicitly computed labels of different categories.

Nevertheless, the earlier concern over the stipulation of $K\in\{\alpha,\beta\}$ warrants further investigation. It is clear from the perspective of TLTB that it would be more optimal to eliminate labels in favour of \textit{Simplest Merge}, $Merge(X,Y) = \{X,Y\}$, assuming that the necessary generalisations, such as the derivability of the X-bar projection levels in \pxref{ex:minmaxproj}, were able to be maintained. In fact, precisely this is achieved by \textcite{CollinsC_2002}, who argues that labels can be eliminated from syntax in favour of Simplest Merge. As admitted, however: ``[s]ince virtually every syntactic analysis in the generative tradition makes use of labels on phrasal categories[], the task of eliminating labels from syntactic theory is enormous'' \parencite[44]{CollinsC_2002}. In spite of this, the postulation of a `label-free' syntax plausibly constitute a productive research programme, with clear empirical consequences.

