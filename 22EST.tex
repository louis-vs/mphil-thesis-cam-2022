\subsection{EST and levels of representation}\label{sec:220}

For mostly independent contemporaneous reasons, ST's deep/surface-structure dichotomy was abandoned in favour of the more articulated \textit{Y-model}, postulated within the Extended Standard Theory (EST) as represented by \textcite{ChomskyN_1973,ChomskyN_1976b,ChomskyN_1977} and subsequent work, and diagrammed in \pxref{ex:Ymodel}. 

\begin{example}\label{ex:Ymodel}
    \begin{forest}
        for tree={edge={->}}
        [[{D(eep)-structure} [{S(urface)-structure} [P(honetic)F(orm)] [L(ogical)F(orm)]]]]
    \end{forest}
\end{example}
\noindent
Each node represents a \textit{level of representation}, while each arrow represents a set of transformational rules, or base rules in the case of generation of deep structure. Further, each level of representation may impose representational \textit{filters}, like the Case Filter \parencite{VergnaudJR_2008}, which disqualify certain structures from attaining a valid interpretation. This more finely articulated theory affords further resolution to the kinds of tranformational rules available, clearly demonstrating a `divide-and-conquer' approach to theorising, the kind which could be said to have influenced the extensive modularisation characteristic of the later development of the EST into the Government-Binding (GB) model summarised by \textcite{ChomskyN_1981} and explored below. Crucially, too, this approach entails that both PF and LF are generated on the basis of S-structure, a radical departure from ST which took deep structures to represent the semantic structure of a sentence.

Though with the caveat that this temporarily breaks from the chronology, it is beneficial to compare the EST's Y-model with the Minimalist architecture introduced by \textcite{ChomskyN_1993}, graphically represented in \pxref{ex:MParch}, and that makes up a significant part of the foundation of the theory outlined in more detail earlier, in \autoref{sec:130}.

\begin{example}\label{ex:MParch}
    \begin{forest}
        [{NS} [{SM}]  [{C-I}]]
    \end{forest}
\end{example}
\noindent
As is visually clear comparing \pxref{ex:Ymodel} and \pxref{ex:MParch}, the more recent model abandons the levels of representation (and of derivation, following the application of transformational rules) which obscure the relation between narrowly syntactic structures and interface representations. Curiously, then, this is in spirit a return to the more direct access to structures enabled by ST, albeit without the added complexity of the transformational component, following the unification of the generation and transformational components enabled with the operation Merge, introduced by \textcite{ChomskyN_1993}.%
\footnote{Move was at this point still considered a more complex operation than Merge, as detailed in \autoref{sec:130}.}

Thus far I have described the computational symbols involved in ST's PS-rules to be entirely arbitrary, as the examples in \pxref{ex:PSrulearb1} and \pxref{ex:PSrulearb2} demonstrate. This could, however, be regarded as a misrepresentation of Chomsky's (\citeyear{ChomskyN_1965}) position. \textcite{ChomskyN_1965} explicitly comments on the issue of the set of symbols which he assumes throughout the work, noting that they are not necessarily `substance-free' (see \autoref{sec:150}). He asks ``whether the formatives and category symbols used in Phrase-markers have some language-independent characterization, or whether they are just convenient mnemonic tags, specific to a particular grammar'' \parencite[65]{ChomskyN_1965}, suggesting that ``these elements [] are selected from a fixed, universal vocabulary'', though conceding that the question ``is generally held to involve extrasyntactic considerations of a sort only dimly perceived'' \parencite[66]{ChomskyN_1965}, which I would interpret as being considerations of (semantic) substance---correlating with the phonetic substance proposed as a basis of phonological features \parencite{ChomskyN.HalleM_1968}. This prospect relates to the idea that a `semantic spine' of some sort could have a direct and significant constraining effect on the set of grammatical syntactic outputs, which forms a central hypothesis within a range of theories that could be considered Minimalist to greater or lesser extents. Whilst a full review is out of the question, such approaches are detailed for instance by \textcite{WiltschkoM_2014,AdgerD_2013,StarkeM_2004,BrodyM_2000}. This interesting property of the Minimalist architecture \pxref{ex:MParch} marks a consequence of the reduction of the Y-model which is incredibly important yet not immediately obvious. In the Y-model \pxref{ex:Ymodel}, it is impossible for LF considerations to dictate in any way the operations of the transformational component which generates S-structure, as a result of the modularity entailed by the architecture. The Minimalist architecture, meanwhile, enables regular and rapid interactions between narrow syntax and interpretation---indeed, this is allowed to such an extent that ``access can in principle take place at any stage of the derivation'' \textcite[7]{ChomskyN_2021}.%
\footnote{With the caveat that this access standardly occurs at the phase level---``[a]ccess at any other stage of the derivation will yield some form of deviance or incoherence'' \textcite[23]{ChomskyN_2021}. On deviance, cf. \textcite{ChomskyN.etal_2019,ChomskyN_2020}.}
Models that make use of so-called `telescoped' representations, in particular that of \textcite{AdgerD_2013}, rely on labelling being heavily driven by C-I (see also \autoref{fn:telescope}). Whilst only assuming a backseat role in discussions around the (E)ST period, within the theory, labelling is actually held as a deeply ingrained assumption, inimicly tied to the procedure of structure generation. The base rules necessarily directly encode the labels of constituents, which then proceed to have an impact at every subsequent stage of the derivation. One issue that immediately arises from this is that the labels need to be maintained in computational memory for this entire period---once the derivation reaches S-structure, there is no guarantee that the label is predictable from the label-less structure. This massively violates MinSC, as per the framework for computational optimality introduced in \pxref{ex:mobbscompopt}. This contrasts with the Minimalist architecture, in which labels can be separated from the narrow syntax, precisely because their information is not needed until interpretation. There appears to be a link between these two concepts: the degree of interpretive influence on the state and continuation of the derivation, and the theoretical status of the identifiers assigned to syntactic objects. These observations are apparent (in hindsight) even at this early stage of the development of generative theory, and they prove to provoke probing questions still in the modern era.

Another argument against the EST model arises from this abstraction of the structure-building component from the interfaces. The use of arbitrary symbols as labels severely violates substantive optimality and the SMT, since there is no way of justifying and constraining label selection, inflating the role of the syntax beyond what is necessary. The model can also be shown to introduce vast amounts of computational complexity using the framework in \pxref{ex:mobbscompopt}. The fact that PS-rules must individually be crafted for every permutation of categories allowed as constitutents within a particular context entails massive reduplication of rules, violating MinRedup, and in turn clearly violating MinTC, as this large inventory of rules will need to be searched repeatedly during a derivation. Such search also entails proliferation of choice points in the derivation, necessarily leading to mass caching of incomplete derivations whilst the correct parse is found for a particular expression, hence violating MinCID, and in turn MinSC. Thus, PS-rules fail to maximise throughput (MaxTP), exacerbating the conceptual issues already discussed. In sum, whilst a theory formulated with PS-rules could plausibly meet the goal of descriptive adequacy, on account of their generative power, explanatory adequacy as formulated in terms of the MP remains a remote prospect.

