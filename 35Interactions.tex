\subsection{Labelling interactions}\label{sec:350}

Finally, there is the question of \pxref{ex:questions:interact}. The most obvious case to consider with respect to intra-syntactic interactions comes in the interaction between labelling and Agree. There emerges a clear redundancy here, since both Agree and LA are claimed to reduce to MS. As will be discussed in \autoref{sec:400}, whether this is indeed the case is not so clear, since it hinges on the precise notion of MS that is adopted.

A second, related, and arguably even more important interaction of labelling is with \emph{movement}. Movement can be licensed in two ways: (a) movement can be \textit{base-driven}; or (b) movement can be \textit{target-driven}. The now standard approach to SCM of \textcite{BoskovicZ_2007} adopts option (a)---arguably conceptually necessity in the case of SCM, since movement must begin both before the moving element is transferred and before the ultimate resting place of the moving element is merged and thus even accessible to computation. Since lower copies are invisible to the labelling algorithm, one option to resolve a labelling conflict, as suggested in \subexref{ex:popLA:b}{i}, is for one of the phrases to move. A stronger hypothesis would be that all movement is driven by labelling concerns. This is hypothesis is developed in the \textit{Generalised Dynamic Antisymmetry} (GDA) framework \parencite{MoroA.RobertsI_2020}. This would also have consequences for Agree---there appears to be a redundancy between labelling-driven movement and Agree-driven movement.

In addition to the syntax-internal considerations above, it is worth making some specific comments regarding the interaction between labelling and the interfaces. At a fundamental level, \IC\ is the reason that labelling is presumed to exist at all. Following \textcite[et seq.]{ChomskyN_2004}, the primary motivation for labels is that an SO needs a label in order to be interpreted at C-I. There are, however, a range of suggestions in the literature, some claiming that both \ICSM \emph{and} \ICCI, or only one or only the other, require or at least make use of labels when interpreting SOs \parencite{BarrieM_2021,TakitaK_2020, TakitaK.etal_2016}.

There is a further perspective which has not yet been taken in this discussion thus far but which warrants attention with discussion of this question in particular. The perspective concerns an issue which arises at all levels of syntactic theory: that of \textit{symmetry}. The two options for Merge presented in \pxref{ex:twomerges} can be reframed as the choice between \textit{asymmetric} Merge \pxref{ex:twomerges:label}, which encodes the projection of a label, and \textit{symmetric} Merge \pxref{ex:twomerges:nolabel}, which simply forms a symmetical set. It is a well-founded empirical result that asymmetries are pervasive in human language syntax. This is enshrined in the definition of \textit{c-command}, introduced by \textcite{ReinhartT_1976}.%
\footnote{\label{fn:c-command}\textcite[32]{ReinhartT_1976} defines c-command as follows: ``Node A c(onstituent)-commands B if neither A nor B dominates the other and the first branching node which dominates A dominates B''. Reinhart was not the sole progenitor of the general idea---the relation has its roots in the `in construction with' relation of \textcite{KlimaES_1964} and the `superiority' relation of \textcite{ChomskyN_1973}. There are other potential definitions of c-command, notably the derivational version proposed by \textcite{EpsteinSD.etal_1998}.}
The notion of \textit{asymmetric} c-command was also demonstrated to have the potential to derive linear order from syntactic properties by \textcite{KayneRS_1994}.
If only one of the two inputs to Merge can serve as the label, this enforces asymmetry. However, option \pxref{ex:twomerges:nolabel} enables other options, as labels may be assigned at the interface, according to interface conditions. Furthermore, as demonstrated by \textcite{MoroA_2000}, extending the analysis of \textcite{KayneRS_1994}, syntax appears to have an aversion to symmetrical surface structures.

It is also worth making some more speculative comments on how the result of labelling may have an effect upon interpretation of structures at the interfaces. That this occurs is a central empirical motivation for labelling in the first place: certain SOs must be labelling at the interfaces in order for the structure to be interpreted correctly. [Rizzi 2016 has a note on this, cf. also Donati stuff (indeterminacy of labelling leading to different possible readings at the interface)]

