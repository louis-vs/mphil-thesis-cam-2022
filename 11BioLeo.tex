\subsection{Biolinguistics and the Galilean challenge}\label{sec:110}

A major motivation of linguistics as construed in the present context is to rise to what \textcite[i.a.]{ChomskyN_2017a} formulates as the \textit{Galilean challenge}, citing a passage from Galileo's infamous \textit{Dialogo}:

\begin{quote}\setstretch{1.0}
    ``[Sagredo:] But surpassing all stupendous inventions, what sublimity of mind was his who dreamed of finding means to communicate his deepest thoughts to any other person, though distant by mighty intervals of place and time! Of talking with those who are in India; of speaking to those who are not yet born and will not be born for a thousand or ten thousand years; and with what facility, by the different arrangements of twenty characters upon a page!'' \parencite[105]{Galileo_1967}
\end{quote}

Construed narrowly, this passage refers to the alphabet, truly a human ``invention'', as opposed to endowment. Nevertheless, it would not be unfair to extrapolate from this passage, as \textcite{ChomskyN_2017a} does, a general wonder at the generative property of language, how finite knowledge can have infinite range. In the terms of the later thinker, Willhelm von Humboldt, language is characterised by the ``infinite use of finite means''. This oft-quoted aphorism is insightful within the philosophical context of modern linguistics, as explored in great detail by \textcite{ChomskyN_1966}, but it is also notable for its conflation of language knowledge and use. This Aristotelian distinction was revived in the modern generative tradition by \textcite{ChomskyN_1965}, after its occlusion within the tenets of behaviourism, in which the concept of `knowledge of language', indeed symbolic knowledge of any kind, is effectively unformulable \parencite[cf.][]{ChomskyN_1959,GallistelCR.KingAP_2010}. As outlined by \textcite[3]{ChomskyN_1986a}, this ``shift of focus [] from behaviour or the products of behaviour to states of the mind/brain that enter into behaviour'' provides the grounds for a rich research programme---that of generative grammar. Consequently, language \textit{use} will not be considered much further in this thesis, beyond the necessary empirical selection of instances of language use required to give insight into the knowledge that underlies such use. This follows the general assumptions of the generative programme that date back to \textcite{ChomskyN_1981} and earlier: ``the grammar---a certain system of knowledge---is only indirectly related to presented experience, the relation being mediated by UG [Universal Grammar, to be defined \textit{sub}---LVS]'' \parencite[4]{ChomskyN_1981}. The concern is thus with linguistic \textit{competence}, not \textit{performance} (adapting a distinction made by Saussure, \nptextcite[10]{ChomskyN_1964}).

Some further clarification of the notion of `knowledge of language', the subject matter of this thesis, is thus in order. The term `language', can be and thus far has been used in a non-technical sense, an ``informal rubric'' that allows one to ``select certain aspects of the world as a focus of inquiry'' \parencite[1]{ChomskyN_1995c}. Beyond these introductory remarks, it will be avoided in favour of more specific terms. The object of study of this thesis is language as an ``element[] of the natural world, to be studied by ordinary methods of empirical inquiry'' \parencite[1]{ChomskyN_1995c}. Furthermore, the approach to language taken is an internalist one \parencite{ChomskyN_2003}: language as a property \textit{internal} to the mind/brain of an \textit{individual}, and which is \textit{intensional}, in that it specifies a ``procedure that generates infinitely many expressions'' \parencite[263]{ChomskyN_2003}, a characterisation made plausible with the advent of the theory of computation in the 20th century, enabling the infinite to be compressed into the finite \parencite{TuringAM_1936}. The concern is thus with \textit{generation}, not production, enabling a certain precision in description abandoned by 20th century structuralist-behaviourist-empiricists, as noted by \textcite{ChomskyN_2021c} in recent remarks.

Call this naturalistic, materialist, internalist approach to language \textit{I-linguistics} \parencite[263]{ChomskyN_2003}, which concerns itself with the faculty of language (FL), an organ of the mind/brain dedicated to language, and the states it assumes---call these \textit{I-languages} \parencite[21]{ChomskyN_1986a}. I-language is thus a biological entity, necessitating study following the same principles as any other matter of biology. This forms the central tenet of \textit{biolinguistics}, a term coined by Massimo Piattelli-Palmarini (\nptextcite{MorinE.Piattelli-PalmariniM_1974}, see \nptextcite[1]{DiSciulloAM.BoeckxC_2011}) identifying a domain of study influentially formulated by \textcite[vii]{LennebergEH_1967}, namely the study of ``language as a natural phenomenon---an aspect of [man's] biological nature, to be studied in the same manner as, for instance, his anatomy''.

I-language is characterised by what has been called the \textit{Basic Property}, introduced by \textcite[\pnfmt{1}]{BerwickRC.ChomskyN_2016} and \textcite[4]{ChomskyN_2016}. The property is concisely stated as such: ``each language provides an unbounded array of hierarchically structured expressions that receive interpretations at two interfaces, sensorimotor for externalization and conceptual-intentional for mental processes'' \parencite[4]{ChomskyN_2016}. The Basic Property is arguably irreducible, a revival of Aristotle's dictum that language is ``sound with meaning'', placing considerably more emphasis on the exact nature of the ``with''. The renewed focus on the interfaces and the conditions they impose has been a hallmark of the contemporary iteration of the generative enterprise, the Minimalist Programme (MP) as set forth by \textcite{ChomskyN_1993}.\footnote{A stylistic note: when referring to the principles of and the arguments for the MP, the word \textit{Minimalism} and its morphological family will be capitalised. While we're here, it is also worth noting that single quotes `' always indicate scare quotes, and double quotes ``'' are uniformly used for quotations, with the natural caveat that quote styles are preserved within the bodies of quotations. Generally, APA 7th Edition \parencite{APA_2020} is adopted with some idiosyncracies, linguistic and personal.} This focus on the interfaces later became enshrined in the \textit{Strong Minimalist Thesis} (SMT), as in \pxref{ex:SMT}, now central to Minimalist research.

\begin{example}\label{ex:SMT}
    \textbf{Strong Minimalist Thesis (SMT)}: I-language is an optimal solution to interface conditions. \parencite[1]{ChomskyN_2001}
\end{example}

As \textcite{ChomskyN_2001} notes, the SMT only becomes an empirical thesis once the notions of `optimality' and of `interface conditions' are defined. These ideas will be explored in \autoref{sec:130} and \autoref{sec:150}, respectively. For now, it suffices to state that this thesis is a biolinguistic one, insofar as it concerns I-language and its place within the mind/brain.

As set out by \textcite{MobbsI_2015}, the \textit{argument of linguistic Minimalism} is actually ``a collection of related, but logically independent, proposals [that have] coalesced in the literature of the past twenty-five odd years'' \parencite[1]{MobbsI_2015}. The five proposals identified by Mobbs are summarised in \pxref{ex:minprops}.

\begin{subexamples}[preamble={\textbf{The Minimalist Proposals} \parencite{MobbsI_2015}}]\label{ex:minprops}
    %\setstretch{1.0}
    %\setlength{\itemsep}{1pt}
    %\setlength{\parskip}{0pt}
    %\setlength{\parsep}{0pt}
    \item\label{ex:minprops:1} \textbf{Methodological minimalism}: When faced with two empirically equivalent proposals, ``we should adopt the more parsimonious explanation, that is, the one containing fewer ancillary claims'' \parencite[38]{MobbsI_2015}.
    \item\label{ex:minprops:2} \textbf{Ontological minimalism}: assume the SMT, ``a refusal on the part of the scientist to make pre-theoretic assumptions about the design or `purpose' of language'' \parencite[41]{MobbsI_2015}.
    \item\label{ex:minprops:3} \textbf{SMT and evo-devo}: ``certain facts about language cannot clearly be explained in terms of optimality, and we are forced to propose further innate competence'' \parencite[46]{MobbsI_2015}; this innate competence should be constrained by evolutionary-developmental concerns.
    \item\label{ex:minprops:4} \textbf{The primacy of the CI interface}: ``language is optimized for the system of thought, with mode of externalization secondary'' \parencite[32]{BerwickRC.ChomskyN_2011}.
    \item\label{ex:minprops:5} \textbf{Variation}: ``[p]arameterization and diversity [are] mostly---possibly entirely---restricted to externalization'' \parencite[37]{BerwickRC.ChomskyN_2011}.
\end{subexamples}

The first three proposals collectively comprise the MP, the fourth and fifth proposals are ``specific claims about the design and origin of FL'' \parencite[2]{MobbsI_2015}. They stand logically separate, and need not all be assumed within a Minimalist work. For instance, much work in the comparative syntax tradition does not make assumption \pxref{ex:minprops:4}; rather, the parameters along which different I-languages may vary, encoded as formal features, remain a focus of investigation \parencite[see]{RobertsI_2019,SheehanM.etal_2017,BiberauerT.etal_2010}.

The first proposal, methodological minimalism, is the least controversial, but perhaps also most important. It is effectively a reformulation of \textit{Occam's Razor}, the principle that ``[w]e may assume the superiority \textit{ceteris paribus} of the demonstration which derives from fewer postulates or hypotheses'' (Aristotle, \textit{Posterior Analytics}, cited by \nptextcite{BakerA_2016}; emphasis original). Appropriately, Galileo also adopted the principle: ``it is said that Nature does not multiply things unnecessarily; that she makes use of the easiest and simplest means for producing her effects; that she does nothing in vain, and the like'' \parencite[397]{Galileo_1967}. Simplification must come with justification, however. \textcite[13]{ChomskyN_1981} notes this explicitly: ``it is evident that a reduction in the variety of systems in one part of the grammar is no contribution to these ends if it is matched or exceeded by proliferation elsewhere'' ... ``[i]t is only when a reduction in one component is not matched or exceeded elsewhere that we have reason to believe that a better approximation to the actual structure of mentally-represented grammar is achieved''. In light of methodological minimalism, Chomsky's objection here goes both ways: one must be perpetually wary of the power of the explanatory devices introduced within a theoretical framework, in other words, to avoid ``the temptation to offer a purported explanation for some phenomenon on the basis of assumptions that are of roughly the order of complexity of what is to be explained'' \parencite[233]{ChomskyN_1995}. As a methodological guideline, this serves of vital importance, especially in the discussion to follow in \autoref{sec:200} and \autoref{sec:300}, and in the formalisation itself in \autoref{sec:400}. Different construals of labelling have vastly different implications in terms of complexity (theoretical and computational) and need to be considered within the broader framework of the Minimalist theories they are a part of. Methodological minimalism also serves as a reminder of the consequences of introducing richer theoretical devices, by encouraging a holistic approach, in which the overally complexity of the theory is under constant scrutiny.

The Minimalist principles, now fully established, are thus very useful in the study of biolinguistics. To demonstrate but one example, take again the first Minimalist proposal \pxref{ex:minprops:1}. A perpetual problem of biolinguistics which is perhaps most clearly expressed by \textcite{PoeppelD.EmbickD_2005,EmbickD.PoeppelD_2015} concerns the interrelation of abstract theories of I-language with the neurobiological models that are supposed to implement the algorithms proposed in theories such as that of this thesis. Known as the `granularity problem', the question arises as to whether it is possible to reduce the often complex and theoretically rich proposals within the linguistics literature---and arguably cognitive science more generally---to the simple constructs of neurobiology. There are a number of ideas adopted within the biolinguistic literature to make this chasm of separation less ominous. Firstly, the adoption of methodological minimalism encourages the reduction of theoretical complexity, ultimately making it more likely that such a model could bridge the explanatory gap. Another set of proposals which will provide some useful framing to the present work are the three computational levels for information processing systems proposed by \textcite[25]{MarrD_1982}, adapted from \textcite{MarrD.PoggioT_1976}, as shown in \pxref{ex:marr}.

\begin{subexamples}\label{ex:marr}
    \item\label{ex:marr:1} \textbf{Computational level}: What does the computation do, and what are its goals?
    \item\label{ex:marr:2} \textbf{Algorithmic/representational level}: How are the inputs and outputs of the computation represented, and what is the algorithm that computes said outputs? 
    \item\label{ex:marr:3} \textbf{Physical/implementational level}: How are the representations and algorithms encoded physically?
\end{subexamples}

Though he was working on human vision, Marr's motivation for proposing these levels of analysis is analogous to the reasoning employed by \textcite{ChomskyN_1964,ChomskyN_1965} in devising the metric of \textit{explanatory adequacy} to evaluate a linguistic theory, later adapted into MP as the principle of \textit{genuine explanation} \parencite{ChomskyN_1993,ChomskyN_1995}. Specifically, \textcite[15]{MarrD_1982} notes that ``neurophysiology and psychophysics have as their business to \textit{describe} the behaviour of cells or of subjects but not to \textit{explain} such behaviour''. Indeed, finding anything at all interesting from observational or descriptive work (again, generalising the definitions of these terms given by \nptextcite{ChomskyN_1964}) is if anything surprising: ``[i]f one probes around in a conventional electronic computer and records the behaviour of single elements within it, one is unlikely to be able to discern what a given element is doing'' \parencite[14]{MarrD_1982}. It should thus be surprising that work along these lines in neurobiology has indeed had a lot of success. In vision, as in linguistics (in the form of the descriptive success of the structuralists, and of early investigation in neurolinguistics), some sense of progress had thus been made along such lines, but these results fall short of being explanatory when this is defined in a principled sense. In sum, ``understanding computers [equivalently, human brains---LVS] is different from understanding computations'' \parencite[5]{MarrD_1982}. This is where proper consideration of Marr's levels of analysis becomes particularly revealing, offering a positive way out, in which study of neurobiology is not the be-all-and-end-all, but rather an analysis of one aspect of a computational system. Investigation of the computational and algorithmic levels is arguably of equal importance, both to constrain analysis of the more fine-grained phenomena and also, more generally, to gain a holistic understanding of the system, and on a meta level to constrain our conception of what we can even begin to learn about the system in the first place.

Beyond the introduction, the implementational level will receive no further attention. It is hoped, however, that by clarifying the computational level and providing at least tentative steps towards an algorithmic representation, the explanatory gap between linguistics and neuroscience can be reduced, as should be a major objective of biolinguistics, which emphasises the building of interdisciplinary bridges \parencite{DiSciulloAM.BoeckxC_2011, BoeckxC_2013, WatumullJ_2013}. It could be argued that the biolinguistics programme has been circumspect since its inception in its historical focus on the computational level. As \textcite[2]{MarrD.PoggioT_1976} state, ``although the top [computational---LVS] level is the most neglected, it is also the most important[, because] the structure of the computations that underly perception depend more upon the computational \textit{problems} that have to be solved than on the particular hardware in which their solutions are implemented'' (emphasis original). This view of computational neuroscience is revisited by \textcite{GallistelCR.KingAP_2010}, who repeatedly emphasise that understanding general computational theory, as well as the specific instantiations of computations that must be taking place in order for an organism to do certain things at all, should indeed preclude analysis of the neurobiological mechanisms that could plausibly implement such computations. This `top-down' approach thus equates to a stronger thesis than Marr's. Another important aspect of the research programme advanced by \textcite{GallistelCR.KingAP_2010} is that, as a logical consequence of adopting a computational model, the \textit{symbols} making up the mental representations that are manipulated by computational procedures become of vital significance. Meanwhile, the implementational details of such representations is not inherently necessary to understanding the computation, as the three level model would indeed suggest. As \textcite{GallistelCR_2001} outlines concisely: ``[w]hat matters in representations is form, not substance''. Further, it is these mental representations which should be formalised as part of the algorithmic level, following the definition provided in \pxref{ex:marr:2}.

This focus on the computational level in search of reducing the explanatory gap further relates to a point raised by \textcite[187]{ChomskyN_1994a}: it is often the case in the history of science that the ``more `fundamental' science has had to be revised'' in order that the different levels of analysis be unified. One example given is of the dawn of quantum physics by the 1930s, which allowed theories of chemistry that failed to fit in with the classical model to be explained. In the case of language, the current state of knowledge in neuroscience is very far from being able to give any real explanation as to how I-language is implemented \parencite[cf.][]{GallistelCR.KingAP_2010}. An understanding of the computations, and in following the algorithms, can be sought without overreliance on the implementation. Thus, I adopt a more tempered stance on the biolinguistic programme than \textcite{MartinsPT.BoeckxC_2016}, who effectively insist that implementational details are required for work to be characterised as biolinguistic.

Another useful heuristic that applies to the investigation of any biological system is the `triple helix' model as proposed by \textcite{LewontinRC_2000}. An organism and its components cannot solely be explained through genetics, and the same is evidently true for I-language. Rather, one must, alongside the gene, consider the development of the organism itself, and the pressures and constraints of the environment. Analogously, \textcite{ChomskyN_2005} adapts this notion to I-language, proposing the \textit{three factor model}. The first factor is the \textit{genetic endowment}, assumed to be effectively uniform for all humans, and which allows us to interpret part of our environment as linguistic and construct grammars. The theory of this factor can be termed \textit{Universal Grammar} (UG). The `domain-specificity' of UG has come under considerable scrutiny since its inception \parencite[e.g.][]{TomaselloM_2003}, but in reality the point is somewhat moot. The first factor must \textit{a priori} play a role, since there must be something that enables a human child to acquire language, uniquely within the animal kingdom. This is a weak hypothesis, but it need not be much stronger, when approaching I-language from the perspective of the Basic Property, or in earlier terms ``from bottom up'' \parencite[4]{ChomskyN_2007}. Secondly, factor two is the environmental factor, the \textit{Primary Linguistic Data} \parencite[PLD;][10]{ChomskyN_1981}. The interaction and tension between the first and second factors forms the fundamental basis of the argument from the poverty of the stimulus \parencite{ChomskyN_1980b, ChomskyN_1986a} and also the well-known tension between descriptive and explanatory theoretical adequacy \parencite{ChomskyN_1964,ChomskyN_1965}. More specifically, the primitives provided by the genetic endowment should be sufficient to map the child's intake to the PLD, in other words to interpet the data available to the child into linguistic terms---the notion of \textit{epistemological priority} \textcite[10]{ChomskyN_1981}. As such, the PLD constitutes \textit{intake} as opposed to \textit{input}, following the terminology used by \textcite{EversA.vanKampenJ_2008}. Finally, there is the third factor, effectively a catch-all term encompassing the external, non-language-specific constraints imposed by nature. As stated by \textcite[6]{ChomskyN_2005}, the most important third factors to consider in the case of I-language are ``principles of structural architecture and developmental constraints'', in particular ``principles of efficient computation, which would be expected to be of particular significance for computational systems such as language''.

As an aside, it is worth noting that this model was made explicit at least as early as \textcite[105]{ChomskyN_2004} and, as noted by \textcite[xi]{ChomskyN_2006}, the presence of third factors has been recognised since the genesis of biolinguistics, whilst the development of MP merely made the scale of such concerns more apparent. The exact position is stated already by \textcite[59]{ChomskyN_1965}: ``there is surely no reason today for taking seriously a position that attributes a complex human achievement entirely to months (or at most years) of experience [=2nd factor---LVS], rather than to millions of years of evolution [=1st factor---LVS], or to principles of neural organization that may be more deeply grounded in physical law [=3rd factor---LVS]''. MP and the three factor model are thus a natural step to take in search of the answers to the Galilean challenge.

What is however less often appreciated, as noted by \textcite{BoeckxC_2014a}, is that the critical insight of the triple helix model as originally conceived by \textcite{LewontinRC_2000} is in the \emph{interaction} between the factors. It is thus arguably incoherent to talk about `factors' in isolation, for instance, by conceiving of some or another proposed linguistic module as `belonging to the third factor', as has become somewhat commonplace \parencite[see][]{GallegoAJ_2011}. Indeed, due to the maximally general definition of `third factors', this is arguably not a characterisation that carries much theoretical content. As such, an understanding of the interaction between the factors will be explicitly pursued here. Another example of such an approach is offered by \textcite{RobertsI_2019} in the context of understanding parameters in a Minimalist context, the hypothesis being that the emergence of parameters in the acquisition process can be explained by utilising a combination of all three factors.

These issues are of particular relevance within the discussion of labelling. As will become clear as a central theme of \autoref{sec:200} and \autoref{sec:300}, the somewhat precarious position of syntactic labels since the advent of Bare Phrase Structure \parencite{ChomskyN_1994} has led to a sizeable push towards their relegation into being a third factor constraint: ``The simplest assumption is that LA [the labelling algorithm---LVS] is just minimal search, presumably appropriating a third factor principle'' \parencite[43]{ChomskyN_2013}. The following two subsections will seek to clarify further exactly what ``simplest'' should mean in this context. Furthermore, Boeckx's aforementioned objection will be adopted in what follows, emphasising the points of \emph{interaction} between factors. In the context of labelling, these will predominantly be the first and third factors: how does ``minimal search'' (MS) interact with (constrain, enable) the computational provisions of the genetic endowment in deriving linguistic structures?

A final useful heuristic to keep in mind is the three-way distinction between \textit{metatheory}, \textit{theory} and \textit{analytic} work, as first deconstructed by \textcite{ChametzkyRA_1996}. Metatheoretical work could be considered as much a philosophical endeavour as a linguistic one: it defines the evaluation of theories, as well as how they ought to be constructed, and hence is the focus of this introduction. Theoretical work, on the other hand, is the ``deployment of metatheoretical concepts'' \parencite[xviii]{ChametzkyRA_1996}---the construction of primitives and theorems derived from them. This is finally complemented by analytic work, which involves direct investigation of the phenomena in question. By applying theories to data, they can be tested and refined. Analytic enquiry is, without doubt, the most important, and rightly most time-consuming area of linguistics---as \textcite[xviii]{ChametzkyRA_1996} states, ``[f]or linguistics to be the science of language, this must be where linguists do their work''. This being said, for analytic work to be productive it needs a sufficiently well-defined formal framework.  Furthermore, it must fall in line with the metatheoretical aims of linguistic enquiry, in order to ensure that the data being studied actually have an evidential relation with the theory being tested. Following the introduction, this thesis is firmly theoretical, operating at the computational level of analysis in \autoref{sec:200} and \autoref{sec:300}, progressing closer to the algorithmic level in \autoref{sec:400}.

To summarise the crucial metatheoretical concerns, and following \textcite[7]{ChomskyN_2021}, it is clear that UG must meet three contradictory conditions:

\begin{subexamples}
    \item\label{def:learnability} \textbf{Learnability}: UG (in conjunction with third factor principles) must be rich enough to enable language acquisition, overcoming the poverty of the stimulus.
    \item\label{def:evolvability} \textbf{Evolvability}: UG must be simple enough to have plausibly emerged under the conditions of human evolution.
    \item\label{def:universality} \textbf{Universality}: UG, as the name implies, is universal to all humans.
\end{subexamples}

\pxref{def:learnability} and \pxref{def:evolvability} evidently contradict each other: the theory must be both rich enough that a child has the resources to acquire the language(s) of their environment, but refined enough such that it could have evolved in a relatively short space of time (cf. \nptextcite{BerwickRC.ChomskyN_2011,BerwickRC.ChomskyN_2016}, for more precise discussion of the evolutionary context, and \nptextcite{BalariS.LorenzoG_2012} for a different approach). Satisfying learnability and evolvability is the ``austere requirement'' that constitutes the bare minimum for ``genuine explanation'' \parencite[14]{ChomskyN_2020a}. \pxref{def:universality} is another way of stating the \textit{Uniformity Principle} of \textcite[2]{ChomskyN_2001}: ``assume languages to be uniform, with variety restricted to easily detectable properties of utterances''. It can also be restated as the typological problem: why do languages appear to vary so much on the surface, and how is this variation constrained? This is effectively the problem solved with Chomsky's Principles and Parameters model (cf. \nptextcite{ChomskyN.LasnikH_2015}, for an overview), but this model needs to be adapted from its early, Government-Binding form \parencite{ChomskyN_1981} into something more obviously compatible with Minimalism. Some notes on this, with particular reference to how labelling may play a role, are provided in \autoref{sec:150}.
