\subsection{Interfaces, variation and variability}\label{sec:150}

It is worth breifly expanding upon \autoref{sec:142} in order to provide some more detail on the precise status of the `interfaces' in contemporary Minimalist theory, in particular with respect to features.

The concept of `features' is elaborated in \autoref{sec:143}. Following the Borer-Chomsky Conjecture \parencite{BakerMC_2008}, all linguistic variation---equivalently, everything that is learned; the second factor---is restricted to the lexicon, and hence to the arrangement of lexical features within the lexicon, and their treatment by I-language. One could question exactly what these features are---some properties, like interpretability, have already been discussed. One may further question how distinct the difference sets of features (phonological, semantic, and syntactic) are. Standardly, features come in interpretable-uninterpretable (or valued-unvalued) pairs, where interpretability is an interface property, as will be assumed in \autoref{sec:460}. Nevertheless, some authors argue for a `substance-free' system, adopting the term introduced by \textcite{HaleM.ReissC_2008}: a system is \textit{substance-free} if it involves ``computation over abstract mental entities'' \parencite[22]{HaleM.ReissC_2008}, in other words being \textit{symbolic} in the sense applied to cognitive science by \textcite{PylyshynZW_1984,GallistelCR_2001,GallistelCR.KingAP_2010}, inter alia. By contrast, a theory in which `symbols' are actually embodied in percepts would be considered \textit{substanceful}. For example, a phonological theory in which phonemes directly correspond to aspects of phonetic substance, as in standard generative phonology \parencite{ChomskyN.HalleM_1968}, is substanceful. In the present case, a syntactic theory in which categories `leech' off of semantic properties would be considered substanceful. \textcite{ZeijlstraH_2014} argues for this approach in syntax, although a ful discussion of this would take us too far afield.

Classic formulations allow the label of a syntactic object to be a (categorial) feature, a bundle of features, or a lexical item, and in more recent approaches more complex objects are allowed to serve as labels (see \autoref{sec:300}). As such, an understanding of the feature inventory \setF\ will be essential to formalising the answer to the question of what can be a label. The question of what (kinds of) features are allowed thus has direct bearing on the topic of this thesis. However, adopting the BCC, as typically done in Minimalist work, entails that features also have a significant impact on the class of humanly computable I-languages.%
\footnote{Using the more precise term introduced by \textcite[3]{HaleM.ReissC_2008}, as opposed to a weaker alternative like `possible languages'. Adapting ideas originating in Evolutionary Phonology \parencite{BlevinsJ_2004} to I-language generally, the set of possible I-languages may be irrecovably restricted by historical, cultural, and anthropological factors of a very different nature to the first and third factors considered here to go into a theory of I-language. This is, effectively, a truism that emerges upon consideration of the second factor: the data of the environment have no \textit{a priori} justification to be the way they are except for the fact that they are \textit{a priori} constrained by first and third factors. A different course of history could have led to there being a completely mutually exclusive set of possible languages available to the linguist to study and the child to learn throughout time, but if a child from our world were to travel to this hypothetical world, we still want to say that they could learn the language. Hence the concept of \textit{humanly computable}.}
Thus, the question of language variation---in effect, rephrasing \pxref{def:universality}---is unavoidable, but also inevitable in a feature-based theory. These issues will be explored further in the subsequent discussion.

Another aspect of the theory presented in \autoref{sec:140} is that \MERGE\ can apply freely. There seem to be a number of good reasons, theory-internal but formalisation-external, to adopt the free Merge approach. The approach receives particular justification by \textcite{ChomskyN.etal_2019}, and further reasons crop up on occasion in the following discussion. The general point is that `overgeneration', traditionally thought of as the bane of a sound theory, is actually good, if considered in a restricted manner. Namely, it allows `deviant' structures to be generated, which perhaps satisfy constraints at one interface, but are to some variable degree uninterpretable at another. This brings the theory of I-language more in line with general empirical observations, in which grammaticality is a gradient property of expressions \parencite{SprouseJ.etal_2018}.
